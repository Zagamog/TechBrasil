\documentclass[a4paper,10pt]{article}
\usepackage[utf8]{inputenc}
\usepackage[left=0.5in, right=0.5in, top=0.8in, bottom=0.8in]{geometry}
\usepackage{enumitem}

\setlist[itemize]{topsep=0pt, partopsep=0pt, parsep=0pt} 

\begin{document}

\section*{Painel 1: A estratégia do programa `Juros pela Educação'}

\begin{table}[htbp!]
\centering
\renewcommand{\arraystretch}{1.2}
\begin{tabular}{|p{1in}|p{1.4in}|p{4in}|}
\hline
Name & Assignation & Summary Bullets \\
\hline
Ana Inoue & Itaú Educação e Trabalho & \begin{itemize}
\item Explained how the program aligns with workforce development strategies.
\item Stressed the importance of vocational training and lifelong learning.
\item Outlined potential partnerships between financial institutions and education providers.
\end{itemize}\\
\hline
Priscila Cruz & Todos Pela Educação & \begin{itemize}
\item Advocated for equitable access to educational financing.
\item Addressed challenges in implementing the program at a national scale.
\item Called for data-driven policies to measure educational impact.
\end{itemize}\\
\hline
Camilo Santana & Ministro da Educação & \begin{itemize}
\item Presented the government’s commitment to funding and expanding the initiative.
\item Discussed federal policies supporting financial aid for students, emphasizing new financing mechanisms.
\item Highlighted the role of state-federal partnerships to ensure effective fund allocation and program sustainability.
\item Addressed the need for improving administrative efficiency in distributing educational resources.
\item Outlined the importance of equity-focused approaches to reach underprivileged communities.
\end{itemize}\\
\hline
João Azevedo & Governador do Estado da Paraíba & \begin{itemize}
\item Shared state-level experiences in implementing similar financial programs, detailing Paraíba’s initiatives.
\item Explored best practices from other states to address educational inequality through targeted funding.
\item Discussed regional disparities in education funding and proposed solutions tailored to specific socioeconomic contexts.
\item Emphasized the role of local governments in ensuring educational investments yield long-term benefits.
\item Highlighted successful case studies from Paraíba’s education system as models for national implementation.
\end{itemize}\\
\hline
Dário Durigan & Secretário Executivo do Ministério da Fazenda & \begin{itemize}
\item Explained the fiscal strategies underpinning the program’s funding, ensuring long-term financial viability.
\item Discussed budgetary allocations and financial sustainability concerns, considering economic constraints.
\item Outlined potential tax incentives for institutions participating in the program to encourage investment.
\item Proposed mechanisms for monitoring and evaluating the financial performance of the initiative.
\item Highlighted the importance of private sector involvement in supplementing public educational investments.
\end{itemize}\\
\hline
\end{tabular}
\end{table}

\newpage


\section*{Painel 2: O que é a proposta `Juros pela Educação', ajustes e detalhes técnicos}

\begin{table}[htbp!]
	\centering
	\renewcommand{\arraystretch}{1.2}
	\begin{tabular}{|p{1in}|p{1.4in}|p{4.2in}|}
		\hline
		Name & Assignation & Summary Bullets \\
		\hline
		Murilo Camaroto & Repórter do Valor Econômico & \begin{itemize}
			\item Asked about the most significant adjustments made to the 'Juros pela Educação' program in recent weeks.
			\item Inquired about the technical feasibility and long-term sustainability of the proposed financing mechanism.
		\end{itemize}\\
		\hline
		Fernando Exman & Chefe da Sucursal do Valor Econômico em Brasília & \begin{itemize}
			\item Questioned how the program aligns with broader economic policies and fiscal constraints.
			\item Probed into the expected impact of the initiative on private sector engagement in education funding.
		\end{itemize}\\
		\hline
		Gregório Grisa & Secretário Executivo-Adjunto do Ministério da Educação & \begin{itemize}
			\item Explained the rationale behind recent adjustments to the 'Juros pela Educação' program, emphasizing the balance between ensuring broad access to financing and maintaining financial sustainability.
			\item Discussed mechanisms to ensure fair access to financing, particularly for low-income students and underfunded institutions, including interest rate reductions and targeted financial aid programs.
			\item Highlighted how the program integrates with existing federal education policies, aiming to create a long-term strategy that supports both student success and national economic growth.
			\item Addressed concerns about administrative bottlenecks, outlining steps being taken to streamline application and fund distribution processes, ensuring efficiency and transparency.
			\item Emphasized the importance of a continuous evaluation framework, where data analytics and impact assessments will be used to refine and improve the financing model over time.
		\end{itemize}\\
		\hline
		Rogério Ceron & Secretário do Tesouro Nacional do Ministério da Fazenda & \begin{itemize}
			\item Outlined the Treasury’s approach to ensuring long-term program funding while maintaining fiscal responsibility, detailing how the initiative aligns with broader government budget priorities.
			\item Explained the mechanisms through which the government will subsidize interest rates without compromising macroeconomic stability, ensuring predictability in public accounts.
			\item Discussed the financial safeguards in place to prevent excessive fiscal exposure, including expenditure ceilings and mechanisms for adjusting program parameters based on economic conditions.
			\item Clarified the role of the National Treasury in coordinating fund distribution with public financial institutions, ensuring compliance with constitutional spending limits.
			\item Highlighted the importance of accountability and transparency in managing public funds for education, detailing how periodic audits and reporting mechanisms will ensure efficient resource allocation.
		\end{itemize}\\
		\hline
	\end{tabular}
\end{table}

\newpage

\section*{Painel 3: A proposta sobre a ótica da educação profissional nos Estados}

\begin{table}[htbp!]
	\centering
	\renewcommand{\arraystretch}{1.2}
	\begin{tabular}{|p{1in}|p{1in}|p{4.6in}|}
		\hline
		Name & Assignation & Summary Bullets \\
		\hline
		Murilo Camaroto & Repórter do Valor Econômico & \begin{itemize}
			\item Inquired about the biggest implementation challenges for financing technical education through this initiative.
			\item Probed into the alignment of professional education policies with labor market demands in different states.
		\end{itemize}\\
		\hline
		Fernando Exman & Chefe da Sucursal do Valor Econômico em Brasília & \begin{itemize}
			\item Questioned how the fiscal constraints of each state impact their ability to expand professional education.
			\item Explored how states can maintain financial sustainability while increasing access to technical training programs.
		\end{itemize}\\
		\hline
		Roni Miranda & Secretário de Educação do Estado do Paraná / CONSED & \begin{itemize}
			\item Discussed Paraná’s approach, emphasizing alignment with industry demands and technological advancements.
			\item Highlighted partnerships between public institutions and local industries, explaining how co-designed curricula with businesses ensure students develop job-ready skills.
			\item Explained how the state ensures that students from all socioeconomic backgrounds can access technical training programs, detailing the role of scholarships and government-subsidized tuition.
			\item Emphasized the importance of expanding dual education models, where students split time between classroom learning and apprenticeships in real work environments.
			\item Discussed data-driven decision-making in education policy, using labor market analytics to adjust training programs to employment needs.
		\end{itemize}\\
		\hline
		Fátima Gavioli & Secretária de Educação do Estado de Goiás & \begin{itemize}
			\item Outlined Goiás’ investment priorities in vocational education, focusing on expanding enrollment and infrastructure, particularly in underserved rural areas.
			\item Described strategies to tailor curricula to regional economic strengths, particularly in agribusiness, technology, and service industries, ensuring students are prepared for high-demand jobs.
			\item Emphasized the importance of ongoing teacher training programs to maintain the quality of professional education, ensuring instructors stay updated on industry trends and evolving pedagogical methods.
			\item Explained the implementation of performance monitoring systems to evaluate the success of vocational training graduates in the labor market and adjust programs accordingly.
			\item Addressed budget constraints and how Goiás is leveraging federal funding, private sector collaboration, and efficiency measures to expand vocational education without compromising quality.
		\end{itemize}\\
		\hline
		Guilherme Lichand & Professor da Universidade de Stanford & \begin{itemize}
			\item Presented an economic analysis of the proposed reforms, emphasizing the long-term cost-benefit ratio of investing in technical education, showing how a skilled workforce leads to GDP growth.
			\item Reaching the OECD benchmark of 37\% technical education enrollment would require an estimated R\$50 billion in additional funding over six years.
			\item Highlighted evidence from international models, demonstrating that well-funded vocational programs can boost employment rates by up to 20\% in relevant sectors, particularly in STEM and healthcare fields.
			\item Explained the projected return on investment for government spending in vocational training, estimating that every R\$1 invested in technical education results in R\$3–4 in economic output over a decade.
		\end{itemize}\\
		\hline
	\end{tabular}
\end{table}


\newpage

\section*{Painel 4: A proposta sobre a ótica das Finanças Públicas nos Estados}

\begin{table}[htbp!]
	\centering
	\renewcommand{\arraystretch}{1.2}
	\begin{tabular}{|p{1.2in}|p{1.6in}|p{4.2in}|}
		\hline
		Name & Assignation & Summary Bullets \\
		\hline
		Fernando Exman & Jornalista do Valor Econômico & 
		\begin{itemize}
			\item Asked how state governments are adjusting their fiscal policies to accommodate the costs of the 'Juros pela Educação' program while maintaining overall budget stability.
		\end{itemize}\\
		\hline
		Lu Aiko & Jornalista do Valor Econômico & \begin{itemize}
			\item Raised concerns about long-term financial sustainability and potential budgetary trade-offs required to maintain the program.
		\end{itemize}\\
		\hline
		Felipe Salto & Economista-chefe da Warren Investimentos / Ex-Secretário da Fazenda de São Paulo & \begin{itemize}
			\item Presented a detailed fiscal analysis of the proposed education financing reforms, emphasizing the impact on state budgets over the next decade.
			\item Cited projections  suggesting the impossibility of the reform, for the budget to move from 2\% deficit to 3.5\% surplus equivalent to about 500 billion reais annually on GDP of 11 trillion reais. 
			\item Highlighted that states with stronger fiscal health—such as São Paulo and Paraná—could absorb the costs more easily, while others with high debt ratios, such as Rio de Janeiro and Rio Grande do Sul, might struggle without federal intervention.
			\item Discussed potential sources of additional revenue, including tax reforms and efficiency measures, to free up resources for education funding.
			\item Used a comparative approach, referencing OECD data to show that Brazil lags behind in public education spending relative to economic output, needing at least an additional R\$30 billion annually to close the gap.
		\end{itemize}\\
		\hline
		Carlos Xavier & Presidente do Comsefaz / Secretário de Tributação do Rio Grande do Norte & \begin{itemize}
			\item Explained how state tax revenue distribution impacts the ability of different regions to finance educational reforms, emphasizing disparities between wealthier and poorer states.
			\item Discussed the role of the Comsefaz in negotiating a fairer fiscal framework to ensure equitable funding for all states, highlighting recent legislative proposals.
		\end{itemize}\\
		\hline
		Luis Claudio Gomes & Secretário de Estado de Fazenda de Minas Gerais & \begin{itemize}
			\item Outlined Minas Gerais’ financial constraints in expanding education funding, citing the state’s ongoing debt renegotiation efforts with the federal government.
			\item Emphasized the importance of multilateral agreements between states and Brasília to secure long-term financial sustainability for education policies.
		\end{itemize}\\
		\hline
		Vilma Pinto & Diretora da Instituição Fiscal Independente (IFI) do Senado Federal & \begin{itemize}
			\item Provided an independent fiscal perspective on the proposed reforms, warning that states with high fiscal deficits may face additional borrowing constraints if not properly managed.
			\item Discussed the need for transparency in state budget planning to prevent future financial imbalances caused by new education-related expenditures.
		\end{itemize}\\
		\hline
	\end{tabular}
\end{table}



\end{document}
