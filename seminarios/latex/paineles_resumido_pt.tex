\documentclass[a4paper,10pt]{article}
\usepackage[utf8]{inputenc}
\usepackage[left=0.2in, right=0.2in, top=0.3in, bottom=0.3in]{geometry}
\usepackage{enumitem}

\setlist[itemize]{topsep=0pt, partopsep=0pt, parsep=0pt} 

\begin{document}
\section*{Painel 1: A estratégia do programa `Juros pela Educação'}

\begin{table}[htbp!]
	\centering
	\renewcommand{\arraystretch}{1.2}
	\begin{tabular}{|p{1in}|p{1.4in}|p{4in}|}
		\hline
		Nome & Atribuição & Resumo dos Pontos \\
		\hline
		Ana Inoue & Itaú Educação e Trabalho & \begin{itemize}
			\item Explicou como o programa se alinha às estratégias de desenvolvimento da força de trabalho.
			\item Enfatizou a importância do treinamento profissional e da aprendizagem ao longo da vida.
			\item Esboçou possíveis parcerias entre instituições financeiras e provedores de educação.
		\end{itemize}\\
		\hline
		Priscila Cruz & Todos Pela Educação & \begin{itemize}
			\item Defendeu o acesso equitativo ao financiamento educacional.
			\item Abordou os desafios da implementação do programa em escala nacional.
			\item Reivindicou políticas baseadas em dados para medir o impacto educacional.
		\end{itemize}\\
		\hline
		Camilo Santana & Ministro da Educação & \begin{itemize}
			\item Apresentou o compromisso do governo com o financiamento e a expansão da iniciativa.
			\item Discutiu políticas federais de apoio à assistência financeira para estudantes, enfatizando novos mecanismos de financiamento.
			\item Destacou o papel das parcerias entre estados e governo federal para garantir uma alocação eficaz de recursos e a sustentabilidade do programa.
			\item Abordou a necessidade de melhorar a eficiência administrativa na distribuição de recursos educacionais.
			\item Esboçou a importância de abordagens focadas em equidade para alcançar comunidades desfavorecidas.
		\end{itemize}\\
		\hline
		João Azevedo & Governador do Estado da Paraíba & \begin{itemize}
			\item Compartilhou experiências estaduais na implementação de programas financeiros semelhantes, detalhando as iniciativas da Paraíba.
			\item Explorou melhores práticas de outros estados para combater a desigualdade educacional por meio de financiamento direcionado.
			\item Discutiu disparidades regionais no financiamento da educação e propôs soluções adaptadas a contextos socioeconômicos específicos.
			\item Enfatizou o papel dos governos locais na garantia de que os investimentos educacionais gerem benefícios a longo prazo.
			\item Destacou estudos de caso bem-sucedidos do sistema educacional da Paraíba como modelos para implementação nacional.
		\end{itemize}\\
		\hline
		Dário Durigan & Secretário Executivo do Ministério da Fazenda & \begin{itemize}
			\item Explicou as estratégias fiscais que sustentam o financiamento do programa, garantindo sua viabilidade financeira a longo prazo.
			\item Discutiu alocações orçamentárias e preocupações com a sustentabilidade financeira, considerando as restrições econômicas.
			\item Esboçou possíveis incentivos fiscais para instituições que participam do programa, incentivando investimentos.
			\item Propôs mecanismos para monitorar e avaliar o desempenho financeiro da iniciativa.
			\item Destacou a importância do envolvimento do setor privado para complementar os investimentos públicos na educação.
		\end{itemize}\\
		\hline
	\end{tabular}
\end{table}


\newpage


\section*{Painel 2: O que é a proposta `Juros pela Educação', ajustes e detalhes técnicos}
\begin{table}[htbp!]
	\centering
	\renewcommand{\arraystretch}{1.2}
	\begin{tabular}{|p{1in}|p{1.4in}|p{4.2in}|}
		\hline
		Nome & Atribuição & Resumo dos Pontos \\
		\hline
		Murilo Camaroto & Repórter do Valor Econômico & \begin{itemize}
			\item Perguntou sobre os ajustes mais significativos feitos no programa 'Juros pela Educação' nas últimas semanas.
			\item Indagou sobre a viabilidade técnica e a sustentabilidade a longo prazo do mecanismo de financiamento proposto.
		\end{itemize}\\
		\hline
		Fernando Exman & Chefe da Sucursal do Valor Econômico em Brasília & \begin{itemize}
			\item Questionou como o programa se alinha com políticas econômicas mais amplas e restrições fiscais.
			\item Investigou o impacto esperado da iniciativa na participação do setor privado no financiamento da educação.
		\end{itemize}\\
		\hline
		Gregório Grisa & Secretário Executivo-Adjunto do Ministério da Educação & \begin{itemize}
			\item Explicou a justificativa por trás dos ajustes recentes no programa 'Juros pela Educação', enfatizando o equilíbrio entre garantir amplo acesso ao financiamento e manter a sustentabilidade financeira.
			\item Discutiu mecanismos para assegurar o acesso equitativo ao financiamento, especialmente para estudantes de baixa renda e instituições subfinanciadas, incluindo redução de taxas de juros e programas de auxílio financeiro direcionado.
			\item Destacou como o programa se integra às políticas educacionais federais existentes, visando criar uma estratégia de longo prazo que apoie tanto o sucesso estudantil quanto o crescimento econômico nacional.
			\item Abordou preocupações sobre gargalos administrativos, delineando as medidas que estão sendo tomadas para agilizar os processos de candidatura e distribuição de fundos, garantindo eficiência e transparência.
			\item Enfatizou a importância de um modelo contínuo de avaliação, no qual a análise de dados e as avaliações de impacto serão utilizadas para aperfeiçoar e melhorar o modelo de financiamento ao longo do tempo.
		\end{itemize}\\
		\hline
		Rogério Ceron & Secretário do Tesouro Nacional do Ministério da Fazenda & \begin{itemize}
			\item Apresentou a abordagem do Tesouro para garantir o financiamento do programa a longo prazo, mantendo a responsabilidade fiscal, detalhando como a iniciativa se alinha com as prioridades orçamentárias mais amplas do governo.
			\item Explicou os mecanismos pelos quais o governo subsidiará as taxas de juros sem comprometer a estabilidade macroeconômica, garantindo previsibilidade nas contas públicas.
			\item Discutiu as salvaguardas financeiras implementadas para evitar exposição fiscal excessiva, incluindo tetos de gastos e mecanismos de ajuste dos parâmetros do programa com base nas condições econômicas.
			\item Esclareceu o papel do Tesouro Nacional na coordenação da distribuição de fundos com instituições financeiras públicas, garantindo conformidade com os limites constitucionais de gastos.
			\item Destacou a importância da responsabilidade e transparência na gestão dos fundos públicos destinados à educação, detalhando como auditorias periódicas e mecanismos de prestação de contas garantirão a alocação eficiente dos recursos.
		\end{itemize}\\
		\hline
	\end{tabular}
\end{table}


\newpage

\section*{Painel 3: A proposta sobre a ótica da educação profissional nos Estados}

\begin{table}[htbp!]
	\centering
	\renewcommand{\arraystretch}{1}
	\begin{tabular}{|p{0.75in}|p{0.75in}|p{5.1in}|}
		\hline
		Nome & Atribuição & Resumo dos Pontos \\  
		\hline
		Murilo Camaroto & Repórter do Valor Econômico & \begin{itemize}
			\item Perguntou sobre os maiores desafios de implementação para o financiamento da educação técnica por meio desta iniciativa.
			\item Investigou o alinhamento das políticas de educação profissional com as demandas do mercado de trabalho em diferentes estados.
		\end{itemize}\\
		\hline
		Fernando Exman & Chefe da Sucursal do Valor Econômico em Brasília & \begin{itemize}
			\item Questionou como as restrições fiscais de cada estado impactam sua capacidade de expandir a educação profissional.
			\item Explorou como os estados podem manter a sustentabilidade financeira ao mesmo tempo em que ampliam o acesso a programas de treinamento técnico.
		\end{itemize}\\
		\hline
		Roni Miranda & Secretário de Educação do Estado do Paraná / CONSED & \begin{itemize}
			\item Discutiu a abordagem do Paraná para integrar a educação técnica dentro da estratégia educacional mais ampla do estado, enfatizando o alinhamento com as demandas da indústria e os avanços tecnológicos.
			\item Destacou parcerias entre instituições públicas e indústrias locais, explicando como currículos co-projetados com empresas garantem que os alunos desenvolvam habilidades adequadas ao mercado de trabalho.
			\item Explicou como o estado garante que estudantes de todas as origens socioeconômicas possam acessar programas de treinamento técnico, detalhando o papel de bolsas de estudo e mensalidades subsidiadas pelo governo.
			\item Enfatizou a importância da expansão dos modelos de educação dual, nos quais os alunos dividem seu tempo entre aprendizado em sala de aula e estágios em ambientes de trabalho reais.
			\item Discutiu a tomada de decisões baseada em dados na política educacional, usando análises do mercado de trabalho para ajustar os programas de treinamento conforme as necessidades de emprego projetadas.
		\end{itemize}\\
		\hline
		Fátima Gavioli & Secretária de Educação do Estado de Goiás & \begin{itemize}
			\item Apresentou as prioridades de investimento de Goiás na educação profissional, com foco na expansão da matrícula e da infraestrutura, especialmente em áreas rurais carentes.
			\item Descreveu estratégias para adaptar os currículos às forças econômicas regionais, especialmente nos setores de agronegócio, tecnologia e serviços, garantindo que os alunos estejam preparados para empregos de alta demanda.
			\item Enfatizou a importância dos programas contínuos de capacitação docente para manter a qualidade da educação profissional, garantindo que os instrutores estejam atualizados sobre tendências da indústria e métodos pedagógicos em evolução.
			\item Explicou a implementação de sistemas de monitoramento de desempenho para avaliar o sucesso dos formandos da educação profissional no mercado de trabalho e ajustar os programas conforme necessário.
			\item Abordou as restrições orçamentárias e como Goiás está aproveitando o financiamento federal, a colaboração do setor privado e medidas de eficiência para expandir a educação profissional sem comprometer a qualidade.
		\end{itemize}\\
		\hline
		Guilherme Lichand & Professor da Universidade de Stanford & \begin{itemize}
			\item Apresentou uma análise econômica das reformas propostas, enfatizando a relação custo-benefício de longo prazo do investimento em educação técnica, demonstrando como uma força de trabalho qualificada leva ao crescimento do PIB.
			\item Forneceu estimativas financeiras, observando que atingir o índice da OCDE de 37\% de matrícula na educação técnica exigiria um investimento adicional estimado de R\$50 bilhões ao longo de seis anos, com investimentos anuais de R\$8–10 bilhões.
			\item Destacou evidências de modelos internacionais, demonstrando que programas profissionais bem financiados podem aumentar as taxas de emprego em até 20\% nos setores relevantes, particularmente em STEM e saúde.
			\item Explicou o retorno projetado do investimento para os gastos do governo em treinamento profissional, estimando que cada R\$1 investido na educação técnica resulta em R\$3–4 em produção econômica ao longo de uma década.
		\end{itemize}\\
		\hline
	\end{tabular}
\end{table}



\newpage

\section*{Painel 4: A proposta sobre a ótica das Finanças Públicas nos Estados}

\begin{table}[htbp!]
	\centering
	\renewcommand{\arraystretch}{1.2}
	\begin{tabular}{|p{1.2in}|p{1.6in}|p{4.2in}|}
		\hline
		Nome & Atribuição & Resumo dos Pontos \\
		\hline
		Fernando Exman & Jornalista do Valor Econômico & 
		\begin{itemize}
			\item Perguntou como os governos estaduais estão ajustando suas políticas fiscais para acomodar os custos do programa 'Juros pela Educação', mantendo a estabilidade orçamentária geral.
		\end{itemize}\\
		\hline
		Lu Aiko & Jornalista do Valor Econômico & \begin{itemize}
			\item Levantou preocupações sobre a sustentabilidade financeira de longo prazo e os possíveis trade-offs orçamentários necessários para manter o programa.
		\end{itemize}\\
		\hline
		Felipe Salto & Economista-chefe da Warren Investimentos / Ex-Secretário da Fazenda de São Paulo & \begin{itemize}
			\item Apresentou uma análise fiscal detalhada das reformas propostas para o financiamento da educação, enfatizando o impacto nos orçamentos estaduais na próxima década.
			\item Citou projeções sugerindo a impossibilidade da reforma, considerando que o orçamento precisaria passar de um déficit de 2\% para um superávit de 3,5\%, equivalente a aproximadamente 500 bilhões de reais anuais sobre um PIB de 11 trilhões de reais.
			\item Destacou que estados com maior solidez fiscal—como São Paulo e Paraná—poderiam absorver os custos mais facilmente, enquanto outros com altos índices de endividamento, como Rio de Janeiro e Rio Grande do Sul, poderiam enfrentar dificuldades sem intervenção federal.
			\item Discutiu possíveis fontes de receita adicional, incluindo reformas tributárias e medidas de eficiência, para liberar recursos para o financiamento da educação.
			\item Utilizou uma abordagem comparativa, referenciando dados da OCDE para mostrar que o Brasil está atrasado nos investimentos em educação pública em relação ao seu produto econômico, necessitando de pelo menos R\$30 bilhões adicionais por ano para reduzir essa defasagem.
		\end{itemize}\\
		\hline
		Carlos Xavier & Presidente do Comsefaz / Secretário de Tributação do Rio Grande do Norte & \begin{itemize}
			\item Explicou como a distribuição da receita tributária estadual impacta a capacidade das diferentes regiões de financiar reformas educacionais, enfatizando as disparidades entre estados mais ricos e mais pobres.
			\item Discutiu o papel do Comsefaz na negociação de um arcabouço fiscal mais justo para garantir um financiamento equitativo para todos os estados, destacando propostas legislativas recentes.
		\end{itemize}\\
		\hline
		Luis Claudio Gomes & Secretário de Estado de Fazenda de Minas Gerais & \begin{itemize}
			\item Expôs as restrições financeiras de Minas Gerais na ampliação do financiamento da educação, citando os esforços contínuos do estado para renegociar sua dívida com o governo federal.
			\item Enfatizou a importância de acordos multilaterais entre estados e Brasília para garantir a sustentabilidade financeira de longo prazo das políticas educacionais.
		\end{itemize}\\
		\hline
		Vilma Pinto & Diretora da Instituição Fiscal Independente (IFI) do Senado Federal & \begin{itemize}
			\item Forneceu uma perspectiva fiscal independente sobre as reformas propostas, alertando que estados com altos déficits fiscais podem enfrentar restrições adicionais de endividamento caso não haja uma gestão adequada.
			\item Discutiu a necessidade de transparência no planejamento orçamentário estadual para evitar desequilíbrios financeiros futuros causados por novos gastos com educação.
		\end{itemize}\\
		\hline
	\end{tabular}
\end{table}

\end{document}
