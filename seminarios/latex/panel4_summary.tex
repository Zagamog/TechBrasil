\documentclass[a4paper,10pt]{article}
\usepackage[utf8]{inputenc}
\usepackage[left=0.5in, right=0.5in, top=0.8in, bottom=0.8in]{geometry}
\usepackage{enumitem}

\setlist[itemize]{topsep=0pt, partopsep=0pt, parsep=0pt} 

\begin{document}

\section*{Painel 4: A proposta sobre a ótica das Finanças Públicas nos Estados}

\begin{table}[htbp!]
\centering
\renewcommand{\arraystretch}{1.2}
\begin{tabular}{|p{1.2in}|p{1.6in}|p{4.2in}|}
\hline
Name & Assignation & Summary Bullets \\
\hline
Fernando Exman & Jornalista do Valor Econômico & 
\begin{itemize}
\item Asked how state governments are adjusting their fiscal policies to accommodate the costs of the 'Juros pela Educação' program while maintaining overall budget stability.
\end{itemize}\\
\hline
Lu Aiko & Jornalista do Valor Econômico & \begin{itemize}
\item Raised concerns about long-term financial sustainability and potential budgetary trade-offs required to maintain the program.
\end{itemize}\\
\hline
Felipe Salto & Economista-chefe da Warren Investimentos / Ex-Secretário da Fazenda de São Paulo & \begin{itemize}
\item Presented a detailed fiscal analysis of the proposed education financing reforms, emphasizing the impact on state budgets over the next decade.
\item Cited projections  suggesting the impossibility of the reform, for the budget to move from 2\% deficit to 3.5\% surplus equivalent to about 500 billion reais annually on GDP of 11 trillion reais. 
\item Highlighted that states with stronger fiscal health—such as São Paulo and Paraná—could absorb the costs more easily, while others with high debt ratios, such as Rio de Janeiro and Rio Grande do Sul, might struggle without federal intervention.
\item Discussed potential sources of additional revenue, including tax reforms and efficiency measures, to free up resources for education funding.
\item Used a comparative approach, referencing OECD data to show that Brazil lags behind in public education spending relative to economic output, needing at least an additional R\$30 billion annually to close the gap.
\end{itemize}\\
\hline
Carlos Xavier & Presidente do Comsefaz / Secretário de Tributação do Rio Grande do Norte & \begin{itemize}
\item Explained how state tax revenue distribution impacts the ability of different regions to finance educational reforms, emphasizing disparities between wealthier and poorer states.
\item Discussed the role of the Comsefaz in negotiating a fairer fiscal framework to ensure equitable funding for all states, highlighting recent legislative proposals.
\end{itemize}\\
\hline
Luis Claudio Gomes & Secretário de Estado de Fazenda de Minas Gerais & \begin{itemize}
\item Outlined Minas Gerais’ financial constraints in expanding education funding, citing the state’s ongoing debt renegotiation efforts with the federal government.
\item Emphasized the importance of multilateral agreements between states and Brasília to secure long-term financial sustainability for education policies.
\end{itemize}\\
\hline
Vilma Pinto & Diretora da Instituição Fiscal Independente (IFI) do Senado Federal & \begin{itemize}
\item Provided an independent fiscal perspective on the proposed reforms, warning that states with high fiscal deficits may face additional borrowing constraints if not properly managed.
\item Discussed the need for transparency in state budget planning to prevent future financial imbalances caused by new education-related expenditures.
\end{itemize}\\
\hline
\end{tabular}
\end{table}

\end{document}
