\documentclass[a4paper,10pt]{article}
\usepackage[utf8]{inputenc}
\usepackage[left=0.5in, right=0.in, top=0.3in, bottom=0.3in]{geometry}
\usepackage{enumitem}

\setlist[itemize]{topsep=0pt, partopsep=0pt, parsep=0pt} 

\begin{document}

\section*{Painel 3: A proposta sobre a ótica da educação profissional nos Estados}

\begin{table}[htbp!]
\centering
\renewcommand{\arraystretch}{1.2}
\begin{tabular}{|p{1in}|p{1in}|p{4.6in}|}
\hline
Name & Assignation & Summary Bullets \\
\hline
Murilo Camaroto & Repórter do Valor Econômico & \begin{itemize}
\item Inquired about the biggest implementation challenges for financing technical education through this initiative.
\item Probed into the alignment of professional education policies with labor market demands in different states.
\end{itemize}\\
\hline
Fernando Exman & Chefe da Sucursal do Valor Econômico em Brasília & \begin{itemize}
\item Questioned how the fiscal constraints of each state impact their ability to expand professional education.
\item Explored how states can maintain financial sustainability while increasing access to technical training programs.
\end{itemize}\\
\hline
Roni Miranda & Secretário de Educação do Estado do Paraná / CONSED & \begin{itemize}
\item Discussed Paraná’s approach to integrating technical education within the state's broader educational strategy, emphasizing alignment with industry demands and technological advancements.
\item Highlighted partnerships between public institutions and local industries, explaining how co-designed curricula with businesses ensure students develop job-ready skills.
\item Explained how the state ensures that students from all socioeconomic backgrounds can access technical training programs, detailing the role of scholarships and government-subsidized tuition.
\item Emphasized the importance of expanding dual education models, where students split time between classroom learning and apprenticeships in real work environments.
\item Discussed data-driven decision-making in education policy, using labor market analytics to adjust training programs according to projected employment needs.
\end{itemize}\\
\hline
Fátima Gavioli & Secretária de Educação do Estado de Goiás & \begin{itemize}
\item Outlined Goiás’ investment priorities in vocational education, focusing on expanding enrollment and infrastructure, particularly in underserved rural areas.
\item Described strategies to tailor curricula to regional economic strengths, particularly in agribusiness, technology, and service industries, ensuring students are prepared for high-demand jobs.
\item Emphasized the importance of ongoing teacher training programs to maintain the quality of professional education, ensuring instructors stay updated on industry trends and evolving pedagogical methods.
\item Explained the implementation of performance monitoring systems to evaluate the success of vocational training graduates in the labor market and adjust programs accordingly.
\item Addressed budget constraints and how Goiás is leveraging federal funding, private sector collaboration, and efficiency measures to expand vocational education without compromising quality.
\end{itemize}\\
\hline
Guilherme Lichand & Professor da Universidade de Stanford & \begin{itemize}
\item Presented an economic analysis of the proposed reforms, emphasizing the long-term cost-benefit ratio of investing in technical education, showing how a skilled workforce leads to GDP growth.
\item Provided financial estimates, noting that reaching the OECD benchmark of 37\% technical education enrollment would require an estimated R\$50 billion in additional funding over six years, with annual investments of R\$8–10 billion.
\item Highlighted evidence from international models, demonstrating that well-funded vocational programs can boost employment rates by up to 20\% in relevant sectors, particularly in STEM and healthcare fields.
\item Explained the projected return on investment for government spending in vocational training, estimating that every R\$1 invested in technical education results in R\$3–4 in economic output over a decade.
\end{itemize}\\
\hline
\end{tabular}
\end{table}

\end{document}
