
\documentclass[a4paper,12pt]{article}
\usepackage[utf8]{inputenc}
\usepackage[T1]{fontenc}
\usepackage{geometry}
\geometry{left=2cm, right=2cm, top=2cm, bottom=2cm}
\usepackage{parskip}

\title{YouTube Transcript - JeSZGEUUrhM}
\author{Auto-generated}
\date{}

\begin{document}

\maketitle

\section*{Transcript}

eh o Brasil é a nossa fonte de inspiração acreditamos no poder transformador das

histórias contadas todos os dias de norte a sul do país somos curiosos nos

aprofundamos nos temas e a consistência ao longo de décadas nos deu a autoridade

reconhecida pela audiência ajudamos a fazer as perguntas certas com inteligência

e conteúdo de alto nível vamos além das notícias para chegar a intimidade de

cada história e inspiramos as tomadas de decisão da saúde dos filhos ao futuro

do Planeta das pequenas compras as grandes aquisições do agronegócio a

gastronomia das tendências da moda aos clássicos da [Música] decoração das

tecnologias do dia a dia as grandes inovações da escolha do candidato ao seu

próximo carro de onde investir a como empreender traduzimos a complexidade de um

mundo em transformação Nosso propósito é conectar pessoas e histórias informando

educando entretendo [Música] debatendo ouvindo e nos transformando com o nosso

público reconhecemos os talentos e celebramos suas conquistas criamos

experiências únicas Antecipamos tendências e refletimos a diversidade cultural

do nosso país nossa paixão pela comunicação a inquietude criativa o pioneirismo

o inegociável compromisso do jornalismo com o fortalecimento da democracia e a

busca incansável pela qualidade nos levam à liderança acreditamos na

interdependência da sociedade e das instituições e que cidadãos inspirados e bem

informados poder para construir um país melhor trabalhamos com entusiasmo e

disciplina e mudamos constantemente para permanecermos fiéis à nossa essência e

ao Espírito Criador dos nossos fundadores Olá muito bom dia mais uma vez

senhoras e senhores sejam bem-vindos sejam bem-vindas ao seminário juros por

educação uma iniciativa do valor econômico em parceria com Itaú educação e

trabalho e Todos Pela Educação eu sou Débora Freitas Âncora da Rádio CBN fico

muito orgulhosa por estar no comando deste encontro que é o primeiro dos fóruns

Valor Econômico um novo ciclo de seminários focado em pautar discussões

essenciais para o desenvolvimento do Brasil sempre reunindo líderes empresariais

autoridades e especialistas em uma rica e produtiva troca de ideias hoje a

discussão é sobre a expansão e o fortalecimento da educação profissional e

tecnológica em nosso país com o recente lançamento da proposta juros por

educação pelo Governo Federal em conjunto com o ministério da fazenda e o

Ministério da Educação Vivemos um momento certamente decisivo Então vamos

aproveitar a oportunidade para conversar sobre o assunto entender suas minúcias

e mais importante garantir a eficácia da implementação desse projeto para isso

contaremos com as vozes mais interessadas e bem Preparadas em cada painel de

discussão mas antes de dar início a intensa programação desta manhã eu gostaria

de agradecer o prestígio da plateia aqui presente e a audiência de todos que

estão nos acompanhando ao vivo pelo YouTube LinkedIn e Facebook do valor

econômico e informar que além das plataformas do valor nosso evento conta também

com a divulgação e cobertura do Jornal Globo CBN e canal Globo News agora sim

abrindo oficialmente o nosso fórum vamos ouvir as boas-vindas do valor econômico

realizador do evento e de seus parceiros Itaú educação e trabalho e Todos Pela

Educação gostaria então de chamar ao palco Fernando exman chefe da sucursal do

valor econômico em Brasília Fernando por favor Ana in do Itaú educação e

trabalhoa por gentileza e Priscila Cruz do Todos Pela Educação Priscila por

[Aplausos] favor por gentileza Fernando o microfone é seu você pode dar início

aqui à nossas falas Bom dia uma ótima manhã a todos eh em nome da nossa diretora

de redação Maria Fernanda delmas eu queria dar eh boas-vindas a todos e todas

com educação tem sido um tema cada vez mais presente eh nas nossas entrevistas

conversas com empresários investidores autoridades eh de todas as esferas do

poder público eh até porque enfim tem eh desdobramentos em suas várias eh

camadas sociais econômicas eh reflexões que que acabam eh desdobrando para eh

como a produtividade qualificação profissional e obviamente isso tem sido

refletido no nosso material tanto em reportagens como em colunas com certeza

isso cada vez mais ocorrerá e é com muito prazer que a gente também eh inicia

esse evento sobre ensino técnico eh formato amplo vai garantir o A reflexão e a

discussão desse tema em seus diversos ângulos assim como eh se espera de um

jornal como Valor Econômico queria agradecer a presença a todos e passar a

palavra para nossos convidados também obrigada F Opa pronto obrigada Fernando a

palavra tá com você Ana por gentileza muito obrigada Bom dia a todos muito

obrigada pela presença agradecer a presença do do ministro em nome de quem eu

cumprimento todos aqui eh a educação técnica profissional é um assunto bastante

relevante para juventudes brasileiras e para o desenvolvimento econômico do país

eh temos que lembrar que nós temos 88,8 por dados do censo matriculados na

educação básica na educação pública do país eh então 88% dos jovens hoje dos

adultos de amanhã seram egressos da escola pública e desses jovens de 18 a 24

anos nós temos 25% que estão na universidade os outros 75% quando acaba o ensino

médio eh nós temos que pensar Qual é a política a ser oferecida para eles então

Eh esse evento ele é muito bem-vindo justamente porque por iniciativa do

Ministério da Educação eh junto com o ministério da fazenda Eh estamos pensando

na verdade numa perspectiva paraas juventudes e no futuro do Brasil Então tô

muito feliz de estar aqui muito obrigada muito obrigada complementar aqui nossa

a nossa abertura Vamos ouvir a Priscila por favor bom bom dia a todos todas uma

alegria em primeiro lugar ter esse evento sendo realizado em parceria com o

Instituto educação e trabalho um grande parceiro com o valor econômico que de

fato tem conseguido dar muito Eco para essa causa da educação profissional quero

agradecer a presença do ministro Camilo Santana que tem liderado vários

processos muito importantes paraa retomada de uma nova educação pública no país

secretários governadores a gente tem aqui sistema de Justiça tribunais de contas

sociedade civil organizada jornalistas a gente tem aqui uma um pessoal muito

engajado em educação e que entende a importância da educação profissional para o

desenvolvimento social e econômico do país mas especialmente como direito desses

jovens né direito a uma formação de qualidade direito a um emprego melhor a sair

da pobreza né a ter um novo de vida quero também dizer aqui Ministro que esse

evento ele também só é possível porque justamente por conta da sua liderança a

gente tem agora um novo ensino médio melhor do que o que a gente tinha antes a

gente vai ter esse novo ensino médio que vai demandar mais uma expansão da

educação profissional eh que por sua vez vai demandar mais recursos né E para

fazer isso de uma forma consistente garantindo qualidade para todos essa

integração entre os Ministérios é muito fundamental esse tipo de inovação como

juros por educação que une a educação com A Fazenda é algo que a gente precisa

cada vez mais secretário seron expandir no país porque problemas complexos só

vão ser resolvidos com a integração e o trabalho de muita gente De forma

conjunta e articulada então quero agradecer a presença de todos agradecer a

parceria agradecer especialmente as equipes do todos para educação do Instituto

de Educação e trabalho do valor econômico que tornou possível esse evento um bom

evento para todo mundo que as discussões que a gente consiga sair daqui com uma

boa discussão para fazer com que essa política avance para o benefício do

Estudante brasileiro muito obrigada muito obrigada Priscila Ana prazer ter vocês

aqui com a gente bem Vamos então iniciar aos nossos debates o título do primeiro

painel é a estratégia do programa por educação Fernando chefe da sucursal de

Brasília participará de todos os painéis como mediador ao lado de outros colegas

alternadamente no painel um a dupla de Fernando será luot repórter especial e

colonista do valor Lu por gentileza muito obrigada os entrevistados aqui serão

Camilo Santana Ministro da Educação Ministro por gentileza João Azevedo

governador do Estado da Paraíba Governador por favor e Dário durigan secretário

executivo do Ministério da Fazenda secretário por favor Muito bom dia será uma

honra ouvi-los obrigada bom dia bom dia a todos novamente eh nossa proposta né a

gente é que cada um tenha de 5 a 7 minutos para uma eh umas falas iniciais E aí

depois a gente segue com um bate-papo e e vai ser um prazer depois enfim ter

também a a possibilidade de ouvir as perguntas da Lu com quem eu tenho muita

honra queria primeiro passar a palavra pro secretário durigan secretário

executivo da fazenda para que ele possa fazer suas considerações iniciais

Obrigado Fernando Ministro Governador Lu Priscila queria agradecer o convite do

valor do Todos Pela Educação e cumprimentar todos os presentes obrigado pela

oportunidade pela fala queria dividir essa consideração inicial em talvez três

tópicos principais O primeiro é o compromisso que a equipe do Ministério da

Fazenda tem e eu em especial com o tema da educação a minha a família toda meu

pai foi reitor da Unesp durante muito tempo em São Paulo meu falecido pai e ele

dizia isso a todo o tempo ele dizia que o papel da educação era emancipar as

pessoas e via na educação uma oportunidade de vida para todo mundo e pregava

isso com muito afinco e meu falecido pai que que faleceu em 2017 e deixou uma

série de sementes plantadas seja na minha família e seja eh no no trabalho que a

gente tem desenvolvido Então queria fazer essa essa marca que é pessoal mas é

também de toda a equipe do do Ministro Fernando hadad que já foi como todos

sabem ministro da educação do país um segundo ponto é talvez a questão mais

institucional eh o ministério da fazenda tem duas grandes premissas para atuação

a primeira premissa é olhar pro que é preciso fazer pro Brasil se desenvolver e

esse desenvolvimento ele passa por uma série de Capítulos mas um deles é a

responsabilidade fiscal fazendo o ajuste fiscal não qualquer ajuste fiscal mas o

ajuste fiscal que vá pelo caminho da Justiça pelo caminho da eliminação de

distorções mitigação de benefícios ineficientes que façam com que a gente

recomponha a base fiscal do país e com isso permita olhar paraa frente e olhar

paraa frente é o quê proteger o colchão de proteção social que a gente tem no

país então tem fazendo ajuste fiscal a gente mantém um orçamento hígido o

orçamento de educação por exemplo sem maiores corte sem maiores eh sem maiores

limitações e a gente olha pra frente no sentido de eh galvanizar o

desenvolvimento que é preciso que o Brasil tenha com crédito barato Com inflação

sob controle e na Parceria institucional que a gente tem feito a gente tem

conseguido avançar nesse sentido e o segundo Pilar é a valorização do Diálogo

institucional e eu digo o ministro hadad tem dito a todo o tempo o diálogo com o

Congresso Nacional o diálogo com o Supremo Tribunal Federal com o tribunal de

contas com o sistema de justiça como um todo e em especial Claro dentro do

governo a gente acredita que não é quebrando o prédio público não é com um

diálogo que a gente vai construir um Brasil desenvolvido que tenha credibilidade

internamente e externamente e pegando gancho no no compromisso institucional

alguns deles não são muito caros né a gente tem falado muito do plano de

transformação ecológica que é uma parceria eh muito feliz que o ministério da

fazenda tem com vários outros Ministérios mas em especial com o ministério de

meio ambiente com a ministra Marina Silva e a gente acredita que grande impulso

o grande impulso do desenvolvimento do país Vem de um desenvolvimento

sustentável que olhe pras cadeias produtivas olhe pro Impacto das mudanças

climáticas a gente tá vivendo agora um momento triste no Rio Grande do Sul e a

gente diferente eu digo às vezes diferente da iniciativa privada em que quando

você tem uma nova prioridade você consegue afastar prioridades antigas a gente

acumula prioridades no setor público Então acumulamos essa outra prioridade

agora que é fazer a resposta ao Rio Grande do Sul temos feito isso de maneira

muito forte começando do fim de semana com o presidente ontem e hoje a gente vai

seguir com novas medidas o Diogo institucional ele precisa permear el precisa

virar uma normalidade que a gente perdeu por algum tempo no Brasil e eu queria

dizer ao Ministro Camilo que uma parceria também que a gente tem muito carinho e

é priorit pro Ministério da Fazenda é com o ministério da educção a gente

discutiu mais cedo nessa gão que a reforma tributária ela seria um grande ganho

de produtividade pro país e é preciso entender a grandeza da reforma tributária

que o congresso aprovou ano passado e que nós estamos coordenando a

regulamentação esse ano Mas o segundo grande tema em termos de ganho de

produtividade do ponto de vista econômico né Para Além dos outros aspectos que o

ministro Camilo vai certamente se referir o tema da educação em especial a

educação profissionalizante a gente tem clareza no Ministério da Fazenda que

esse é um um elemento de ganho de produtividade um elemento de ganho de de

competitividade pro país um elemento de inovação para além dos ganhos de

emancipação e de qualidade de vida das pessoas é bom pra economia que a gente

tem uma boa educação no país é bom pra economia que os nossos jovens tenham

oportunidades de estudar de se capacitar de ir pro mercado de trabalho de ter

bom salário então a gente não mede esforços no ministério para que a gente use

os instrumentos financeiros use a inteligência financeira instalada na equipe

Econômica Para viabilizar Eh esses bons projetos pro país seja a título de

exemplo a transformação ecológica seja o juros por educação que a gente eh tem

tem tratado mais recentemente nesse projeto de juros para educação eh a gente

trago em linhas Gerais depois meu amigo Rogério Ceron vai trazer Aí talvez mais

detalhes e eu tô disposição para responder há um problema em especial de alguns

estados mas de toda a Federação de uma dívida com a união que vai sendo rolada

com o tempo e que há uma dificuldade de fato em arcar eh em pagar toda essa

dívida e a sensibilidade do ministro Rad permitiu com que a gente tivesse um

olhar não simplesmente de eh perdoar ou simplesmente de abrir o espaço fiscal

mas de induzir com bons estímulos esse projeto então a gente tá olhando para um

dinheiro que seria pago a título de rolagem de dívida a título de juros pra

União a união abre mão de receber esse dinheiro como recurso financeiro mas em

prol de um estímulo pela educação média profissionalizante a gente acredita

nisso é bom paraas pessoas é bom pra educação e vai ser bom pra economia do

Brasil se esse projeto for adiante Então hoje para que vocês tenham ideia o o a

a a taxa com que é rolada a dívida dos Estados com a união que é em linhas

Gerais IPCA mais 4% ela já tem algum tipo de subsídio do do do ponto de vista da

União isso já se rola num num num num num nível mais baixo do que por exemplo a

união capta e rola sua própria dívida mas o estímulo Aquele é maior então a

gente tem proposto reduzir ainda mais esse ônus pros Estados eh desde que a

gente tenha esse plano de de fortalecimento do ensino médio e profissionalizante

pra gente levar o país a um outro nível de produtividade Então queria agradecer

aqui em especial o ministro Camilo pela parceria próxima e fecunda com o

ministério da fazenda e reforçar o compromisso do Ministério da Fazenda com

esses bons projetos nós estamos no Ministério da Fazenda fazendo eh O que é

preciso do ponto de vista fiscal do ponto de vista regulatório mas a gente tem a

sensibilidade a gente tem o olhar eh cond de onde a gente tem que manter a

condução do país pros caminhos certos sem negacionismo com amplo diálogo com

qualificação com ciência com educação com valorização do meio ambiente e é para

aí que a gente gostaria de seguir trilh obrigado obrigado secretário Governador

nós quer ouv Senhor por fa Bom dia Bom dia a todos parabenizar pela organização

de um evento que trata de um tema tão importante então a todos da organização os

meus parabéns e dizer de uma maneira muito rápida o que aconteceu na Paraíba nos

últimos anos em termos de ensino técnico eu tenho uma leitura da importância que

é o ensino técnico para o p inte eu sempre brinco que eu sou fruto do ensino

técnico lá atrás eu tive a oportunidade de fazer curso técnico eu fui aluno do

Instituto hoje Instituto Federal da educação naquela época escola Industrial

Escola Técnica e eu fiz curso técnico de estradas e durante boa parte da minha

vida eu me sustentei me mantive e tive a oportunidade de fazer um curso de nível

superior de engenharia exatamente com tudo aquilo que o ensino técnico me me eu

acho que fortaleceu na verdade porque essa essa formação tão importante ela

gerou abriu portas e eu tive a oportunidade de trabalhar em diversas empresas

tudo e isso fez com que eu tivesse condições vindo de uma família pobre bancar

até meus estudos na universidade então o ensino técnico para mim eu tenho sempre

um olhar completamente diferente por isso que na Paraíba a gente tem avançado

muito a gente tem buscado fazer com que essa esse segmento tenha um olhar

Diferente ao chegarmos ao Governo nós encontramos cerca de 50 escolas de ensino

técnico hoje são 161 equipamentos que permitem esse ensino nós avançamos no

número de matrículas triplicamos o número de matrículas fizemos uma caminhada

que eu acho que foi importante eu trouxe alguns slides eu não sei se estão

disponíveis aí para talvez facilitar um pouquinho essa compreensão Nossa a

respeito do que aconteceu eu não sei onde é que tá o computador então no no

nosso plano Estadual de Educação já tá lá estabelecido que expandir a oferta da

educação profissional técnica ela tem que ser buscada constantemente então

triplicamos o número de matrículas ela responsável hoje por 50% da oferta de

matrícula já em 2019 houve uma evolução significativa de matrículas no ensino

técnico chegando a a mais de 38.000 alunos já matriculados esse ano no ensino t

t e da mesma forma a estrutura física acompanhou também hoje são 161 unidades

que ministram cursos técnicos das mais diversas áreas e Aqui vai um

agradecimento especial à Fundação Itaú através D Ana que nos dá essa Assessoria

extraordinária da Identificação do que é que deve ser implantado em termos de

cursos técnicos por região do estado nós identificamos no estado as suas

demandas e há uma uma constante atualização do processo até porque quando você

tem um segmento que começa a crescer muito caso das energias renováveis Paraíba

hoje tem mais de 60 parques instalados de energia solar e eólica e isso precisa

ter uma mão de obra então a gente direciona essa área vem indústrias na área de

de na área farmacológica Então você direciona na área automobilística ou seja há

uma interação constante com relação aos cursos que são ministrados Então essa

estrutura ela permite essa flexibilização e nós estamos trabalhando para ampliar

porque lá na Paraíba nós temos 65% das escolas de ensino médio em tempo integral

65% das nossas escolas de Ensino Médio elas são de tempo integral isso fez uma

uma uma uma revolução dentro desse processo eh do ensino médio Esses são os

eixos são os diversos cursos aí fizeram o slide pra gente não ler mesmo né mas

cada linhazinha dessa é um curso que é ministrado lá na em cada escola e a gente

tem também algumas ações voltadas para o ensino técnico o programa dinheiro

direto na nas escolas nas escolas técnicas ele tá aumentando mais de 60% o valor

que nós destinamos diretamente pra escola Esse é um recurso que chega na escola

que melhora o funcionamento das escolas porque você tem a a a Aquela pequena

demanda Aquela pequena necessidade da escola sendo resolvida e nós estamos aí

aumentando 60% do valor a outra meta obviamente também dentro do plano estadual

da educação é é promover a expansão do estágio nós temos na Paraíba criado desde

2019 um programa chamado primeira chance esse programa a primeira chance ele

surgiu em função da necessidade de fazer com que o aluno que terminasse o curso

técnico ao chegar numa empresa ele respondesse a mais difícil pergunta que ele

escuta na vida Qual é a sua experiência como é que você entra numa empresa para

trabalhar se você não tem experiência você tá saindo da escola e aí Nós criamos

um programa primeira chance em que o governo do estado banca uma bolsa paga a

bolsa ele vai para a empresa trabalha e graças a Deus é é um percentual muito

alto mas cerca de 65% dos alunos eles são contratados 65 75% São contratados na

empresa que eles fizeram Esse estágio é um número muito alto a gente tem esse

relatório mas isso atingi um número relativamente pequeno de alunos e a partir

desse ano nós fizemos a universalização o que é que significa 100% dos alunos

que saem do terceiro ano do ensino técnico eles farão estágio remunerado na

verdade é importante porque casa um pouco com o pé de meia é mais um recurso que

ele vai receber e ele pode pode aproveitar esse recurso então nós estamos

investindo aí 28 26 milhões de reais só nessa nesse programa primeira chance que

tem dado realmente uma oportunidade para muita gente e temos obviamente em

parceria com Pronatec o Paraíba Tec Paraí Tec e o Pronatec na Paraíba nesse

período já formou mais de 6.000 e jovens porque envolvemos inclusive diversas

secretarias secretaria da agricultura da Administração Penitenciária um programa

chamado Porto cidade que Prepara Cursos e e ministra cursos voltados para para

atuação não só no porto mas na cidade em que o porto está inserido com as

prefeituras enfim é um um um um um um avanço importante porque são cursos de

curta duração e que tem uma um resultado importante Essas são as ações do do

Pronatec 16 editais envolvendo recursos R 18 milhões deais Esses são recursos de

uma forma geral eh envolvidos só em 2024 no ensino técnico na Paraíba mais de R

50 milhões deais que eu não tenho dúvida nenhuma permitirão que a gente continue

nessa evolução Como eu disse nós triplicamos a matrícula precisamos em função

das metas que foram estabelecidas até pela CD é que tem uma meta de 37% dos

alunos de Ensino Médio estejam na condição de de também concomitantemente ou não

estar fazendo curso técnico essa é uma busca constante e o estado da Paraíba tem

feito eh o seu dever de casa e com relação à questão do juros por educação eu

tenho só uma observação a fazer conceitualmente eu acho que o programa é muito

bem pensado eu acho que ele atende agora é preciso Um pouco mais para os estados

que fizeram seus deveres de casa se você faz uma análise do programa você vê lá

que para os estados que não tenam dívida ou que tenham pequenas dívidas com o

estado será permitido um uma agilização dos empréstimos concedidos Eu acho que

isso é muito pouco para quem fez o dever de casa e por que que eu digo isso

porque o Estado da Paraíba é reado o Tesouro Nacional por 4 anos consecutiva nós

temos uma relação eh dívida receita líquida que é negativa ou seja porque a

liquidez Nossa tá lá em cima pela SP Global rates nós somos triplo A então nós

fizemos o dever de casa nós estamos avançando muito no ensino técnico então como

casar esses interesses eu acho que a gente tem que pensar no juros com educação

mas tem que ter o cuidado de fazer uma análise estado por pro estado para que os

estados que fizeram o seu dever de casa estão devidamente organizados Ministro

possam ter um outro tipo de benefício porque senão não justifica muito parece

fazer gestão fiscal correta considerando que a dívida de 700 e tantos bilhões de

reais aí se concentra em quatro ou cinco estados do Brasil no restante não então

dentro dessa lógica a única coisa que eu gostaria realmente é que a gente

pudesse discutir formas de incluir os estados que avançaram o Ceará é um estado

que avançou muito também no ensino técnico e que tem sua condição fiscal boa

então é preciso Balancear essas duas coisas mas a Paraíba avançou e fruto dessa

leitura que nós temos da importância do curso técnico a formação técnica é

fundamental para nosso jovem abre verdadeiramente portas que só o ensino médio

não consegue abr obrigado muito obrigado Governador antes de passar a palavra

pro Ministro queria dizer Lu o governador Já facilitou o nosso trabalho porque

já fez a primeira provocação para depois a gente seguir no debate pensando

exatamente isso Ministro por favor bom bom dia a todos e a todas bom dia bom dia

bom dia queria cumprimentar aqui o daril secretário executivo do Ministério da

Fazenda o secretário seron aqui presente cumprimentar também o Luiz e a Lu

Agradecer o convite cumprimentar aqui a Ana e a Priscila e cumprimentar meu

colega meu ex-colega governador da Paraíba João Azevedo e também cumprimentar a

secretária Fátima Gavioli e cumprimentar toda a imprensa e o público aqui

presente fala além de parabenizar e agradecer também ao daril ao Ministério da

Fazenda ao Ministro hadad pelas parcerias que nós temos construído nesse um ano

praticamente 1 ano e 4 meses um pouco mais de 4 meses do mandato do do

presidente Lula eh primeiro foi um momento de reconstrução né sabe o o desmonte

que foi feito com o Ministério da Educação nos últimos anos eu digo isso porque

eu sou testemunho como ex-governador da dificuldade de relação de diálogo né

portanto reconstruir do ponto de vista orçamentário do ponto de vista de pessoal

em tão pouco tempo já fizemos um concurso público já estamos chamando os novos

servidores do ministério que há 10 anos não tinha concurso público eh reativar

uma série de políticas públicas eh que foram desativadas no ministério enfim eh

e também uma das coisas mais importantes que eu considera é retomar o diálogo

federativo até porque nós estamos tratando de um tema que não que quem tá lá na

ponta quem executa são os estados e municípios eu sempre digo que o MEC tem um

papel de coordenar de induzir a política pública né portanto Mas quem executa a

educação básica lá na ponta são os estados e municípios Então reestabelecer esse

diálogo é fundamental e o olhar que nós temos tido no ministério é um olhar

sistêmico de olhar desde a creche a pós-graduação eu digo isso porque a gente tá

tratando hoje de um tema que é educação profissional mas H dentro desse tema

algo fundamental que hoje é um debate no Brasil é a questão da que nós queremos

primeiro garantir acesso a todos educação públ brasileir depois garantir

qualidade na aprendizagem depois garantir permanência desse jovem dessa crian na

escola Claro com inclusão e com Equidade e nós sabemos a quantidade de jovens e

crianças que tem abandonado a escola pública no Brasil isso tem um impacto

direto forte do ponto de vista na economia do ponto de vista social pro nosso

país Portanto quero dizer aqui fazendo essa análise inicial de que um dos

primeiros compromissos que nós Assumimos e construímos junto com os estados e

municípios nesse 1 ano e qu meses foi o compromisso Nacional criança

alfabetizada por que isso porque nenhuma política Hoje nós estamos implementando

no mec ela é feita sem Evidências e todas as evidências já mostraram que quando

a criança não aprende a leer escrever ao final do segundo ano de ensino

fundamental isso compromete todos os anos escolares da Educação Básica aumenta a

distorção idade série aumenta o abandono aumenta evasão E chegamos ao que nós

chegamos hoje em perder Quase meio milhão de jovem do ensino médio brasileiro

que deixam a escola pública em um ano no Brasil nós estamos falando de meio

milhão de jovens que é a força né de um país da Juventude de um país Um país que

tá mudando a sua trajetória começa a envelhecer a sua população portanto nós

estamos perdendo o momento mais importante do país né Então esse é um ponto

Fundamental e todas as evidências já mostraram as mudanças que ocorrem quando

essa ação e eu quero dizer aqui Governador e João que todos os estados aderiram

a política 99 9,1 para mim até me surpreendeu dos Municípios brasileiros

aderiram a essa política nós estamos hoje com mais de 7.000 articuladores

alfabetizadores em todos os municípios em todos os municípios do Brasil hoje

Portanto para mim isso fundamental fundamental quando o governo apresenta uma

política de escola tempo integral porque as evidências também já mostraram saiu

até agora uma publicação mostrando que aprendizagem de matemática aumenta em 35%

no aluno quando está numa escola de tempo integral além de ter o aspecto social

para isso portanto é fundamental eu nós Como eu como fui governador do Ceará da

mesma forma com o João implementamos uma grande rede de escola de ensino téc

aliás eu quero parabenizar as iniciativas individuais dos governadores e

prefeitos nesse país né E todas as nossas escolas técnicas estaduais no Ceará

todas são tempo integral em dois turnos inclusive com uma bolsa no final do

semestre para o estágio dos alunos portanto a escola tempo integral o Brasil

hoje o Ministério da Educação tá investindo R 4 bilhões deais por ano para

induzir para estimular a nossa meta esse ano foi 1 milhão de novas matrículas e

conseguimos implementar hoje nós temos 1 milhão de novos jovens e crianças em

sala de aula em 2024 em escola de tempo integral e chegar ao ensino médio numa

consulta que nós fizemos recentemente em relação ao tema do novo ensino médio

80% dos jovens que participaram mais de 130.000 jovens querem o ensino técnico

no ensino médio a gente precisa evitar a evasão e o abandono escolar no ensino

médio brasileiro o ibg mostrou 69 milhões de brasileiros Não terminaram a

educação básica no Brasil nós estamos falando de 1/3 da população brasileira

portanto quando o governo do presidente Lula lança o programa pé de meia ele faz

parte de um conjunto de ações porque não é só isso que vai resolver mas nós

sabemos que o grande motivo do aluno abandonar a escola é questão financeira não

é opção dele não é escolha é necessidade portanto faz parte desse esforço né é

ter uma escola com uma boa infraestrutura é ter uma boa formação de professores

é uma cimento o aluno que é um jovem quer ir para uma escola que tenha

conectividade que tenha Esporte que tenha uma escola de tempo integral mas

fundamentalmente o jovem brasileiro hoje quer formação técnica Portanto o

programa juros educação ele vem a Se somar a essa estratégia porque tem um

impacto forte na economia na produtividade todo o crescimento de um país tá

vinculado diretamente à produtividade a qualificação da mão de obra os estudos

já mostraram isso o efeito que tem no PIB brasileiro a gente ampliar a matrícula

de tempo de de escola técnica no Brasil portanto é um esforço criativo porque

nós precisamos de recursos para induzir essa política né precisamos investimento

para isso portanto aproveit a redução dos juros dos estados e buscar levar Levar

o país a patamares dos países da ocde é fundamental são ganhos sociais e

econômicos importantes né para o Brasil e para os estados brasileiros e o João

tem razão né Tem estados que tem grandes dívidas O que é que o ministério da

fazenda em parceria com o Ministério da Educação estão fazendo os estudos qual é

o valor das dívidas dos estados qual é a nossa meta chegar a 37% das matrículas

né de cursos de regulares do ensino médio concomitantes ou integrados ao ensino

técnico ao ensino médio Essa é a meta portanto nós temos ho algo em torno de

1.15.00 matrículas no ensino médio nessa natureza nós vamos agora lançar

lançamos agora 100 novos institutos federais estamos considerando já que você

vai ampliar em 140.000 novas vagas de ensino técnico do Brasil nós queremos

chegar a 1000 unidades de ensino técnico institutos federais até o final do

governo do presidente ele já tem anunciado isso para ampliar a matrícula do

ensino técnico mas não é suficiente para garantir é preciso ter um esforço

coletivo dos Estados do sistema S pra gente garantir um avanço mais rápido do

ensino técnico do Brasil então é definir a meta de cada estado a dívida de cada

estado a redução do juro como tá propondo o ministério da fazenda e estimular

que com esse recurso a gente possa ampliar a curto prazo é a curto médio prazo

aí é uma meta usada para aí 6 anos a gente poder chegar ao patamar sair de 11%

chegamos a 15% agora em 2023 mas elevar para 37% em 6 anos n o patamar da da

matrícula de Ensino Médio concomitante ou integrado ao ensino técnico e é claro

João que é preciso ter olhar para todos os estados a gente tá discutindo a

possibilidade nesses estados haver uma implementação por parte do Ministério da

Educação de recursos para para que cada estado possa atingir a Mia de 37% nós só

temos um estado no Brasil hoje que é o Piauí que atingiu a meta de 37% de ensino

técnico eh vinculado ao ensino médio eh outros estados estão perto disso mas tem

estados que estão muito longe portanto é uma estratégia que nós estamos

construindo portanto faz parte repito de um conjunto de ações que nós estamos

olhando uma visão sistêmica desde a creche até o ensino superior mas hoje é

fundamental darmos um salto O Brasil precisa dar esse salto rápido e aproveitar

é o período Jovem do nosso país a força da qualificação da a qualificação da mão

deobra e passa pelo ensino técnico nesse país muito obrigado obrigado Ministro

só eu só ultrapassei um pouquinho o tempo que eu vi que todo mundo ultrapassou

temos a nossa margem de erro a a má notícia é que depois vai comprimir o tempo

para as considerações finais mas a gente tenta aproveitar agora no debate Lu por

favor é bom dia Ministro Bom dia Governador Bom dia secretário Fernando bom dia

à nossa audiência eh eu queria fazer começar a fazero uma pergunta pro uma

pergunta dois em um pro Ministro Camilo e pro secretário daril eh volta e meia a

gente vê pressões aqui em Brasília para repactuar a dívida dos Estados com a

união Mas essa é uma iniciativa inédita de fazer essa esse bem bolado aí com

investimentos em educação eu queria que vocês compartilhassem com a gente como é

que surgiu essa ideia dessa dessa articulação Como ela foi levada ao presidente

Lula e aí eu vou aproveitar então e aí pedir ao secretário daril para responder

a pergunta que ficou quicando na área aqui do do governador e e também há uma

crítica que se faz ao programa que ele novamente né Eh favorece estados muito

endividados e o ministro Camilo se puder falar um pouco de como é que tá a

conversa com os estados a receptividade dos Estados essa proposta se já tá sendo

articulado contar um pouco pra gente sobre como é que tá a implementação

Obrigada começo respondendo aqui ao que me cabe depois peço ao Ministro que me

complemente veja que nós estamos a 1 ano e 4 meses de uma de uma nova gestão

nesse nesse período me permito aqui dar um contexto que a gente tem trabalhado

uma pauta extensa no país junto ao judiciário ao Congresso mas foi feita uma

recomposição da perda que teve do ICMS combustível aos Estados foi feita uma

pactuação no Supremo foi antecipado o ano passado isso a todo o conjunto de

estados o ano passado o presidente Lula nos pediu que complementasse o fpe e o

fpm para além da do que tava na inércia do que era devido ano passado para que

não houvesse eh perda real então o estados e municípios receberam complementação

dos Fundos e a gente vem trabalhando numa linha e aí chego na na ideia do juros

por educação que é uma linha de contencioso que tava presente no país enfim o

Brasil é um país que tem muito contencioso a gente tem tratado disso do ponto de

vista tributário as empresas e o fisco e os contribuintes pessoas físicas eh e

as receitas dos Municípios estados e da união e uma das lógicas eh do do

Ministério da fazend a reforma tributária aponta para isso é reverter esse

padrão de de ineficiência da economia brasileira em que se aposta no conflito se

gasta muito tempo se gasta muito tempo com eh pagamento de encargos e se

privilegia pouco a discussão do que é principal e a discussão de do juros por

educação nasce Nesse contexto o que a gente recebe desde o começo em que os

estados se mostravam eh muito fragilizados pelas decisões eleitoreiras do do

governo anterior era de ter uma reposição paraos estados isso foi feito houve um

atendimento pronto que foi feito pelo governo federal e a gente vem discutindo

isso na esteira dessa primeira ajuda que veio do ano passado começa a surgir o

tema das dívidas dos Estados eh em especial daqueles estados mais endividados

que já tavam judicializado mais uma vez o tema na Suprema corte e isso movimenta

o Congresso Nacional Então veja seja a gente poderia seguir apostando na mesma

dinâmica que a gente sempre viu no Brasil né a gente tem as leis aprovadas pelo

congresso os estados entram nos regimes de recuperação a gente acompanha como

Executor das políticas os regimes de recuperação aponta que os estados não

cumpriram aquelas diretrizes eh pune o estado o estado vai ao supremo o Supremo

muitas vezes permite que haja uma uma moratória ou alguma espécie de suspensão

do pagamento pros Estados e a gente aposta no conflito então dado esse

diagnóstico o ministro hadad foi prefeito de São Paulo e viveu essa situação

enquanto prefeito de São Paulo São Paulo tinha uma das maiores dívidas eh era um

dos municípios mais endividados do país e depois Graças às boas medidas da

gestão dele na prefeitura eu e o seron Távamos lá com ele hoje São Paulo é dos

das cidades que mais bem avaliadas em termos de avaliação de risco do país e

passou por uma negociação de juros com a união então a ideia vem desse contexto

num crescendo de complementação pros Estados o que perderam em 22 complementação

dos fundos para além e da da reposição do ICMS combustível e deixar de apostar

no conflito não queremos ficar discutindo no Supremo a de eterno Qual é a taxa

de juros que os estados têm ou não Ou não pagar até porque muitos estados não

vão conseguir pagar e essa é a mensagem que chega Clara Então por que não

apostar numa novidade essa é a grande questão aqui por que não apostar no novo

muito sensível ao caminho que a gente quer de desenvolvimento pro país então com

esse impulso do ministro adad a equipe do Ceron e o Ceron foi peça fundamental

nesse projeto vamos remodelar em vez de insistir no contencioso insistir na

cobrança de juros em cima de juros juros compostos Vamos abrir um estímulo eh

pro bem do país então nasce desse aprendizado que a gente teve tanto na

prefeitura quanto na no Prim ano de gestão e estamos desenhando um programa

apresentamos a proposta é importante dizer que nenhum estado vai ser Deixado

Para Trás não é essa dinâmica que a gente vem trabalhando não vai ser essa

dinâmica que a gente vai trabalhar aqui então os estímulos já estão colocados

seja do MEC na fazenda a gente tá tentando ver linhas de financiamento especial

para que os estados consigam os estados que tenham menos dívidas importante

dizer que todos os estados Têm alguma rolagem de dívida com a união é natural

que isso exista Mas claro que um tem menos que outros para esses que T menos

eles vão ter um benefício proporcionalmente menor porque a dívida é menor mas

vão ser abertas outras frentes de trabalho para que o objetivo final eh do

ensino médio seja alcançado então procurei aqui responder tanto de onde vem a

ideia e é importante dizer que é uma ideia aberta hoje é uma ideia que não é

mais nossa do ministério nem do governo nem do ministro Camilo é uma ideia que

tá aqui aberta ela vai ter que passar pelo crio do congresso nacional e é

importante que a gente enriqueça a primria o debate tendo em vista esse essa

direção que a gente quer alcançar bom além da do que o daril colocou eu quero

cumprimentar aliás cumprimentar minha secretária executiva secretário Gregório

seron foi fundamental nesse processo nessa discussão eh nós sabemos as

limitações orçamentárias que nós temos né a importância de garantir o equilíbrio

fiscal do estado brasileiro mas a importância de se investir na educação né E aí

é um debate eh que precisaríamos de mais tempo pra gente fazer uma discussão

mais profunda Então eu acho que o a a a a ideia do programa é usar um pouco da

criatividade né para encontrar ferramentas e mecanismos para implementar o

orçamento no investimento da educação brasileira né Eh repito diante das Lim

ações orçamentárias que hoje o Ministério da Educação tem então por exemplo a

gente tá investindo hoje R bilhões de reais num programa de escola de tempo

integral Imagina aí para atingir a meta do Plano Nacional de Educação que era

meta para ser atingida agora em 2024 então nós estamos nós precisaríamos de 3.

600.000 novas matrículas na Educação Básica tempo para atingir a meta que foi

estabelecida há 10 anos atrás portanto nós temos tínhamos uma a meta do pne que

era triplicar o número de matrícula de ensino técnico Brasileiro né talvez não

conseguimos aí chegar a aumentar 20% disso nesses nesses últimos 10 anos então A

ideia é se você for fazer uma conta algumas projeções que o ministério da

fazenda e o Ministério da Educação estão fazendo ranqueando estado por estado

dívida de cada estado brasileiro né e o que é o que seria necessário o número de

matrículas que a gente tem hoje para ampliar para chegar cada estado A 37% então

é só coloca fazer a conta o valor por exemplo o estado da Paraíba tem uma dívida

praticamente a dívida pequena então ele precisaria de uma para ele chegar a 37%

com apoio da mesma proporção que os estados mais individuais estão tendo ele

precisaria de um aporte Extra né porque a dívida dele não é não seria pelo valor

tá está se calculando a hora aula do do do para para estimular o ensino técnico

um percentual que serja feita pela Rede Pública estadual um percentual que ser

feito pelo sistema S portanto fazendo essa essa equação e garantindo aí a

ampliação disso então vai terá que ser uma decisão que o governo terá que tomar

se nós vamos manter aí e E olhe que se eu reduzir de 4 para 2% e eh o juros eu

para eu garantir a ampliação dessas matrículas para chegar a 37% Eu só preciso

usar algo em torno de 1/4 do valor né da redução desses juros desse pagamento

dessa dívida não sei se você estão entendendo né É como se eu tivesse assim 100

bilhões vamos dizer assim em dívida reduzindo de 4 para 2% a economia que vou

ter dos estados que Eu precisaria de 25 bilhões para ampliar as matrículas dos

Estados para chegar a 37% padrão da ucd então eh eh E agora tem estado que vai

ter recursos para fazer isso por conta da redução Ainda vai sobrar e vai ter

estado que não terá então essa equação que o João tá colocando que nós

precisamos avaliar há uma receptividade muito positivo Por parte dos Estados dos

governadores mas há um questionamento pros Estados que fizeram o dever de casa

que não tem uma dívida eh pequena com a união todos têm acho que quase todos têm

como estados tê mas eh eh eh eh eh e a gente nós estamos vendo de que forma o

Ministério da Educação poderá ter recursos extra para ampliar para proporcionar

na na mesma igualdade dos estados que vão receber essa redução para garantir eh

eh eh a implementação desse programa Então acho que é vamos dizer assim usar um

pouco da criatividade do governo para garantir a ampliação dos investimentos na

educação pública desse país eu acho que é uma uma parceria importante que eu não

tenho dúvida que trará resultados fundamentais para a economia e para sociedade

brasileira Lu perfeito e eu ia fazer uma pergunta pro Governador mas antes só

para não deixar desamarrado uma uma uma dúvida uma curiosidade o secretário

durigan comentou dessa necessidade de articulação com o congresso para que esse

programa siga adiante em breve eh em relação ao judiciário existe também uma

expectativa eh de articulação uma vez que eh enfim com frequência a gente vê

estados recorrendo eh algumas eh também decisões liminares suspendendo pagamento

de juros ou seja o incentivo também teria que vir do outro lado da da Praça dos

Três Poderes qual que é a perspectiva dessa possível articulação se é que ela

existe como ela se daria eu acho que a resposta a isso é a valorização desse

diálogo institucional o governo como um todo mas em especial Ministro Radade tem

tido um diálogo com o judiciário e aqui eu não vou gastar mas a gente acho que

talvez desde o começo da gestão não teve uma grande decisão que não teve diálogo

a gente não preparou material para que o o próprio Ministro muitas vezes fosse

ao Supremo fosse ao STJ mostrar as razões de porque determinada decisão tinha um

impacto pro interesse coletivo dessa ou daquela forma Então eu acho que

modernizar o arcabouo legislativo com programas meritórios com programas que

apontem na na direção certa que inspirem e essa inspiração ela é de todos nós

mas ela também chega e nos outros poderes Eu acho que isso renova o arcabouo

legal e também a a gente espera que renove a o ciclo de decisões o ciclo de

dessa dinâmica de contencioso o que a gente espera é que a gente consiga abrir

espaço para um país que tenha menos menos eh eh menos custos de encargos de

litígio e mais gasto em educação Então passa por aí e passa por um convencimento

que evidentemente passa pelo jogo democrático por argumentação por

sensibilização e menos por por qualquer tipo de imposição mas a sensibilização

do Judiciário dentro desse de todo esse contexto dessa construção que começa no

legislativo já tá sendo feita por nós nas várias frentes de de diálogo já

abertas eh queria aproveitar então Eh mais uma pergunta que ficou aqui queria

saber do governador eh Azevedo se as respostas se o senhor ficou satisfeito com

as respostas que deram e em segundo lugar queria uma uma avaliação sua como como

Governador se essa proposta ela é capaz de mobilizar o todos os estado o seu

estado investe muito mas o senhor acha que isso Eh cai no agrado dos outros

governadores é claro que a a as respostas aqui elas vão na direção daquilo que a

gente precisa ouvir realmente de que os estados que fizeram o seu dever de casa

terão terão eh também um tratamento que seja diferenciado em função do trabalho

que foi feito isso precisa ser feito a Paraíba tem 28% dos alunos de Ensino

Médio já associados ao ensino técnico então nós temos ainda uma caminhada Grande

para fazer o que o programa prevê como eu disse conceitualmente eu acho que

todos os governadores concordam agora important entender que volumes de recursos

considerando que o que vai ser levado em consideração para aplicação dentro do

programa será a economia gerada pela redução do juro se você imagina um estado

qualquer que tem uma uma dívida gigantesca se ele reduzir 1% ele vai ter também

um volume de recurso gigantesco para aplicar diferente de um estado que tem uma

dívida pequena que 1% será um valor pequeno e como fazer essa compensação Essa é

a questão precisa ser colocada até porque dentro do próprio programa n na

proposta do programa prevê inclusive que para o estado caso o estado não consiga

aplicar aquele valor no montante maior do que 50% previsto ele poderá ser

utilizado em parcerias com universidades e Outras aplicações fora do ensino

técnico E aí no caso de estados que T dívidas elevadas em função do montante que

que é muito elevado poderá haver eu não vou chamar desvio não desvio é uma

palavra muito feia mas poderá haver um direcionamento muitas vezes desses

recurso para um para outras áreas que não o ensino técnico que nós queremos

priorizar aqui então são apenas observações que eu acho que precisam ser levadas

em consideração e lógico que isso vai passar ainda por dentro do do da pela

análise dos principalmente de dos consórcios que representam os estados todos o

Brasil é uma é uma é um local em que se discute de uma forma muito intensa

qualquer projeto que o governo federal apresenta secretário falou aqui das

compensações é claro que as compensações o que nós perdemos com a lei

complementar 192 e 194 Não Foram compensadas ainda há uma distância muito grande

os municípios mais estados menos não tiveram sua suas seus prejuízos corrigidos

como for por caso dos Municípios mas tudo isso é uma outra é uma outra discussão

acredito aposto no programa acho que é importantíssimo a gente precisa só ter

esse Cuidado para não se criar mais uma vez tratamentos diferenciados Ministro

queria saber eh do Senhor qual que é agenda que o Senhor trabalha de articulação

com com os estados enfim enfim qual que é quais são os próximos passos para

tentar engajar os diversos governadores se o presidente Lula também pretende de

alguma forma eh encabeçar eh eh esse processo chamando reuniões no Palácio do

Planalto o que se pode esperar dessa desse processo de articulação para dar

tração ao programa Eu acho que o primeiro passo foi dado com a reunião que o

ministro Haddad teve com os governadores né depois hoje também houve uma reunião

com os governadores também que não porque ele fez a reunião com os estados mais

individados teve também a reunião com os governadores que têm dívidas menores

apesar de todos terem dívidas com a união e agora um pouco a gente dentro

internamente dentro do governo definir as regras eh eh estabelecidas né de como

é que será o tratamento né para os estados que TM dívidas vou dar um exemplo

aqui claro que isso aqui são são estimativas porque eu tenho hoje estados por

exemplo tem 5% de matrículas de ensino técnico eh eh concomitante ou integrado

com o ensino médio tem estados que já tem 38% como o estado do Piauí por exemplo

Então você tem realidades diferentes então se você pegar também a realidade do

Estado de São Paulo que se eu reduzir aqui eh vamos dizer assim de 4 para 2% a o

juros da dívida eh que ele vai ter aqui um uma redução de 26 em C em 6 anos de

25 a 2030 a estimativa é que ele tem uma redução de 26 bilhões da dívida mas

para atingir a meta dos 37% ele precisaria apenas de R bilhões 600 milhões deais

então tem estados que tem condições e aliás sobra né recursos para isso tô

falando de São Paulo porque ano 2022 60% dos alunos do ensino médio de São Paulo

queria ensino técnico e só conseguiram 5% de vaga né 5% desses alunos conseguiu

no ensino técnico em São Paulo tô vendo aqui o secretário Ron do Paraná que tem

uma política hoje de estimular o ensino técnico Bras aliás O Grande Debate da

reforma do novo ensino médio foi garantir que pudéssemos estimular ampliação na

carga horária ao ensino técnico profissionalizante no ensino médio brasileiro

Esse foi o grande tema dos secretários e secretárias da educação eu sempre digo

que nenhuma política pública ela pode ser construída e implementada sem se

diálogo com os entes Federados Diferentemente por exemplo Como como o estado

como Distrito Federal que tá aqui a secretária que num estimativa tô falando

aqui só dados estimativos não é dados eh que tem uma com essa redução dos juros

teria uma uma redução no juro da dívida de 100 100 102 bilhões mas para

implementar para chegar a 37% porque o Distrito Federal hoje tem apenas 88% eh

de alunos do ensino médio com comitantes ou integrado ao ensino técnico para ele

chegar a 3 7% ele precisaria eh de nesses nesses de R 302 milhões de reais então

você vê ele não tem hoje com a redução os recursos só tem um terço dos recursos

suficiente para implementar isso então é um tratamento nós vamos ter que definir

nesson dentro do governo de que forma se nós vamos ter condições de de

complementar esse recurso é o que eu defendo Governador n a gente possa

complementar aos Estados eh a complementação Claro que tem limitações

orçamentárias e é uma coisa que nós vamos discutir internamente dentro do

governo mas precisa ter algum estímulo pros Estados que não ten as dívidas

suficientes para para cumprir essa meta de chegar eh a cada estado 37% o que nós

também podemos fazer é chegar a m a média Nacional a meta Nacional a 37% se tem

estados que vão ter que passar por exemplo São Paulo tem condições de passar

muito mais de 37% pelo pela redução dos recursos do juros então é um tema que

nós vamos ter que eh luí deixarmos aqui e o presidente Ministro adad bateu o

martelo e assim a gente poder eh convocar todos os governadores todos os estados

conversar com o Congresso Nacional dialogar com com os presidentes das casas mas

eu não tenho dúvida que é um tema muito bem aceito muito bem visto né Eh eh pela

pelo congresso pelos Estados né então acho que é um darmos agora acelerarmos

para que a gente possa implementar essa política que nós temos sempre defendido

no min Ministério é a necessidade quando a gente fala do tempo integral a gente

tem estimulado para os estados que que implementaram a matrícula de tempo

integral no ensino médio seja um ensino técnico a gente estimula isso mas quem

define é a rede né é o diálogo com a rede ele que vai definir se quer

implementar ou ampliar a matrícula de tempo integral se é no ensino fundamental

um ou dois ou Ensino Médio Mas quando é no ensino médio a gente estimula o

ensino técnico mas eu não tenho dúvida eu não tenho dúvida quero dizer aqui pros

senhores e para as senhoras aqui presentes da necessidade do país hoje dar um

salto no ensino técnico na qualificação da mão de obra pelo efeito que vai ter

do ponto de vista social e econômico do nosso país é evitar evasão dos alunos eu

digo isso eh eu fui um áo defensor da poupança do ensino médio e aliás quero

aqui parabenizar o presidente por uma decisão pessoal do presidente chegar oo

patamar que nós chegamos e agora anunciar que vai ampliar para todos os alunos

do Cadú issso significa gente a gente procurar garantir que esses jovens

permaneçam na escola garantam a qualificação técnica ao sair da escola e

estimular que no ensino médio repito a gente possa garantir esse saldo que o

Brasil precisa tá aí os estudos mostrando o efeito que tem no PIB a cada 1% que

aumento estudo da Fazenda cada 1% que é o aumento do ensino da qualificação

técnica né eu posso ter um potencial de aumentar 0.3 032 do PIB brasileiro para

se aumentar 10% são 3.2% do PIB brasileiro aumentado isso tem um efeito no

emprego tem efeito na exportação tem emprego na renda Enfim uma série de de de

fator aumente 20% a remuneração de quem tem qualificação profissional de quem

não tem isso aumenta também a qualidade da produtividade de um país Portanto tem

um efeito gigantesco que o Brasil precisa rapidamente dar esse salto de

qualidade esses próximos anos no nosso país Luiz obrigado eu tenho uma pergunta

aqui eu não sei quem poderia responder eh vou jogar aqui para para ficar a

critério Eu queria saber como é que vai ser a articulação do programa com o

setor privado eh o tá uma das Linhas eh do programa é justamente ou fazer uma

articulação com empresas com setores para ver qual é o curso mais adequado Então

eu queria queria entender como é que vai ser essa articulação Obrigada Não

primeiro é fundamental e essa é uma discussão que a tem feito no ministério é

focar e a a a os cursos né Eh a oferta de cursos dentro da da Necessidade hoje

do mercado de trabalho né Nós temos hoje por exemplo uma uma demanda enorme eh

na área de Tecnologia da Informação hoje falta mão de obra para essa área nós

precisamos olhar paraas novas matriz energética do Brasil para energias

renováveis né enfim então cada região tem o seu tem suas potencialidades né

Então esse é é um debate cada estado hoje tem feito essa discussão é fundamental

envolver o sistema S nesse processo né envolver a indústria envolver o setor de

serviços envolver né outr o setor de comércio é importante pra gente não não

ofertar cursos que não vão ser muitas vezes absorvidos pelo mercado de trabalho

dentro do do novo mundo do trabalho que nós estamos vivendo hoje das mudanças

que são rápidas que estão acontecendo eh no mundo inteiro então é fundamental

nesse processo de definição das áreas que serão ofertadas para os jovens do

ensino médio técnico definir eh potencialidades regionais uma região norte é

diferente da região sul é diferente da região Nordeste que hoje tem uma vocação

enorme para energias renováveis Então isso é fundamental para o êxito e o

sucesso do resultado efetivo nós vamos ter lá na ponta né Eh com esse com essa

com essa ampliação da mão deobra qualificada do ensino técnico portanto Lu é

fundamental que haja esse processo né portanto é um conjunto de ações e vai

precisar muito repito da participação dos setores não governamentais nesse

programa da participação dos Estados dos governadores dos secretários e

secretárias Estaduais de educação né nesse esforço coletivo do setor Empresarial

né das federações das indústrias do comércio né para que a gente possa

fortalecer essa política que eu considero inovadora né Eh que é esse juros por

educação Nesse contexto acho que é importante a gente ouvir o governador eh até

enfim resguardando o nosso cuidado com o tempo para que todo mundo possa fazer

suas considerações finais mas Governador o senhor citou essa ponte com o setor

privado na Paraíba né Eh acho que é importante a gente compreender um pouco como

é que isso se dá na prática o senhor chama as entidades eh setoriais O senhor

ouve as grandes empresas O senhor já tem alguma eh algum indicador de como eh

isso pode gerar mais produtividade ou enfim eh eh além do acolhimento dos dos

dos jovens profissionais que chegam ao mercado de trabalho isso se dá já eh como

é possível medir eh do ponto de vista econômico é primeiro é entender que nós

temos a partir inclusive da relação com a Fundação Itaú a identificação das

necessidades e para onde nós devemos direcionar os cursos que tipo de curso por

região do estado porque você terá um envolvimento do jovem muito mais fácil a

inserção no mercado de trabalho o outro ele se ele ele é voltado muito mais para

o aluno que conclui esse curso técnico e precisa ter um estágio e esse estágio

nós fazemos um estágio remunerado o governo do estado paga esse esses recursos a

empresa então a empresa tem uma mão de obra que não custa nada que não lhe custa

absolutamente nada e que normalmente é uma boa mão de obra sai de cursos

técnicos e a partir daí o que nós esperamos é que essa mão de obra seja

incorporada a mão de obra da própria empresa então em função de cada curso na

área da Agricultura nós procuramos as empresas que possam absolver aqueles

alunos na área de tecnologia nós temos ampliado muito essas essa questão de

Formação fechamos recentemente um protocolo com a How para treinar 30.000 jovens

na área de TI na Paraíba isso abre um uma possibilidade extraordinária mas nós

fazemos isso em função volto a dizer do estudo que é feito que identifica Quais

são os eixos e o contato com as empresas não foi me perguntado mas voltando um

pouco ainda o programa juros da educação Eu acho que o Brasil tem com esse

programa uma possibilidade gigantesca para para o enfrentamento dessa questão do

ensino técnico até porque do próprio programa é proposto que caso se obtenha o

resultado e se atinja a meta aquela redução de juro ela passa a ser permanente

daí PR frente se ela vai passar a ser permanente daí PR frente você tem que

imaginar que cada estado em função da sua dívida terá um volume de recursos

muito grande e era importante que no programa desse uma indicação pelo menos de

como deveria ser investido esse recurso que parte já que você atingiu a meta mas

depois depois eu vou só o eu como eu ainda bem que só tenho eu Governador aqui

que teria senão 26 governadores dizend assim esse cara tá jogando contra os

governos não eu não tô dizendo isso que eu tô dizendo é o seguinte se tem um

programa que tá dizendo assim você vai ter uma economia de juro para aplicar na

educação se você atingir a meta depois você fica só com o bônus não amigo você

tem que continuar Investindo na educação Então eu acho que a gente tem que

pensar nisso porque senão você vai ter um uma receita gigante depois para os

estados que T grandes dívidas e a educação pode ser que não tenha essa esse

recurso chegando então vamos perguntar aqui pro secretário Dario eh como é que

vai ser feito esse acompanhamento a pergunta aqui o governador fez as melhores

perguntas aqui do painel quero dier e e e e já já conseguiu engatar o próximo

próximo painel ele já esquentou as os motores pro próximo também então pro

secretári falar como é que vai ser esse acompanhamento né de como é que vai ser

o investimento eh no no no no ensino profissionalizante e o que que acontece

depois de atingid a meta obrigada eu acho que é importante ponderar aqui a as

duas frentes né que é o estímulo à educação o estímulo ao ensino médio

profissional e a autonomia Federativa evidentemente que aqui como você tá

tratando de um de um auxílio da União né de um de um gesto da União num esforço

nacional é preciso condicionar estimular para que a gente atinja o objetivo e é

importante ponderar que no futuro atingindo esses objetivos é preciso manter o

patamar então a manutenção é é preciso não ter retrocessos do ponto de vista da

manutenção do ensino médio profissionalizante E com isso não tendo retrocesso o

estado ganha em autonomia para desenvolver outras eh outras políticas do do

próprio Estado mas sem dúvida que o acompanhamento vai ser feito pelo governo

como um todo o Ministério da Educação Ministério da Fazenda em especial o

Tesouro Nacional como a gente já tem essa expertise em acompanhar outros planos

né alguns estados T planos de recuperação fiscal conosco a gente tem já um

padrão de acompanhamento no que se gasta se o estado tá gastando algo para além

do plano aprovado de recuperação fiscal e essa essa articulação que muitas vezes

se se perdeu nos últimos anos ela é feita de maneira tripartite no na na

recuperação fiscal Você tem um representante da União um representante do

próprio Estado e um representante do Tribunal de Contas da União em que os três

tomam a decisão de encaminhar de propor quais os caminhos para para para pros

pro futuro da recuperação fiscal do Estado então a gente já sabe fazer esse tipo

de avaliação vai seguir acompanhando e com com a expertise que temos e é

importante dizer que não pode ter retrocesso e o prêmio pro estado que mantiver

o nível de Educação no patamar alto é eventualmente também abrir espaço para

para outros investimentos e que o próprio estado demande secretário O senhor já

não quer iniciar a rodada de considerações finais a gente tá com já com o tempo

correndo aí a gente faz uma uma última rodada já que acho que vale agradecer

mais uma vez a oportunidade de est aqui agradecer o ministro agradecer o

governador Lu e exman eh O que eu o que eu gostaria de dizer no fim é o seguinte

a gente tem agora olhando pro pra situação do Rio Grande do Sul eh de novo a

dívida dos Estados aparece agora com a outra vertente a vertente de

eventualmente pensar em saídas para pra dívida quando a gente passa a ter o Rio

Grande do Sul é o exemplo que a gente tá vivendo agora mas a gente passa a ter

um outro modelo eh e um uma outra demanda de resposta a emergência climática

então a gente também tá estudando alguma possibilidade de ter um arcabouço mais

sofisticado para todos os estados do ponto de vista Nacional em respostas

mudanças climáticas que venham de uma mesma lógica de pensar um Brasil e do

amanhã um Brasil que já tenha os e as as previsões os arcabouços os gatilhos

colocados eu reforço aqui a o compromisso com o diálogo federativo a gente tem

tratado por exemplo para dar o exemplo de novo da reforma tributária que

necessariamente tem que ter um diálogo tem que ter consensos construídos com os

governos dos Estados com as prefeituras então o compromisso de diálogo

federativo e de tentar fazer esses grandes pactos seja na tributária seja no

juros pela educação seja na recomposição ao que foi tirado dos estados e

municípios no governo anterior que evidentemente o governador mencionou a gente

tem feito isso na medida das forças do orçamento de tudo que é que é possível do

governo federal mas o diálogo federativo tem tem tem sido reforçado o presidente

Lula tem chamado governadores os mais diversos matizes ideológicos paraa

composição para pensar no estado que é importante pro estado e esse é o caminho

que a gente vai seguir adotando até o fim do governo Obrigado nós que

agradecemos Governador por favor as suas considerações finais não só agradecer a

oportunidade de est aqui participando contribuindo minimamente com o debate de

um tema tão importante eu não tenho dúvida nenhuma que a educação é o caminho

para qualquer país se desenvolver Então eu tenho apenas esse fim não é para mim

não né tá aparecendo aqui eu fico pensando aqui que isso aqui é um debate aquela

coisa do debate que fica lá pra gente ter responder ainda falta ainda falta um

tempo ainda falta tempo debate mas mas o trauma é o mesmo mas só agradecer

realmente a oportunidade de conversar um pouco aqui a respeito dessa dessa pauta

que para mim é importante eu sou ex-professor sou professor aposentado do

Instituto Federal da educação eu sei o que é a importância da educação eu sei o

que é tá dentro de uma sala de aula e isso para mim me motiva cada dia a gente

buscar com todas as dificuldades que nós vamos ter uma ampliação do de de de um

ensino técnico isso gera uma demanda de professores da base técnica enorme isso

todo mundo aqui que que passa na secretaria sabe como é difícil você ter

professores de base técnica para atender a uma demanda dessa então nós vamos ter

que pensar conjuntamente numa formação de professores para atender a essa

demanda também enfim são vários vários temas que poderiam ser discutidos aqui

mas agradecer pelo convite senti extremamente honrado de participar aqui eh com

todos vocês muito obrigado obrigado Governador senhor sempre bem-vindo Ministro

por favor não eu queria só agradecer o convite valor P agradecer a Priscila Ana

Luiz a Lu e agradecer também aqui a Dario em nome do Ministério da Fazenda que

tem sido um grande parceiro e dizer que a ideia dessa política ela faz parte

Lembrando que ela faz parte de uma estratégia maior de um olhar mais sistêmico

da educação brasileira principalmente o olhar do ensino médio brasileiro porque

o que motiva um aluno a permanecer na escola né É se a escola o atrai se a

escola gera expectativa para ele né sonhos e repito A grande maioria dos alunos

do ensino médio Brasileiro hoje desejam ensino técnico profissionalizante então

além do efeito econômico além do efeito social nós vamos garantir somado repito

a outras políticas que nós estamos desenvolvendo n como pé de meia como a

poupança como pque paraas escolas para melhorar suas infraestruturas como o

programa de conectividade nas escolas Qual é o jovem hoje que não quer ir para

uma escola conectado né então qual o jovem hoje que não quer ir para uma escola

que tenha atividade esportiva que tenha atividades culturais que complemente o

seu projeto de vida então o jovem precisa acordar de manhã e ter vontade ir pra

aula ter vontade pra escola escola né então esse programa vem Se somar a um

conjunto de estratégias de forma criativa que amplie os recursos de investimento

né do dos estados e da educação brasileira mas repito quero aqui finalizar

dizendo a importância da relação Federativa para qualquer política pública

Educacional brasileira é decisão política né quando o governador quando o

prefeito quando o presidente decide investir priorizar porque a gente precisa às

vezes sair daquela retórica educação é importante o Brasil eh eh se transforma é

o caminho mas a gente precisa ir na prática para garantir que essa essa e vamos

dizer esse discurso possa se efetivar na prática né Em cada estádio em cada

município brasileiro de forma inteligente sempre focado em resultados medir os

resultados nenhuma política pública ela tem efetividade se nós não medirmos os

resultados e avaliarmos constantemente as políticas a gente precisa melhorar

corrigir isso só se faz com diálogo com relação Federativa E esse tem sido a

nossa orientação dentro do Ministério da Educação com todos os estados e

municípios brasileiros Independente de questão política partidária a educação

precisa tá acima disso né é olhar pra qualidade da aprendizagem da melhoria e do

acesso ao povo brasileiro ao que é dever do Estado é garantir o mínimo que nós

precisamos garantir pro povo brasileiro é uma formação de uma escola pública de

qualidade para todos os jovens e crianças e jovens desse país eu não tenho

dúvida que a gente precisa investir mais em educação eu tenho feito essa defesa

até porque se a gente for pegar os dados divulgados recentemente pelos países

pela CDE mostra que o Brasil já investe per capita no ensino superior a média

dos países da uc que é algo em torno de 11.000 e poucos dólares mas no ensino

básico nós investimos 1/3 disso mudança evidência já mostra a necessidade de

investirmos mais na educação básica por isso esse olhar hoje do Ministério da

Educação para Educação Básica né de olhar necessidade porque esse jovem essa

criança que vai pro ensino superior que vai abrir as portas pra universidade pro

ensino superior então quero só encerrar Agradecer o convite parabenizar aos

organizadores que a gente possa fazer esse grande movimento né no país com a

sociedade civil com o setor Empresarial com setor público com o congresso

nacional com poder legislativo com poder judiciário liderado aí nessa nesse

compromisso do presidente Lula para que a gente possa dar avanços importantes um

país que não investe em educação né Tá afado ao insucesso a história já tem

mostrado isso portanto é fundamental a gente correr andarmos mais rápido o

Brasil é hoje para que a gente possa ampliar a qualidade o acesso e o ensino

técnico é fundamental é um passo importantíssimo nos próximos anos a meta aí até

2030 a gente possa alcançar Olha que é um é é um um prazo ousado para que a

gente mas isso vai depender muito da liderança de cada governador da liderança

de cada secretário e secretária da decisão política de cada um de querer olhar

eh e garantir esses indicadores esses avanços da educação pública do nosso país

e do nosso Estado então parabéns viva a educação e vamos ao trabalho obrigado

[Aplausos] gente muito obrigada pela contribuição Obrigada Lu Governador

Ministro secretário Fernando que continua conosco aqui pro próximo painel

próximo debate muito obrigada por favor uma pose para foto por gentileza Já já

vamos dar início ao segundo painel muito obrigada mais uma vez Bom dia a todos

continuando vamos ao segundo painel de hoje e nós vamos tratar vamos saber o que

é a proposta juros por educação ajustes já realizados nas últimas semanas seus

detalhes tcnicos Murilo camaroto repórter do valor estará ao lado de Fernando

desta vez fazendo a mediação do painel Murilo por gentileza Bom dia obrigada

como palestrantes teremos Gregório griza secretário executivo adjunto do

Ministério da Educação secretário Gregório por gentileza Obrigada também vai

fazer parte desta mesa Rogério Ceron Secretário do Tesouro Nacional do

Ministério da Fazenda por gentileza secretário sejam bem-vindos o palco de vocês

Bom debate Muito obrigado acho que no primeiro painel a gente teve um um

Panorama eh [Música] bem amplo embora profundo também a gente tem a oportunidade

aqui de detalhar ainda mais esse programa né Queria agradecer a presença do

Murilo eh a ideia aqui então a gente ter também tempo para considerações

iniciais pra gente poder detalhar o programa e depois a gente passa para um para

um bate-papo né secretário Gregório senhor pode começar por favor bom bom dia a

todos bom dia todas iniciar agradecendo o convite Luiz Murilo cumprimentando

colega ceon eh nessas palavras iniciais eh e o e o objetivo parece ser aqui

tentar alcançar um grau de detalhamento maior do que vem sendo construído e

concebido em relação ao programa dentro do governo falando em relação ao

Ministério da Educação a gente a partir das primeiras conversas ainda ano

passado com o ministério da fazenda a gente costurou vamos dizer assim o

esqueleto do que seria a proposta eh que foi levada ao presidente da república

foi apresentada aos governadores pelo Ministro Fernando Haddad e a partir daí eh

no que cabe mais a a ao nosso trabalho da parte mais técnica eh começaram um

conjunto de reuniões de atividades um plano de trabalho mesmo eh em especial no

caso do MEC com as secretarias Estaduais de educação iniciando com aqueles

estados com o volume de dívida mais elevado eh do Sudeste e Rio Grande do Sul né

e alcançando agora outras regiões do Estado tivemos encontros também com com

secretário Ceron com secretários Estaduais de Finanças né E de de fazenda e de

educação juntos pra gente eh dar o start nessas nessas atividades e a gente tá

nesse processo ainda de ouvir eh as redes estaduais sobre aquilo que melhor as

atende em relação à implementação dessa proposta cuja idealização eu acho que

ficou muito clara no primeiro painel que tem como prioridade Total expansão das

matrículas de ensino técnico articulado ao ensino médio articulado é a palavra

da legislação né da da Lei diretrizes e bases e por isso que o ministro repete

bastante a ideia de concomitante ou integrado ao ensino técnico aqui já dou um

primeiro eh informe do ponto de vista Educacional portanto a gente tá

priorizando eh matrículas concomitantes integrados e não a matrícula subsequente

que é um curso de natureza média mas ele requer a formação média para quem vai

ingressar isso está em debate ainda mas a princípio eh esses cursos hoje atendem

no Brasil um perfil de estudante acima de 28 anos em média né ingressantes com

28 anos e não é o perfil do estudante do ensino médio e quando a gente fala da

Meta da ocde de 37% a gente tá dando um dado que tem um recorte etário de 15 a

19 anos ou seja a média do CDE de 37% é justamente da faixa etária do ensino

médio tanto que se a gente olhar a média da ocde expandida para 24 anos ela é

uma média maior acima de 40% Então essas reuniões estão acontecendo a ideia

portanto já explicando um pouco da natureza das matrículas nós estamos falando

daqueles cursos que têm eh planejamento curricular único conhecido como

integrado né e os cursos concomitantes que podem ser feito por meio de de de

parcerias ainda tá no nosso cálculo as matrículas de de curso mal Priscila eh o

magistério né que são residuais mas ainda existem no Brasil e inclusive por

muitos concebidos como um curso técnico de educação também né então elas estão

no nosso nosso cmputo eh a ideia também na conversa com os estados é a gente na

medida em que caminha a idealização como Bem dito também no primeiro painel eh

de um projeto de lei complementar que precisa ser apreciado pelo congresso eh

como que a gente endereça enquanto governo esses dois grandes momentos o

primeiro da elaboração de um plano de trabalho por parte das redes estaduais

vocês devem imaginar E aqui tem secretários e pessoas que conhecem muito a

dinâmica eh de cada estado e região a realidade é absolutamente heterogênea você

tem estados com capacidade instalada para conseguir ampliar suas matrículas no

curto prazo se tem estados que terão mais dificuldades se tem estado que já têm

rodando eh diferentes naturezas de parcerias outros não TM uma escala em relação

a isso e também a diferença em relação à meta é muito grande então como como Bem

dito Então os planos de trabalhos eles necessariamente eles terão de ter a

peculiaridade da natureza de cada realidade eh estadual e o outro momento é a

ideia que também foi objeto de algumas perguntas eh de como você monitora isso

do ponto de vista racional sei que o que o que o Rogério vai entrar eh eh no

debate do monitoramento vamos dizer assim fiscal né mas a gente tem que

construir um um conjunto de mecanismos para monitorar a política e avaliar a

política pública Dando um exemplo de uma ação que tá que tá começando agora no

Brasil que é o pé de meia a gente tá num esforço de musculatura institucional

dentro do Mec para construir base de dados administrativos aferição de de de

frequência aferição de matrícula e muito Provavelmente nós vamos eh ou se

utilizar dessas ferramentas ou criar ter que criar outras para ter um

monitoramento a contento para não ser um monitoramento que que passe uma imagem

de que ele não é vamos dizer assim rigoroso em relação às às metas elencadas no

nos planos né outro outro ponto que que por Óbvio é objeto e tá muito presente

na fala do governador é a ideia do que acontecerá com aqueles estados cuja

dívida é tão elevada que provavelmente ele vai atingir aquela meta com recurso e

e e teria vamos dizer assim disponibilidade para outros investimentos estamos

conversando com com os estados sobre isso também eu tô chamando isso de gatilhos

ou seja aqueles gatilhos que podem ser acionados eh em relação aos Estados que

já cumpriram sua meta né e a gente tá focando muito especial ente na eh mantendo

a ept como grande ept educação profissional né como grande foco na ideia de você

eh ampliar por exemplo a infraestrutura de oferta de tempo integral em ept Esse

é um detalhe importante também que foi muito cobrado pelos secretários estaduais

eh inicialmente o programa tava sendo concebido para que essas matrículas novas

fossem em curso técnico e tempo integral mas como a realidade dos Estados é

muito diferentes e alguns conseguiram expandir eh o ensino técnico em tempo

parcial ou no meio do caminho eu brinco que é e ele não é nem parcial ou seja

manhãs apenas manhãs Mas ele também não é integral dois turnos inteiros né Tem

uma mediação de carga horária que é possível chegar então A ideia é a gente

iniciar eh não condicionando ao tempo integral necessariamente mas sempre o

fomentando sempre ou induzindo como o ministro já disse o próprio programa de de

indução de matrícula de tempo integral já foca em ept no ensino médio como como

fomento da da drenagem de recursos para as redes estaduais no caso do do

programa do tempo integral e finalizando um pouco essa essa Minha introdução eh

na agenda de plano de trabalho e na agenda de monitoramento a gente já tem

construído e tá eh e do debate com os estados também um conjunto de ferramentas

de alinhamento desses cursos com o mundo do trabalho de alinhamento desses

cursos com os projetos inclusive que que estão sendo lançados pelo governo por

outros Ministérios e pelo governo como um todo ou seja a coerência em relação ao

tipo de curso que é ofertado e e a demanda do mundo do trabalho a demanda do

projeto nacional que eu acho que o ministro Camilo e o secretário da ril deram

do ponto de vista conceitual o horizonte assim né a de ter coerência do perfil

de curso que a gente abre com o perfil de país que a gente quer pros próximos 20

30 40 50 anos né então todos os temas ligados a a transição ecológica todos os

temas ligados à à sofisticação eh de serviços precisa tá no horizonte dessas

ferramentas que o governo federal já tá produzindo para ofertar aos Estados o

que a gente tá chamando de ferramentas de alinhamento pra gente assessorar as

redes Estaduais na expansão da da matrícula ept nós temos a uma lei relacionada

à educação profissional que foi aprovada recentemente a mais recentemente ainda

foi criado um grupo de trabalho interministerial para elaborar a política

nacional de educação profissional e esse espaço né esse grupo de trabalho ele

também vai ser importante pra gente pensar a implementação do jur pela educação

que como bem disse o ministro a gente deseja ser breve eh eh resguardado e

ressalvado por Óbvio todo debate com as redes e com o Congresso Nacional eh em

relação a a a como o projeto de lei se se materializa e uma última frente que aí

é um é um advoca muito do da educação é a ideia de que se a gente conseguir

dentre os gatilhos que a gente tá propondo infraestrutura expansão de tempo

integral a ideia de você talvez nesses casos que já atingiram a meta conseguir

fazer um programa de elevação da escolaridade do brasileiro ligado a educação de

jovens e adultos integrada educação profissional não sei se não sei se é da

informação de todos mas eu reforço que o ministro disse nós temos uma França

inteira que não concluiu o ensino médio né são 70 milhões de pessoas no Brasil

nós temos um passivo Se você olhar um recorte muito curto entre 15 e 16 anos até

29 né que tá no Estatuto da Juventude é mais de 1 milhão de pessoas gente que

não sab ver então assim quando a gente olha estados como Minas São Paulo que são

densamente povoados eh se um gatilho permitir um processo de elevação da

escolaridade desses jovens e adultos combinado educação técnica que é uma meta

do Plano Nacional de Educação que não foi atendida a gente a gente interessaria

muito bem e o outro tema foi abordado também é a formação de de professores É tá

muito claro para nós hoje que não há como expandir a educação profissional na

escala e o desejo que a gente quer se a gente não formar professores para

educação profissional o Brasil não tem professores para educação profissional

suficiente hoje quem tá à frente das redes estaduais sabe disso dificuldade

queer contratar diretamente pessoas e aí a gente vai ter que provavelmente olhar

para um gatilho ligado à formação de professor como um todo que é um gargalo

Nacional Eu acho que o apagão docente é objeto de estudo de de todo mundo que

que tá na educação mas a formação para a educação profissional de professores é

uma necessidade premente de um país que quer expandir suas vagas Então a gente

tem esse esse tema como gatilho também no horizonte Para para pensar aí não

necessariamente eh em projeto de lei aí aver né ou em em regulamentação em infra

legal né A ideia é que a gente eh mantenha a parceria Ministério da Educação e

fazenda do começo ao fim em portarias interministeriais sempre entre os dois

Ministérios eh tem sido fecundo a parceria em outros temas e acho que essa aqui

é um é um tema chave que endereça projeto de país que endereça vamos dizer assim

visão de gestão pública e e essa consonância precisa ser objeto de conhecimento

da opinião pública preca conhecer isso né então inicialmente seria isso sobre os

Estados enfim não endividados e outros temas eu entro depois Muito [Aplausos]

obrigado obrigado temos também vários insights para aprofundar durante a nossa

conversa secretário Ceron palavra tá com o senhor por favor Obrigado Fernando

primeiro parabenizar o valor pelo pelo evento eh cumprimentar aí na figura do

Fernando todos os jornalistas do do do valor Profissionais que são que nos

acompanham aqui sempre estão envolvidos com as nossas discussões eh

profissionais sérios que que Ten o meu respeito e admiração cumprimentar a

Fundação Itaú e Todos Pela Educação na figura da Anne e da Priscila eh pelo

trabalho que fazem eh em relação à educação o Ministério da Educação o ministro

Camilo já já não está mais presente mas também registro aqui a a parceria e a

admiração pelo trabalho que ele vem conduzindo aqui o Gregório grande parceiro

na na em várias iniciativas como ele bem comentou e eu vou falar um pouquinho

sobre elas isso tem sido uma uma prática comum da gente na fazenda eh fazer ter

a a educação de fato o Ministério da Educação como um grande parceiro e e como

uma grande prioridade do governo nós estamos conseguindo acho que avançar nessa

nessa Seara então aqui eu queria destacar desde o começo né do do governo

algumas ações nós eh ajudamos e apoiamos o Ministério da Educação no desenho da

do pé meia que é algo fundamental e Fantástico eh para além dele nós eh lançamos

algo que que tem que tem uma menor dimensão Mas tem uma importância relevante

que é o título educar mais no âmbito do tesouro direto que é uma iniciativa para

valorizar E para trazer a luz do sol a discussão sobre a poupança voltada ao

financiamento eh do ciclo Educacional é algo muito relevante paraas famílias

isso tem tem tido um sucesso muito importante e ele tem até um casamento com com

o programa pmail os alunos osos jovens que estiverem beneficiados pelo programa

também poderão uma parte da poupança aplicarem nesses títulos isso gera um ciclo

eh Virtuoso muito interessante eh nós temos fomentado é um trabalho em conjunto

com o PPI eh a geração né o a difusão de conhecimento boas práticas e projetos

de parcerias público-privadas para infraestrutura Educacional que é algo que não

no Brasil eh é um setor que avançou muito as ppps e as concessões em várias

outras áreas no setor Educacional tem um potencial incrível de prover

infraestrutura e serviços de suporte a a parte pedagógica e nós estamos

avançando Provavelmente nós vamos chegar aí ao ao fim do do ciclo de governo aí

tendo mudado de fato a realidade com dezenas de projetos eh tendo saído do papel

e agora eh estamos nesse debate em transformar eh uma parte dos juros

relacionados às dívidas dos Estados em ente iOS para para gerar um um objetivo

Nobre né que é as a expansão das matrículas no ensino técnico o daril comentou

um pouquinho eu não vou repisar todos os pontos do ponto de vista de política

econômica mas fazendo eh corroborando tudo que ele disse eh nós temos Claro

desafios de curto prazo que estão sendo enfrentados do ponto de vista de

política econômica mas sem perder de vista o que importa que para nós o que

importa é o médio longo prazo criar condições para que o país tenha um

incremento de produtividade condições de um de um ambiente econômico saudável e

sustentável então e a a essa ação é mais uma Entre várias outras né reforma

tributária eh avanços institucionais em várias outras eh eh setores e reformas

regulatórias que criam esse ambiente favorável ao retorno do investimento

produtivo a uma economia saudável e com crescimento eh sustentado e com com

equilíbrio social mas essa ação efetivamente ela gera um ganho Fantástico do

ponto de vista de incremento da produtividade do trabalho a médio longo prazo eh

além de todos os outros benefícios que o o próprio Ministro Camilo comentou eh

de abrir a oportunidade da eh de incremento eh para ter uma profissão incremento

de renda etc etc eh ou como mesmo daril comentou ser um processo de emancipação

desses jovens né dar condições a ele por meio da educação at ter um futuro um

futuro melhor então nós estamos completamente alinhados em em torno desse

objetivo para descer um pouco e e falar um pouquinho da estrutura do que foi

pensado Claro o governador fez ponderações muito importantes e legítimas em

relação a quem fez o seu dever de casa e aqui Governador aproveitar para

registrar meus cumprimentos e parabéns pelo pelos resultados alcançados e pela

dedicação foco no no ensino no ensino Profissionalizante do seu estado isso é

muito relevante realmente transforma transforma vidas assim como como o senhor

comentou da sua formação técnica também passei por um processo semelhante

valorizo muito o ensino o ensino técnico profissional e o ensino público eh isso

transforma vidas de fato Então parabéns pelo pelo trabalho eh mas nessa nessa

linha de criarmos esses incentivos o Que Nós pensamos primeiro Claro os estados

que estão muito endividados de fato eles têm um benefício maior por outro lado

Eles estão numa situação como o daril comentou e como até a Lu provocou um pouco

sempre isso vai volta nessa discussão que acaba envolvendo muitas vezes o

judiciário e nós precisamos encontrar um caminho de saída eh para esse para essa

recorrente recorrente conflito claro que sempre vai haver a discussão de uma eh

desigualdade né em relação a quem legitimamente fez um um bom trabalho

reestruturou suas Finanças e acaba tendo um benefício proporcionalmente menor do

que aquele que está endividado e tá com uma uma gestão financeira um pouco mais

eh fragilizada Mas de fato nós precisamos encontrar equilibrar esse processo de

Equilíbrio federativo é muito importante e por um lado nós precisamos

compreensão mas claro também de forma legítima encontrar os caminhos para poder

tornar eh esse objetivo e essa meta alcançável por todos os estados e o acho que

o Gregório o ministro já sinalizou isso aqui o Gregório vai poder comentar um

pouco mais essa preocupação do próprio Ministério em atender nós estamos levando

isso eh com muito cuidado e com muita atenção todos os subsídios para de fato

ser algo que mobilize o país e falando em mobilização do do país eu eh entendo

sempre falo sobre isso não adianta fazermos a as mesmas coisas esperando

resultados diferentes Então o que eu vejo de muito diferente nesse processo de

renegociação vamos dizer assim eh desses passivos com esses estados é é criar o

que nunca foi feito um objetivo Nobre a ser alcançado pelo conjunto da sociedade

e ele só vai funcionar se a sociedade se houver o controle social se houver o

engajamento da sociedade eh de todas as instituições não em relação a esse

objetivo então não adianta só Ministério da Fazenda só o Ministério da Educação

juntos tentar cobrar e fazer com que isso funcione com os estados eh tem um

papel nosso a cumprir e ele será feito Mas de fato precisa haver esse

engajamento e essa cobrança porque aí sim eh todos os entes vão ter que vão ser

obrigados vão ser eh induzidos a atingir esse objetivo então considero essa

questão do controle social um relevants Sima e de engajamento eh para poder

efetivamente ter o esforço de todos centes para cumprir esse objetivo acho que

eh falar um pouquinho sobre a governança acho que um aspecto importante Gregório

também pincelou algumas palavras sobre isso nós estamos tomando muito cuidado

para que isso seja eh não seja uma ação que nós só colocamos na eh eh em prol

dos estados e vamos fazer um acompanhamento apenas do ponto de vista fiscal se

cumpriu ou não se gastou ou não mas de fato fazer disso um engajamento e um

programa para que isso dê certo e para que isso dê certo a governança importa

importa muito então ontem mesmo estava estava reunido com com o bid né o banco

interamericano de desenvolvimento que se colocou completamente à disposição da

da Fazenda de educação para não só reunir fundos mas eh dar suporte técnico com

por meio de especialistas enfim Já conseguiram formatar aí uma rede de suporte

internacional de especialistas que poderiam apoiar o programa e apoiar os

estados Tecnicamente para poder atingir seus objetivos da melhor forma de forma

articulada com o setor pro produtivo com o setor Empresarial eh com a sociedade

civil então criar um de fato uma governança profissional uma governança de fato

que que tem um ciclo mais longo e há um ganho importante de trazer um

multilateral para para uma discussão como essa que ele evita que isso acabe

sendo uma relação muitas vezes política né que se eventualmente divergências

políticas acabem afetando um objetivo que não deveria que ele deveria ser uma um

olhar de estado e não e não e não de governo Então eu acho que traz um gan de

governança e eh pro programa de forma muito expressiva eh nós pactuamos que

vamos fazer aí uma agenda de trabalho junto com o ministrio da educação para ter

fazer o o desenho desse processo e acho que aí tem um ganho eh importante eh

para todos eu não vou me alogar para deixar aqui ter teru chance de responder as

perguntas há vários detalhes técnicos que nós acho que podemos Esclarecer aqui

nós temos ouvido muito o feedback dos secretários de fazenda do Secret de eh de

educação acho que todas cada realidade né o Brasil é um país muito heterogêneo

então há realidades muito distintas o Gregório falou um pouco sobre isso né

alguns têm condições de avançar no ensino integral outros não outros querem

outros não querem então estamos nesse trabalho de ouvir E aí ouvir construir e

encontrar um equilíbrio claro que sempre há uma ansiedade sempre aqueles que já

já acham que isso é uma uma proposta pronta e acabada e não há um espaço para

diálogo mas não é assim que nós temos conduzido não foi assim que nós conduzimos

o ano passado e não vai ser assim como que nós vamos conduzir dessa vez vamos

fazer isso com diálogo para que tenhamos sim um projeto o melhor possível n que

equilibre os diferentes anseios as diferentes necessidades aí ao longo de todo

de todos os estados e atender o nosso Brasil e nossos jovens que é o que importa

então para encerrar a fala Inicial só parabenizar pelo evento e tô aqui

completamente à disposição para contribuir Obrigado sec começar bom bom dia a

todos eh bom dia obrigado secretário Gris Obrigado secretário cão Fernando eh eu

gostaria de insistir um pouco mais em um um tema que foi falado um pouco no

painel anterior um pouco introduzido pelo secretário guisa grisa que ainda é a

questão eh eh dos mecanismos aí de acompanhamento né E no caso específico do é

uma pergunta pros dois no caso específico do MEC eh uma dúvida se essa questão

da Integração com o ensino médio vai ser um critério eh ou se não se pode ser

podem ser eh pode ser investido em em em cursos técnicos não integrados que

também vai vai contar para essa meta e no caso aqui do do do secretário Ceron

como é que fica a possibilidade dos Estados por exemplo trocar dinheiros né ou

seja tirar um dinheiro de um de um outra rubrica para investir nisso e sem que

seja eh um investimento novo como é que que isso vai entrar eh eh no

monitoramento aí do governo obrigado bom sobre o o monitoramento da natureza da

da matrícula assim a gente tem duas grandes naturezas de cursos técnicos no

Brasil o curso técnico eh ambos de nível médio a gente chama nível médio né o

curso técnico integrado ao ensino médio com currículo único e o curso técnico

concomitante ao ensino médio onde o aluno tem uma matrícula por exemplo na rede

estadual e faz um curso técnico paralelamente num outro Turno eh por meio de

parceria ou evento parceria com Instituto Federal parceria com sistema s ou numa

outra escola técnica eventualmente do próprio Estado né esses dois tipos de

matrícula eh na nossa proposta contarão na expansão o outro tipo de curso

técnico é o curso subsequente onde o Estudante tem que ter concluído o ensino

médio para entrar no curso subsequente eh e a princípio como como eu antecipei e

ao olhar os dados e as evidências sobre esses cursos no Bras Brasil primeiro

Eles já constituem a imensa maioria das matrículas de cursos técnicos chegando a

quase 1 milhão de matrículas no Brasil o subsequente e ele tem sido eh utilizado

por pessoas que ingressam entre 27 a 28 anos não mais um perfil de pessoas que

estão no ensino médio que do ponto de vista eh da proposta é o foco o ensino

médio é o foco né e eh outra distinção que pode eh suscitar dúvidas é a seguinte

a gente tem eh no catálogo Nacional de cursos técnicos outras tipologias que são

os cursos de qualificação profissional ou cursos Fix a princípio que são cursos

mais curtos eh bem menores do que 800 horas ou 1000 horas ou 1200 horas que são

as três eh formatações de cursos técnicos a princípio esses cursos eh eles não

estariam computando a menos que E aí é uma construção que alguns estados TM

trazido pro debate que a gente a que a gente tenha uma soma né de um conjunto de

cursos de qualificação chegando a uma um patamar de carga horário de um curso

técnico inteiro diplom o menino eh como curso técnico aí essa matrícula contaria

E aí a a ferramenta pra gente eh monitorar ela como eu disse antes ela vai

passar pela ideia de de uma base de dados administ rativos que é possível ser

feito porque no pé de me nós estamos construindo isso com os 27 os 26 do

Distrito Federal né os entes e também pelo Censo né a captura do censo eh no

caso do tempo integral quando a gente fomenta 50% para as redes de Ensino em

relação ao que elas propuseram de expansão a gente vai conferir no Censo no ano

seguinte né ou no dado administrativo que foi enviado para apagar a outra

parcela então o monitoramento vai vai se D nesses termos não sei se te respondi

obrigado secretário secretário bom quanto ao ao ao uso dos recursos e o famoso

troca de fontes né o risco de eventualmente eh se utilizar o recurso que já

seria utilizado para para alguma outra função da educação eh nossa ideia tá há

um diálogo há um há um há pleitos dos secretários eh de algum de alguns Estados

para ter alguma flexibilização mas nós temos insistido dialogar tem temos uma

posição aqui fechada e educação e Fazenda no sentido de que esses recursos

comoes são recursos adicionais eles vão ser aplicados na educação na expansão do

ensino eh do ensino profissionalizante mas ele não não entraria pro cmputo dos

percentuais de aplicação na educação então portanto você realmente adicionaria

recursos ao cumprimento desse objetivo sem precarizar outras políticas

relacionadas à à educação a há um debate em torno disso tá por são são

ponderações ponderações sempre são válidas tem que ser discutidas de alguns

estados que fal posso fazer isso de uma forma um pouco mais eficiente ao menos

uma parte eu poderia eventualmente utilizar para poder ter algum ganho e poder

utilizar esse ter um estímulo para ser mais eficiente e poder direcionar

recursos para algum outro objetivo para alguma outra política pública que seja

válida Por enquanto não é não é essa o nosso o nosso caminho tá a gente tem

insistido da necessidade de que de f ele seja completamente um recurso adicional

há casos de estados que tem o volume de recursos como foi comentado aqui que vai

ser tão brutal que nós por mais que nós tenhamos lá Outras aplicações por

exemplo caso ele consiga comprovar que ele vá atingir aqueles resultados as

metas pactuadas pode uma parte dos recursos ser direcionadas para expansão de

infraestrutura para ensino integral para centros de pesquisa e desenvolvimento

enfim questões que perpassam a questão do do Ensino Médio Mas é um pleito para

que se Estenda para essa possibilidade para alguns casos para outras áreas para

infraestrutura de saneamento etc então aí se caso uma proposta dessa avance para

que alguma parte desse recurso precise eh possa ser utilizado para algum outro

objetivo Aí teria que se pensar em algum também algum algum alguma outra regra

trava para que Garanta que isso seja um adicional de investimento Tá mas

Voltando à à proposta original nós estamos mantendo ele completamente esse

excepcional ele não entra pro cmputo dos percentuais da educa mínimos da

educação e portanto na nossa nosso entender a gente nós garantimos assim que ele

não vai ter uma uma uma utilização uma troca de fonte uma precarização da

política de educação secretário serão aproveitando eh importância do

planejamento né dos governos estaduais depois para que eh haja uma manutenção de

de recursos a médio longo prazo enfim para que essas políticas não se dissolvam

como é que isso vai ser estruturado no programa e do ponto de vista do governo

federal se e essa negociação enfim se essa e esse jogo pode de alguma forma eh

refletir nos mínimos que o governo federal tem que investir e e e e e direcionar

pra educação ou são recursos totalmente eh distintos e não entrariam em contas

como mínimo constitucional por exemplo olha Eh pensando a médio e longo prazo né

Acho que tem uma não sei se é claro então deixa eu eh tentar compartilhar o

macro arranjo que eu acho que isso é importante eh a escolha do do ensino eh do

ensino técnico né articulado aí ao ao ao ensino médio ele tem algumas vantagens

Claro do ponto de vista econômico como nós já mencionamos tem todo um ganho de

produtividade a médio e longo prazo enfim é é muito importante mas ele também

ele se casa perfeitamente com com a própria estrutura do com débito nós estamos

incentivando que a abertura dessas vagas ocorra por meio de parcerias essas

parcerias no primeiro momento no no no ano de zero né Há o investimento por

parte do ente para abertura daquelas vagas Então tem de fato e eles vão ser

custeados com a economia desse de juros da da dívida Mas a partir do segundo ano

ele gera um duplo cômputo do fundeb para essa matrícula que manté esse custeio

sob controle Então isso é uma questão muito importante porque nós nos

preocupamos muito com a sustentabilidade do programa também não adianta eh nós

investirmos ampliarmos as vagas até 2030 chegando lá não há condições de

suportar e felizmente também uma uma felicid acho que uma uma feliz coincidência

do destino Nós aproveitamos esse arranjo também para que permitir isso nós

estamos num processo de expansão dos recursos do fundeb dos aportes do governo

federal no fundeb então também ele permite que essa expansão esse duplo cmputo

seja suportado sem ser sem gerar um prejuízo aos demais então do ponto ponto de

vista financeiro e olhando a sustentabilidade eh do modelo ele ao nosso ver ele

tá bem desenhado ele tem condições de de prosperar Então você não vai criar o

que é uma nossa preocupação como como Ministério da Fazenda e responsável de

alguma maneira por por zelar por pela saúde eh fiscal dos entes né não não não

fomentar uma fragilização dos entes subnacionais eh tem uma preocupação em não

fragilizar isso a médio longo prazo isso não vai ocorrer então nós não vamos

impulsionar despesas correntes que depois não vão poder ser suportadas isso é é

muito relevante que se diga então ele foi pensado justamente para permitir um o

atingimento do Objetivo Nobre ao mesmo tempo que essa redução de juros ela afeta

todo o estoque uma parte ela é uma economia de fluxo uma parte ela gera uma

economia no na na dinâmica de crescimento desse estoque da dívida então ao longo

do tempo a dívida vai caindo como proporção das suas receitas e permitindo

cumprir esse objetivo que é tão nobre então o arranjo financeiro ele foi feito

para garantir uma sustentabilidade a médio e longo prazo sim eh o secretário

grisa eh eu tenho mais uma para você o sen o senhor mencionou aqui nessa questão

essa importância do do alinhamento com o mundo do trabalho né Eh e também falou

um pouquinho na sua fala Inicial sobre a questão das Diferenças regionais e eu

queria que o senhor falasse um pouquinho sobre como que também é encarado são

encaradas essas diferenças regionais na questão do ensino técnico né Por exemplo

a diferença de um ABC Paulista paraa vocação rural de alguma região como é que

isso pode ser trabalhado para que as diferenças regionais sejam contempladas aí

com essa expansão de vagas por favor eh durante as conversas com as redes

Inclusive a gente foi sentindo a necessidade e aí você tem redes assessoradas

por diferentes atores que conseguem ter um grau vamos dizer mais elaborado de eh

estudo de demanda de alinhamento com o mundo do trabalho outras menos e E aí a

ideia de criar ferramentas nacionais ou que possam dar esse assessoramento é

justamente para atender a a essa diversidade e as peculiaridades das redes né

hoje a gente tem no Brasil um leque vamos dizer um portfólio de cursos técnicos

muito mais vasto do que no passado e a ideia eh de você contemplar a as regiões

ela passa muito pela as vocações regionais os arranjos produtivos né Eh que são

que hoje inclusive gente mudam numa velocidade diferente do passado né você tem

regiões que tê vocações hoje ligadas ao meio ambiente que não estavam

identificadas há 10 15 anos atrás né então você ter ferramentas que consigam

identificar os arranjos locais para para a abrir cursos coerentes eh parece que

é uma questão de de de política de estado mesmo né o governo federal tem que

construir essas ferramentas não é obrigado que a rede tenha que usá-las porque a

rede tem as suas Mas elas as nossas devem servir de complemento para esse estudo

eh de em alinhamento e outra questão que tá ligado a isso eh eu posso querer

fazer um curso muito interessante muito inovador eh com ótimas intenções mas se

não há eh Professor o curso não vai sair então daí a preocupação de eh que o

leque de ferramentas de alinhamento também Contemple a ideia de formação de

profissionais para para dar esses cursos né Eh quando a gente fala da

experiência bem-sucedida dos institutos federais a gente tá falando entre outras

coisas disso não é só carreira não é só infraestrutura é a ideia de para além eh

da cabeça das pessoas que fazem o Instituto Federal quando se concebe novos

cursos você tem um grau orgânico de consulta à comunidade um grau orgânico de

estudo de demanda que faz com que a atratividade que já por natureza já é muito

boa seja ainda maior dos institutos Federais e aí você tem um eh um um um perfil

de de trabalho docente por exemplo de pesquisa que que a gente ainda não

encontra em escala nos estados há experiências positivas Mas a gente não não

encontra com essa expansão do ensino técnico eh e aí é combinada com a expansão

dos próprios institutos já anunciada pelo presidente pelo Ministro aqui eh você

tem potencial de parceria aí que talvez você não tivesse antes Se vocês olharem

as cidades as 100 cidades anunciadas de institutos nós estamos chegando em

cidades de 45 50.000 habitantes 35.000 habitantes 60.000 habitantes que

provavelmente tem uma rede Estadual de Educação que não é tão gigante que

potencializa a parceria com os institutos que antes estavam mais ou ainda estão

concentrados em cidades médias grandes né de 150 200 50 300.000 habitantes então

eh acho que o esse processo de alinhamento dialoga com a formação né e e e com a

dinâmica eh menos burocrática e acho que isso é um elemento importante das redes

eh terem no horizonte a gente não pode mais pensar cursos técnicos no ensino

médio para durarem 30 anos entende eles não vão durar 30 anos é muito mais

dinâmico a mutação econômica dos arranjos locais nós temos que pensar cursos que

atendam ali a a a a duas turmas três turmas que vão se formar e precisam eh sem

nenhum digamos assim receio passar por uma avaliação de transição de transição e

troca de curso se for se for o caso né a gente tem eh cases no Brasil você

mencionou o abc enfim que tem potencial Industrial mas a gente tem cases no

Brasil em que há necessidade constante de troca do perfil ou do eixo tecnológico

como como mostrou o governador a troca do eixo tecnológico daquelas áreas né

então Eh por vezes você tem uma gestão de 4 anos que pode eh dentro dela mesmo

fazer essas essas transformações tem que ter mais flexibilidade nesse olhar

porque a dinâmica é diferente mas a a inovação do programa é termos ferramentas

federais para assessorar os estados nesse alinhamento com mundo do trabalho a

gente tem um Um Desafio aqui de tempo de tentar amarrar com considerações finais

e e de alguma forma dar uma luz paraa frente sobre o que a gente pode esperar es

né enfim secretário Gregório se puder fazer uma ração final faça uma provocação

também em relação senhor falou muito questão formação de professores como enfim

o que que tá de fato no pipeline né no radar ou seja em relação a isso O que

pode se esperar quantos em que tempo Qual a distribuição pelo país enfim que que

isso pode pode que podemos esperar Eu mencionei a formação de professores como

um potencial gatilho para aqueles casos em que os estados eh já atingiram a meta

e é uma informação pública né serão que quatro Estados concentram cerca de 90 né

cerca de 90% da dívida nós estamos falando de quatro Estados muito populosos

muito endividados e que qualquer inserção de escala de política pública nesses

estados tem um impacto Nacional muito grande São Paulo Rio de Janeiro Minas

Gerais Rio Grande do Sul eh então assim falar como gatilho é a ideia de que se

se prosperar essa ideia incluímos ou no projeto de lei ou na ou na na Norma

infralegal para fomentar desde já o planejamento desses estados para essa oferta

de formação no caso da ept nós não estamos falando necessariamente de uma

licenciatura inteira de 4 anos mas de complementação pedagógica para

profissionais que já são da área técnica mas para titulável pedagogicamente para

dar aula na educação eh profissional a expansão dos institutos federais dialoga

com isso a lei de criação dos institutos prevê que Obrigatoriamente 20% de todas

as matrículas dos institutos tenham que ser em formação de professores Então

essa é outra iniciativa bastante concreta que já endereça nesse sentido e por

meio das universidades federais que é o que tá mais ligado à gente né Eh a gente

tem desenhado e E aí tem um debate de orçamento por óbvio que eh a a situação

das universidades federais mas tem desenhado um edital de fomento de ampliação

de formação dos professores com com com vocacionadas paraa educação profissional

também nas federais isso pode ser e já é objeto de fomento tanto do pro Uni

quanto do fi também o Fi agora nessa nova edição do fi social já priorizou

matrículas de licenciatura que é um era um cenário muito ruim para noss da

educação não sei se vocês sabem né o fi não alcançava nem 2% das suas matrículas

em licenciatura Isso é muito ruim isso é muito ruim eh encaminho para as minhas

considerações pegando o exemplo do Pr un fazer um breve comentário em relação eh

a essa crítica eventual que eu tenho ouvido de que [Música] eh o programa seria

uma ultra premiação para os os estados individados que não fizeram o seu dever

de casa eh eu lembro na época do debate do ProUni em que isso era muito forte e

o ProUni para quem não conhece é isenção fiscal para instituições em troca de

bolsas eh de estudos e Os relatos de quem tava no mec na época era muito singelo

o raciocínio é muito simples eh Nós já tínhamos uma um não pagamento de impostos

muito volumoso já já não era não era pago aquele volume de recurso e a ideia

dentro da lógica da criatividade que o ministro aqui mencionou é justamente já

que não é pago o recurso pel aquelas instituições trocar por vagas em no ensino

superior para alunos de escola pública né que é o esforço de expansão o proun

hoje é muito bem avaliado é um é um programa regular que tem uma maturidade bem

importante eh chegando na casa de quase 400.000 matrículas anuais e E você tem

não não não não quebrou o ecossistema privado de educação superior pelo

contrário fomentou eh a vamos a recuperação e a maturação das instituições né

então eh eu eu faço um paralelo eu faço um paralelo em relação a às dívidas

muito grandes como bem comentou o o secretário daril periodicamente a lógica do

litígio é uma lógica que não tem dado certo a lógica do conflito por quê Porque

os estados E aí é super legítimo né Eles encontram alternativas políticas dentro

do congresso e dentro do do do supremo para de alguma forma ou se isentar ou

suspender ou ter eh uma alternativa de não pagamento e se a lógica tem sido essa

olhando pregresso pros últimos 10 15 20 anos se se trata de um recurso que a

grosso modo simplificando aqui não virar um recurso que não virá faz todo o

sentido e aí o o serão usa a palavra corretíssima né você endereçar para uma

pauta nobre que inclusive no médio e longo prazo vai potencializar a saúde

financeira daquele Estado então é essa a lógica A A A A A lógica é substituir a

a postura punitiva punitiva do ponto de vista vamos dizer assim até do campo

moral simbólico por uma lógica produtiva e de otimização por uma lógica

pragmática de um recurso que se virar virar os pingos e eh e pro próprio tesouro

até sem potencial de planejamento de Médio prazo de receita quase nenhum porque

suspender eh eu lembro eu sou de um estado até encerro dizendo isso né Eh que é

o Rio Grande do Sul eh que conseguiu a suspensão da dívida por um período e

minha mãe e é professora minha irmã é professora meu pai é professor só que a

minha mãe e minha irmã são do Estado ficaram 5 anos 5 anos sem receber eh o

pagamento do Servidor Estadual em dia são cinco anos em governos eh anteriores e

com a dívida suspensa com a dívida então assim o servidores não recebendo graves

problemas fiscais no estado e o tesouro não recebendo e se aquele recurso

tivesse sido endereçado eh para o para para um fim Nobre a gente poderia ter

alavancado eh no caso do Rio Grande do Sul eu eu agradeço novamente ao convite

eh deixo aqui o registro também público da minha solidariedade a população do

Rio Grande do Sul minha minhas meus filhos estão lá meus pais estão lá situação

de Porto Alegre é muito grave nesse momento né Deve a princípio as notícias são

de que eh as chuvas vão se intensificar bastante agora de terça sexta-feira vai

voltar a chover bastante e eh para quem conhece o Rio Grande do Sul ali o Guaíba

ele ele desemboca na Lagoa dos Patos que é gigante então a gente tem que olhar

com muito muito Alerta paraa região sul Pelotas Rio Grande que vai receber um

volume de água muito grande agora nos próximos dias e dizer que eu faço parte do

do comitê dentro do governo federal por eh representando do MEC eh do do grupo

emergencial criado né da sala de situação a gente tá tentando organizar do ponto

de vista normativo inclusive financeiro serão eh as as as ações emergenciais

para essa semana já ligadas ao ao pdde né programa de direto na escola dinheiro

direto na escola e outras ações eh e agradecer também toda a solidariedade que a

gente tem a gente Gaúcho tem recebido eh e dizer que o MEC tá de portas abertas

gente por Óbvio o diálogo travado com os secretários estaduais ele é o que tem

que ser mais orgânico nesse momento mas para as entidades paraos jornalistas

para todos que T interesse em contribuir com essa pauta a gente tá em construção

desse projeto eu acho que como bem disse o seron se a gente conseguir criar no

Imaginário social mesmo que seja Imaginário restrito eh daqueles atores que têm

influência política eh e convencê-los da potencialidad de reter na educação o

destino desses recursos a gente tem uma janela de oportunidade histórica para

fazer em 5 6 anos Ministro disse que é audacioso né Mas por que que é audacioso

porque não foi feito em 130 fazia em 6 anos o que não foi feito em 100 né quer

passar esse patamar de 15 16% para um outro patamar de oferta técnica eh eh no

Brasil há um um estudo do do Paz de barco que ficou conhecido em que ele diz que

eh se a gente reduzisse a evasão do ensino médio que hoje até 24 anos só 60% se

forma no ensino médio né Nós deixaríamos de gastar 135 bilhões ano ano não não é

num record de 6 anos ano né e o pé de meia por exemplo que envolve 7 8 bilhões

faz o paralelo né para tentar reduzir PED meia quer reduzir invasão um dos

programas que quer reduzir esse aqui é outro gente esse aqui é outro se os

meninos querem educação técnica se a gente conseguir ofertar vamos dizer assim

uma educação técnica é um diálogo com o sistema S também muito importante tá a

gente tá à disposição Tenho recebido vários atores do sistema s para que a gente

troque vamos dizer ideias para que nas parcerias seja ofertado para esse menino

de periferias que tá na rede estadual o filé do sistema s o curso grande o curso

de 1200 o curso de 1000 horas aquele curso que realmente vai alavancar a

capacidade pessoal dele eh e a capacidade do seu entorno da sua família né então

agradeço muito e estamos à disposição nós que agradecemos sim eu só eu ia deixar

passar a palavra pro seron e aproveitar não perder a chance de também fazer uma

provocação e pedir para encaminhar paraas considerações finais com com olhar

paraa frente mas eh inspirado no que disse o secretário Gris é sobre duas coisas

né sobre o dinheiro que não virá né E sobre pautas nobres né Eh quando esse

programa foi anunciado eu eu vi como uma uma saída criativa para um problema

inexorável né talvez de um dinheiro que não viria e eu queria que o senhor

falasse um pouco sobre a possibilidade de que outras pautas nobres apareçam eh

para Para que sejam usadas em relação a essa dívida né de futuramente os estados

usarem argumentos para investimentos por exemplo em infraestrutura ou em

preservação ambiental que possa ser usado E aí qual que seria uma posição do

Tesouro também olhando paraa frente e já agradecendo com um detalhe né

secretário que eh desses quatro Estados eh eles que concentram né o grande

volume de dívida são administrados por eh eh políticos ligados à à oposição e

que muitos têm pretensão eleitoral que Diverge com da da do do do projeto do do

Governo Federal em 2026 então assim também quando se diz que vai oficiar quem

não quem não eh fez o a lição de casa isso de alguma forma é um paradoxo porque

são governadores da oposição também né Eh bom as provocações são são são boas eh

mas vamos lá dentro do que eu posso tenho liberdade e condições de responder eh

primeiro eu acho que tem um esse macrocontexto Gregório comentou sobre isso

daril comentou sobre isso desse processo histórico brasileiro quem conhece a

história fiscal brasileira sabe desse esse processo cíclico que vem lá da década

de 90 dessas da renegociação de dívida desses passivos e dessa dificuldade do

dos entes subnacionais em especial os estados terem um caminho de saída alguns a

maior parte deles conseguiu ao longo do tempo se você olhar eh ou os efeitos da

lei de responsabilidade fiscal é notório que o endividamento agregado dos

estados e municípios vem caindo ao longo do tempo então foi foi muito bem

sucedido mas nós temos problemas principalmente no eh Rela o estado de Minas

Gerais Rio Grande do Sul e Rio de Janeiro tem problemas eh muito muito

relevantes do ponto de vista fiscal que ainda precisam ser endereçados eh Claro

vou ter a delicadeza de não entrar no mérito do do da parte que que compete as

próprias decisões de cada um desses estados e aquilo que foi questões mais

estruturais que levaram a esse resultado Mas de fato são três estados mais

fragilizados e tem São Paulo que tem a maior dívida Embora tenha ao longo das

últimas décadas tem feito um trabalho fiscal na minha opinião consistente tem

tem conseguido apesar da dívida elevada Honrar seus pagamentos em dia esses

estados vão ser muito beneficiados então quando a gente olha pro o volume dos

recursos numa discussão como essa de redução de juros é claro que ele vai gerar

um suporte financeiro relevante né Eh então eu acho que para nós o que nós

estamos abrindo de eh de juros vai somar algo em torno de R8 bilhões deais por

ano claro todo recurso importa é óbvio Mas ele tem um impacto de minuto e

pensando que é maior parte desse desse processo ao longo das últimas décadas

fica em Idas e Vindas no judiciário Então não é um recurso que tão simples de

contar com ele embora nós fizemos com com a com a com a serenidade com a

inteligência de olhar isso para frente nos juros para que não tenha um impacto

fiscal eh relevante se nós tivéssemos que discutir estoque coisas como eh como o

tipo Então acho que eh tem tem esse caminho aqui de e é de saída que precisa que

precisa ser enfrentado não tem não tem não tem como eu acho que é legítimo a

gente tentar pensar uma forma de sair e tentar dar condições para que esses

entes dado que estão nessa situação ten um caminho de saída não dá para

simplesmente é optar não é pelo menos a nossa postura optar por por deixar a

própria sorte ou num processo de conflito judicial que eu acho que ninguém ganha

como o Gregório comentou o Rio Grande do Sul passou no passado anos e anos em

discussão judicial suspensão da dívida que não era bom paraa união e com certeza

também fragilizou ainda mais as Finanças do Estado gerando Impacto para

população então fazendo gancho com com a sua colocação sobre sobre a a a o serem

de oposição ou não eu acho que nós como política econômica a gente tem que olhar

paraa população olhar a política de estado a gente isso acho que é inegável

todas as ações que nós estamos conduzindo toda vez que a gente si eu sempre falo

isso assim tenho sempre vou ter eh tranquilidade de vir à luz do sol debater e

discutir e mostrar que aquilo é bom para o estado brasileiro e não para um

determinado eh governo ou para um determinado eh partido político isso é muito

importante a gente tá olhando isso isso nada mais emblemático eh do que esse

ponto nós estamos reconhecendo que há um uma situação muito muito grave e

fragilizada nas Finanças de alguns estados que são estados popular importantes

para pra Federação e que precisam de um olhar ali de um de um cuidado especial

mas também nós não queremos que fazer isso de qualquer jeito para que nós

tenhamos de novo os mesmos resultados do passado Então eu acho que toda

construção e trazer o controle social eu falo muito sobre isso porque para mim

essa é a grande diferente diferença em relação ao passado nós estamos tentando

trazer sociedade para para dentro do debate para se engajar com isso porque se o

estado estou fazendo a conão desses recursos 100% dos recursos para poder

cumprir o objetivo de dar profissão paraos seus jovens não vai ser cumprido

então assim como Qual o grau da de compromisso que a sociedade vai ter em cobrar

esses governos eh para cumprir esse objetivo e cumprindo esses objetivos com a

solução Como ela foi montada do ponto de vista de sustentabilidade fiscal de

médio e longo prazo por um lado o estado vai seguir seu processo de recuperação

fiscal e por outro nós vamos ter um outro patamar de formação de mão de obra do

país então isso para mim Eh cria ali um cenário único em relação ao nosso

histórico passado claro que a desafios claro que há riscos de de muitos eh

tentarem novamente Não cumprirem suas obrigações Mas cabe a nós como como

sociedade de fato tem o engajamento dos atores para fazer esse processo ser

diferente dessa vez eu realmente acredito nisso acredito que ele é diferente e

que ele tem tudo para dar certo se houver o engajamento como tá acontecendo num

espaço como como esse então eu agradeço a todos aqui e vou continuar à

disposição e fazendo um embate público em defesa de uma saída adequada paraa

situação fiscal dos Estados mas cumprindo um um objetivo nobre que é a formação

dos nossos jovens obrigado muito obrigado secretários muito obrigada senhores

pelo debate pelas contribuições foi um prazer tê-los aqui Muito obrigada então

vamos aqui ao nosso próximo painel antes um momento paraa foto por gentileza

muito obrigada senhores mais uma vez e antes de dar início ao nosso terceiro

painel gost gostaria de lembrar que também estamos ao vivo pelo YouTube LinkedIn

e Facebook do valor econômico na transmissão do nosso evento vamos ao terceiro

painel do dia desta vez o tema será a proposta sobre a ótica da educação

profissional nos Estados na mediação continuamos com os jornalistas Fernando

exman e Murilo camaroto respondendo às perguntas teremos os convidados aqui

presentes Ron Miranda secretário de de Educação do Estado do Paraná e também

representante do concede por gentileza secretário Obrigada e Fátima Gavioli

Secretária de Educação do Estado de Goiás por gentileza secretária obrigada E

falando com a gente à distância participará deste painel também Guilherme lichan

que é professor da Universidade de Stanford seja bem-vindo Professor muito

obrigada a todos presença já vamos dar início então bom debate Olá bom dia pros

senhores que eh chegam para nos acompanhar agora Murilo teremos O Grande Desafio

de tentar eh nos conter em relação ao tempo né temos aqui um enfim ainda mais um

um painel e esse certamente eh vai agregar muito em relação ao que a gente já já

teve então assim pra gente ter as as considerações iniciais se a gente puder

enxugar quanto mais mais curto e sucinto a gente pode aprofundar depois no nosso

debate então secretária se a senhora puder começar eh enfim do ponto de vista do

estado da senhora Quais são os pontos fortes os pontos fracos O que que a

senhora tem trazido dessa interlocução com o governo federal que a senhora pode

nos contar e que que pode iluminar a nossa nossa conversa bom bom dia a todos

Obrigada Priscila Obrigada Olavo pelo convite eh eu quero cumprimentar todos que

nos assistem mas antes de qualquer coisa eu quero fazer menção à professora

Raquel na verdade ela é uma das convidadas deveria estar aqui agora e todos nós

sabemos a luta que ela tá enfrentando lá no Rio Grande do Sul então aproveito

também esse momento para mandar aí paraa Raquel energia positiva pensamento

positivo além de tudo aquilo que nós estamos fazendo no Estado de Goiás para

poder ajudar eh com a nossa força né Eh militar eh todos nossos bombeiros estão

hoje lá mas assim a Raquel realmente fez falta aqui hoje para todos nós tá eu

quero falar sobre essa questão dos Desafios alguém que me antecedeu usou uma

frase que no serviço público a gente tem uma prioridade em cima de outra

prioridade e outra prioridade outra prioridade o secretário de educação embora

ele administre a maior fatia de contribuição prevista na Constituição nos

repasses ele não é também a pessoa que faz tudo aquilo que ele poderia fazer

porque ele tem sim uma dependência e isso é muito eh importante que exista né

tanto dos órgãos de controle como também dentro do próprio governo nas

secretarias parceiras secretarias de economia I administração e etc então nessa

nestes últimos 6 anos e aí eu tô falando de últimos 6 anos e aproveito para

abrir um parêntese porque bem agora foi dito aqui que poxa vida mas essa

proposta talvez beneficia inclusive quem está hoje eh dentro do regime de

recuperação fiscal mas que também tem seus sonhos de disputar inclusive uma

eleição daqui a pouco eu ouvi a pergunta e aí o que que eu queria dizer a vocês

a maioria de nós que Hoje fazemos parte do Regime de recuperação fiscal na

verdade não criamos essa situação na verdade herdamos e estamos lutando para

tentar resolver isso e quando eu vejo o ministério da economia solidário tando

tentando trazer uma proposta Educacional de educação profissional paraas

secretarias de educação Isso enche o nosso coração de esperança independente do

posicionamento das sugestões eh das que serão feitas através das secretarias de

economia e dos nossos governadores isso para nós é importante alguém olhando

pela educação alguém querendo discutir Olha nós aqui nesse auditório Olha nós lá

em casa Olha nós agora no trabalho o povo que nos assiste estamos discutindo

educação brasileira de educação profissional então Sem dúvida nenhuma foi um

passo muito importante que foi dado agora o maior desafio é 2019 dívidas

tentando entrar no regime de recuperação fiscal e agora eu tô falando aqui

também pelos meus colegas que com certeza nos assistem agora através da

transmissão os que estão no regime 2019 só dívidas só dívidas pouco a se a se

organizar e planejar a gente não sabia sequer se conseguiria pagar uma folha de

pagamento em dia como é que você quer ter professor para trabalhar educação

profissional se malmente consegue se manter de pé na na na própria

sustentabilidade eh diária da secretaria na sustentação da secretaria depois de

2019 veio o que mesmo em 20 21 22 a efetivamente a pandemia acabou o governo

acabou o primeiro mandato quem teve direito a um segundo mandato Agora sim agora

sim agora eu diria que respirando olhando mesmo pro futuro como alguém que tá

preparando seu estado e uma nação com mais pessoas formadas com mais pessoas

capazes de inserir no mercado de trabalho e com certeza trazendo sim um recurso

tanto tanto recurso humano como um recurso financeiro para o seu estado Então

essas são as minhas considerações iniciais Que bom que estamos discutindo

educação profissional que bom que que falamos aqui hoje sobre licenciaturas Que

bom que falamos aqui sobre educação integral vocês pensam que tá sendo fácil

manter o aluno na escola de tempo integral não meus amigos a maior dificuldade

que nós temos hoje é manter o aluno na escola o dia todo sabe depois da pandemia

os pais preferencialmente querem que eles arrumem um trabalho de 4 a 6 horas por

dia então tudo isso quando discutido evidentemente que nós vamos ter sim

condição de encontrar soluções a médio e longo prazo para esses problemas Muito

obrigado secretário o senhor bom dia bom dia a todas e todos cumprimentar aqui

especialmente Agradecer o convite da priscillia do Olavo eh ao valor também por

esse momento é fundamental discutir educação educação técnica eu vou usar esse

termo mais técnico porque muitas vezes gente subjuga a educação profissional com

qualificação profissional então e a educação técnica a formação técnica eh do

estudante é louvável Eh quero agradecer também a a a presença da minha colega

Fátima que sempre discute educação mas também referendar a a secretária Raquel

do Rio Grande do Sul era ela que era para est aqui nesse lugar Rio Grande do Sul

passando por esse momento de crise obviamente que a educação ela vai ser

extremamente afetada a gente viveu este momento na pandemia e no Paraná a gente

tá contribuindo também com toda a força policial mas também com a estrutura da

secretaria de educação nossos caminhões da secretaria de educação estão saindo

levando água alimentos para pro Rio Grande do Sul neste exato momento inclusive

eh e para nós eh falar de educação profissional no conced a gente ainda não

discutiu esse tema com os secretários a secretária Fátima sabe disso a gente não

discutiu a questão do juros na educação profissional eh mas eu vou falar mais da

da da minha opinião pessoal eh como a Fátima também colocou é um tema que é

sempre salutá falar de investimento educação em educação técnica em educação

técnica na Educação Básica é algo que é princípio básico se a gente pegar os

indicadores da educação dos Estudantes que concluem ensino médio e acesso a

ensino superior é a grande maioria dos Estudantes ficam com somente ensino médio

e aí se torna aquele jovem nem estuda nem trabalha a educação técnica ela abre

essa oportunidade mas é importante também a gente ter em conta que a educação

técnica ela tem que ser com Horizonte vinculado à necessidade do mercado de

trabalho o que o arranjo produtivo da Regional local tem de necessidade porque

senão a gente começa a abrir curso técnico sem tá integrado com o mercado de

trabalho enfim se torna mais um elefante branco que a gente cria no Brasil sem e

finalidade objetiva e no Paraná a gente vem trabalhando muito forte nessa nessa

Perspectiva da Educação técnica eh a gente vem ampliando chegamos agora no ano

de 24 a 25% dos Estudantes matriculado no ensino médio na educação profissional

e obviamente que isso não vai ser feito sozinho com a rede estadual Precisamos

sim do apoio tanto do MEC mas também das parcerias com institutos federais com o

sistema S que são parceiros que podem contribuir na ampliação mas também com o

próprio eh com a educação privada também particular que pode ser só nesse

processo de oferta porque o centro da discussão tem que estar sempre o nosso

estudante é o que o nossos meninos e meninas precisam e eu vejo na na educação

profissional uma grande oportunidade de transformação social e de mudança de

vida e da realidade desses meninos e meninas eh em relação a à dívida eh os

juros da dívida obviamente que o meu estado do Paraná não está nesse nesse

acabolso mas também temos dívida e vamos ter benefício e eh obviamente que isso

ajuda é importante também que foi colocado aqui pelo secretário Dail que que a

gente tem uma convergência que esse recurso que venha que ele seja investido em

educação que enfim porque senão e é um recursos que vem pro pro tesouro do

estado e acaba se diluindo dentro desse Tesouro e não se reverte a investimento

da educação é importante ter congregar nesta perspectiva como sou econômico eu

vou terminar antes aqui e eu queria dar as boas-vindas pro Guilherme já com já

com uma pergunta eh voltando uma página aqui para saber a opinião de um

estudioso que é sobre o ponto de vista Educacional Guilherme se você como é que

você vê né do ponto de vista Educacional e pro desenvolvimento do país é claro

né E se faz sentido esse investimento forte essa aposta no no ensino técnico

profissionalizante E como que essa agenda é vista eh tanto nos países

desenvolvidos como no nos chamados países em desenvolvimento né obrigado e

obrigado e boas-vindas obrigado pela pergunta Murilo e obrigado a todos eh o

Itaú ao valor pelo convite Saúdo também a secretária Fátima o secretário Roni eh

e uma saudação também a secretária Raquel que não pode estar aqui conosco eh

essa pergunta é crucial né o programa no final das contas Vem ajudar a resolver

um desafio grande que o Brasil tem como se aproximar mais do que fazem os bons

sistemas educacionais sobretudo quando a gente tá pensando em baixa taxa de

abandono né uma taxa alta de conclusão do ensino médio e também a proficiência

elevada tanto em matemática quanto em linguagem em testes internacionais como

pisa no estudo que a gente fez em Stanford aqui no ano passado a gente comparou

diversos sistemas educacionais fazendo essa pergunta o que que era mais

diferente entre o que o Brasil fazia no ensino médio e que esses outros sistemas

faziam e a gente descobriu que o grande diferencial desses outros sistemas não

era necessariamente nem um número enorme de ofertas de percursos em cada escola

nem uma flexibilidade de itinerários porque os alunos pudessem escolher nessa

etapa do ensino médio A grande diferença na verdade uma penetração bastante

elevada dessa ept de nível médio articulada né Como disse o Gregório eh nesse

país chegava a 40 a 50% dos alunos de nível médio cursavam ept que é um

contraste enorme com o que o Brasil fazia né menos de 10% antes da reforma Por

que que essa agenda é importante a IPT torna a escola mais Atrativa para esse

jovem né a gente ouviu do ministro uma porcentagem enorme dos jovens brasileiros

80% coisa do tipo querem educação profissional e técnica isso também é a verdade

no mundo todo né então eh a escola se torna mais conectada a construção de

habilidades são relevantes pro mercado de trabalho e a IPT não fecha as portas

pro ensino superior na verdade ela contextualiza a formação geral básica né de

fato no nos países ricos é comum que 80% ou mais dos egressos do DPT Continua

Estudando depois do ensino médio um de cada quatro desses no nível universitário

ept Não é só para jogar o aluno no mercado de trabalho e aumentar desigualdades

muito pelo contrário eu vou voltar nisso depois antes da reforma no Brasil menos

de 10% dos alunos do médio cursava PT como eu falei e quem cursava precisava

permanecer na escola em tempo integral que é uma restrição elitista que a

reforma eliminou eh Então hoje né com o desenho que a gente ficou do número de

horas na FGB dá para cursa IPT sem necessariamente ter que ficar lá em tempo

integral que é Um Desafio enorme como disse a secretária Fátima que a gente vai

precisar superar mas ao menos não existe mais esse condicionante então pra

educação faz todo o sentido expandir IPT e pra economia também eh primeiro a

gente precisa aumentar a produtividade do trabalho no Brasil tá estagnada desde

a década de 80 com a transição demográfica se aproximando do seu fim aí na

próxima nos próximos 10 anos a gente vai ter mais dependência do que população

economicamente ativa e aí o aumento da produtividade vai ser chave para que não

só a Previdência siga solvente mas também para que a gente siga crescendo e a

literatura mostra uma conexão Clara entre formação profissional produtividade

empregabilidade e salários então pra economia pra gente ter mais produtividade

PT ser a chave na minha leitura mais do que isso mais do que aumentar a

produtividade a gente precisa redistribuir renda né o país é extremamente igual

a nossa Elite seja do ponto de vista econômico ou racial capturou retornos da

educação muito mais do que os grupos fora da Elite desde a década de 70 eh no

ensino médio para vocês terem ideia a taxa a diferença das taxas de graduação

por exemplo entre brancos e não brancos na população ainda da ordem de 40 pontos

percentuais é uma diferença inacreditável muito maior aliás do que no ensino

superior onde ter diploma Universitário é um privilégio recente mesmo para Elite

e não é só concluir né condicionar o teru diploma entre os egressos os brancos

ganham 50% mais do que pros pares indígenas Então as diferenças São enormes e aí

PT não só deve diminuir a diferença nas taxas de graduação entre elites e não

elites entre brancos e não brancos mas também diminuir diferença de

empregabilidade e rendimento entre os egressos se a gente fizer isso bem feito

claro que para essa agenda avançar como deve nós temos enormes desafios né O que

que a gente pode aprender com que fizeram bem os países de alta renda e também

os emergentes que já tão mais avançados do que o Brasil no ept basicamente você

precisa de um novo pacto Nacional um pacto social diferente porque para ept de

fato entregar mais empregabilidade de renda as escolas precisam realmente

fomentar carreiras que tenham demanda local como diz o secretário r e mais do

que isso as escolas e as firmas precisam entender que a gente precisa

desenvolver essas competências com qualidade só se organizar para isso então

primeiro pra gente mapear a demanda local as escolas precisam entender o sistema

de ensino precisa entender que não são os únicos responsáveis por definir

currículo a gente precisa articular a oferta com o setor produtivo local e pra

gente desenvolver as competências relevantes a gente precisa definir quais são

essas competências como a gente vai medi-las como é que a gente vai alocar

recursos para que de fato escolas e empregadores possam de fato promover essas

competências e sobre essa questão dos recursos eu quero falar na segunda parte

da minha fala mas agradeço aí o convite mais uma vez nessa primeira parte aqui

da minha participação Obrigado Guilherme eh dando continuidade aqui acho que os

dois secretários Já deram uma pincelada de Quais são as dificuldades que você

tem algumas delas né Eh o senhor mencionou eh bom vamos expandir o número de

matrículas Mas para onde né preciso de da da rede Federal preciso dos do do do

sistema s e e e e a secretária falou sobre a dificuldade de manter né os alunos

por exemplo no tempo integral então Eh no painel anteriores falaram um pouco né

dessas primeiras conversas com com os estados eh sobre o programa então eu

queria que vocês compartilhassem um pouco na de cada um Quais são as principais

dificuldades e os obstáculos para para que esse programa possa evoluir obrigado

bom em primeiro lugar agradecer por essa disponibilidade disposição do

Ministério da economia e conversar com os secretários de educação uma prática

incum mas que pela primeira vez nós fomos convidados para para conversar sobre

isso tô falando aqui dos estados que estão no regime tá e segundo agradecer o Zé

Frederico secretário eh de ciên tecnologia com quem hoje eu tenho uma parceria

para desenvolver educação eh profissional técnico profissional vejam vocês nós

estamos dentro do mesmo governo e temos dificuldades hercúleos para poder tocar

hoje um projeto de educação profissional porque ele tem as escolas do Futuro

equipadas eu tenho o aluno mas o aluno para sair da minha escola e chegar na

escola do futuro ele precisa de transporte precisa de alimentação precisa eh de

de pessoas para o Acompanhar até lá então assim não é fácil dentro do próprio

governo também falando um pouco sobre a experiência de Goiás passado esse

primeiro mandato tão atropelado em em razão da pandemia E aí uma coisa que eu

quero relembrar aqui parece até que já tá resolvido né Ricardo Henriques todos

os problemas que a pandemia trouxe para nós em relação à recomposição de

aprendizagem E aí eu tô falando do aluno do ensino médio e vamos focar só no

médio hoje do aluno do ensino médio como um regular de 6 horas gente nós estamos

trabalhando muito para recompor aprendizagem quando você traz a educação

profissional técnico profissional para dentro do ensino médio essa discussão

também nós não podemos perder o foco nessa questão da aprendizagem que tem que

ser feita a recomposição dela porque um aluno que mal lê mal escreve mal mal mal

como é que ele também vai ser um bom profissional dependendo do curso que nós

vamos vamos levar para ele lá dentro da educação profissional então vejam Como

eu disse que prioridades em cima de prioridade prioridade e nós aqui tentando

resolver ou pelo menos eh trazer propostas para essa situação em relação à

educação integral é isso que eu falei a educação eh integral ela foi criada já

aqui no país né há mais de 10 anos na época eu estava secretária em Rondônia

quando nós implantamos lá e ela veio com aquela aposta de 9 horas Guilherme

então o aluno entrava 7 da manhã saí 5:30 da tarde com horário de almal a

princípio Parecia que tava tudo certo hoje nós já entendemos e já existem

pesquisas que esse tempo todo dentro da escola em determinadas regiões não é

muito apropriado E aí você tem que escutar o tempo todo você tem que escutar Por

que que esses alunos estão desistindo da escola de tempo integral aqui Ah pera

aí nessa região aqui este modelo precisa baixar para 7 horas e aí quando você

fala pergunta pro aluno do tempo integral se ele gostaria de fazer cursos de

educação profissional é tudo o que eles pedem para nós ele diz o seguinte

secretário eu preciso sair aqui do tempo da da da Educação de tempo integral no

ensino médio e eu preciso sair com sabendo fazer alguma coisa além do

propedêutico e muito interessante que mesmo com todas essas dificuldades o

estado de Goiás hoje já se aproxima de 76.000 estudantes no fic o fic eu

consegui tocar Mesmo com pouco recurso o que nós estamos temos dificuldade Eu

não quero ranquear mas devemos estar aí muito abaixo do do desejado hoje em

Goiás é o ept você vê esse ano Ron que eu chego a 9000 estudantes no ept por quê

Porque ele não é ainda não é considerado acessível financeiramente ente para nós

secretários ele é realmente uma modalidade que exige um olhar eh e aí quando

você faz só vou encerrar mas quando você fala em olhar você também tá dizendo

assim eu preciso tirar eh de tal ação para poder investir aqui no ept é por

dentro que você faz isso afinal de contas você também não pode exigir de um

estado no rrf que ele tenha mais eh investimentos na educação se existe todas as

demais secretarias para serem dadas eh secretário só antes do Senhor

complementar aí é uma provocação pros dois a senhora falou das peculiaridades e

especificidades uma forma de reduzir a evasão acho que é um dos grandes

objetivos de todos né nessa nesse contexto eh o ensino técnico no período

noturno seria uma forma que a senhora falou integral enfim um período e Menor de

tempo mas e o noturno isso aí também estaria dentro desse contexto que a senhora

colocou o secretário depois também quiser falar sobre a experiência do Estado

bom Lembrando que esse menino hoje já não tem procurado muito o ensino médio

noturno para vocês terem ideia a nossa rede hoje ela tem lá 198.000 estudantes

matriculados no ensino médio e temos uma Procura pelo noturno de aproximadamente

30.000 meninos dentro desse 198.000 por quê Porque as cidades eh as zonas

urbanas por exemplo você não tem um transporte público em sua grande maioria

você não tem transporte público até 11 da noite você não tem as crianças que

moram mais distantes os pais não autorizam com razão tem tantos fatores que

interferem na na educação noturna Então eu não sei como que seria Mas cabe uma

pesquisa para entender nesse momento até hoje nós fizemos pesquisas para o médio

regular no diurno e para a eja no noturno aí a eja sim Goiás agora inicia a

implantação para educação de jovens e adultos no noturno Mas a secretaria de

educação não tem registros de alunos do ensino regular com demandas para cursos

técnicos noturnos demandas que justificassem Inclusive abrir ali uma modalidade

no período noturno Essa é a minha realidade eh vou falar primeiro dos Desafios e

depois eu respondo a tua pergunta desafios hoje eh eu vejo que são a gente tem

que abrir várias frentes de atuação eh ela não é algo simples a gente começou

essa discussão com a Fundação Itaú em 2019 no Paraná e a gente começa essa

discussão pensando primeiro como que quem que a gente escuta para pensar o curso

currículo eh a forma de de de oferta o arranjo produtivo local então a gente

abriu vários canais de discussão por macrorregiões o Paraná todo mundo muito

conhece muito parecido com Goiás o arranjo produtivo é no agronegócio pecuária

agricultura mas tem os grandes centros urbanos que são na área de saúde o curso

de saúde são cursos muito bem eh procurado hoje na área de educação também tem

demandado muito mas na área de de tecnologia também primeiro você vai olhar pro

mercado você não pode fazer um concurso para esse professor Porque como foi dito

aqui na nos painéis anteriores é um curso que muitas vezes ele tem uma duração

de uma década e depois ele fica saturado e já não tem mais a necessidade de

ofertar mais esse curso Então eu faço um concurso e eu vou ter um professor eh

para 35 anos concursado na rede e o que que eu faço com esse professor Então

esse é o primeiro ponto que tem que ser pensado segundo é como que eu apoio esse

professor da rede quando ele é da rede porque ele não tem uma formação

específica para dar e ministrar aulas do curso técnico ele é um engenheiro ele é

um Jornalista enfim é nessas áreas que ele é formado então ele não tem a parte

pedagógica então para isso você precisa criar uma formação continuada atrelada

esse professor terceiro você cria um material estruturado pedagógico de apoio

nas aulas desse professor e quarto você precisa escutar o principal interessado

que é o estudante Quais são os seus projetos Qual é o seu projeto de vida qual a

sua perspectiva de futuro tem que ter um diálogo constante com o estudante foi o

que a gente fez no Paraná eu posso citar aqui a gente fez pesquisas constante e

70% dos estudantes da rede estadual do Paraná tem interesse de fazer um curso

técnico então eh a gente depois cominou isso com os arranjos produtivos e a

construção inclusive do currículo foi dialogado com o arranjo produtivo e dessa

forma a gente vem ofertando cursos para os nossos estudantes da rede pública do

Paraná chegamos hoje dos 363.000 alunos da Rede do ensino médio chegamos a

97.000 alunos do ensino médio do Paraná realizando curs técnico no ensino médio

em relação à pergunta do noturno eu sou eu fui estudante do noturno e hoje eu

faço um enfrentamento do noturno o noturno é ele é um grande prejuízo ao

estudante todos os indicadores trazem baixa proficiência alto índice de abandono

escolar então o noturno tem que ser a exceção se for a oferta necessária para

aquele menino trabalhador que precisa contribuir na renda de casa a nossa rede

oferta ensino técnico mas a gente prioriza o ensino diurno por quê com a reforma

do ensino médio trouxe a perspectiva de oferta de 1000 horas por ano Ou seja eu

consigo dar seis aulas de 50 minutos por período com isso que era um curso

técnico que eu tinha antes da reforma que Durava 4 anos ele passou agora a durar

3 anos com isso ele passa a ser mais interessante para o adolescente porque o

adolescente ele é imediatista ele tem 15 anos se ele olha uma vai durar 4 anos e

o outro vai durar três obviamente que ele vai se Direcionar para o de 3 anos com

isso o ensino técnico ele passa a ser mais competitivo quando ele tem 3 anos o

ensino médio noturno Eh o meu caso ainda eu tô um pouco além da Fátima a gente

reduziu esse ano os 20.000 matrículas no noturno tínhamos 100.000 caímos para

80.000 então ainda é um número alto de meninos noturno a gente oferta o ensino

técnico e eh um outro Grande Desafio que que a gente precisa do apoio Aí sim do

INEP é na avaliação da educação profissional a gente precisa ter um instrumento

avaliativo como a gente tem o saeb o Enem é olhar paraa educação profissional

com algum criar um instrumento de dessa dessa avaliação a a o sistema S através

da CNI tem uma avaliação da indústria né Eu acho que dá para se inspirar

dialogar construir Mas precisamos pensar numa avaliação Para inclusive avaliar a

qualidade de que tipo de profiss a Gente Tá formando um ponto importante que foi

citado aqui eh já que a grande maioria dos Estudantes do ensino técnico tem boa

proficiência no saeb segundo ele tem alto índice de inserção no ensino superior

a gente vem acompanhando os nossos egressos da educação profissional do Paraná e

a gente já vem observando que eles têm mais facilidade inclusive em construir

qual carreira profissional no ensino superior ele vai ser direcionado eh eu

queria convidar novamente o Guilherme aqui é com o gancho que foi dado aqui pela

secretária no final da fala dela ela falou do custo né Eh da importância do

custo e E aí eu queria entender do Guilherme isso assim se pra gente viabilizar

essa expansão né É tudo sobre dinheiro né Eh Mas qual que é a sua visão se

realmente é é necessário muito mais dinheiro para poder e ampliar o número de

vagas e de matrículas no ensino técnico profissionalizante obrigado obrigado

Murilo eu queria falar justamente da essa questão dos cursos no contexto do

juros por educação né Que problema no final das contas que o programa veio a

resolver vou mostrar a tela aqui porque eu fiz as contas eu queria mostrar as

premissas dessas contas depois eu vou compartilhar esse material também então eh

a preocupação aqui de Por que a gente precisa de de mais dinheiro em particular

esse dinheiro da economia do juro né pra expansão do ept é que PT custa mais

caro que os demais itinerários né que o ensino médio regular de fato antes da

reforma né para torná-lo articulado a IPT custava o dobro do ensino médio

regular porque tinha que pagar o valor do aluno tanto a rede né quanto a rede

conveniada que oferece a formação técnica concomitante né de fato também o novo

fator de ponderação né com a revisão dos fatores no fundeb PR IPT ficou em 2,55

do ensino médio regular 1,25 então é o dobro né Tá entendido que o aluno DPT eh

deveria mobilizar aí o dobro de recursos do que do ensino médio regular Então se

a gente vai fazer uma expansão a gente vai precisar de recursos para fazer isso

eh Será que os recursos que o programa juras por educação eh poderia

disponibilizar poderia financiar pelo menos em parte essa lacuna de recursos

associado a expansão eh então uma pergunta sobre níveis Será que o volume dos

recursos vai ser eh minimamente suficiente diante da da necessidade de

financiamento para expansão e também sobre focalização Será que o dinheiro vai

pros Estados certos os estados que mais vão precisar de dinheiro eh porque é

onde o número de matrículas mais precisaria aumentar para essa expansão é para

fazer essa conta precisa fazer algumas premissas né então primeiro quais são

exatamente esses custos eu fiz com dois cenários um um cenário em que o aluno

DPT realmente custa o dobro do aluno do ensino médio regular eh vocês vão ver

que o a lacuna de recursos que vai ser aberta com esse cenário é tão gigantesca

que eu até fiz um outro cenário que eu acho que é mais pé no chão em que esse

aluno custa 30% mais só eh que ainda assim vai ser uma lacuna é muito relevante

vou mostrar os dois resultados aí a segunda pergunta é Qual a meta de expansão

expandi para quanto então eu fiz dois cenários também um para 30% das matrículas

e outro para 50% das matrículas Lembrando que nos sistemas que fazem isso bem

feito a gente tem 40 50% das matrículas às vezes até 60% das matrículas na ept E

aí depois claro Depende do ponto de partida de cada estado cada estado hoje tá

num numa certa cobertura da ept E esses são Dados que eu peguei do senso escolar

e com definições compatíveis com o monitoramento do PN é E aí quais são os

recursos que o estado cada estado já tem né Que Isso define o o valor disponível

ali por aluno ano eh do do nível médio e obviamente usando esses multiplicadores

DPT usando tanto vaf quanto Bat tá bom Então essas são as premissas dos meus

cálculos eu vou mostrar agora como é que ficam esses números né esse aqui

primeiro supondo que o aluno DP ter custo dobro do ensino médio tradicional

então aqui os as lacunas né os gaps de recursos por estado para expandir eh as

matrículas da para 30% do total do ensino médio na esquerda e na direita para

50% do total vocês podem olhar por estado aqui tá o ranking das lacunas por

estado e aqui o total vocês vem que se a gente quisesse expandir as matrículas

para 50% do do médio eh dos alunos alocados na IPT e se custar realmente o dobro

a gente ficaria com um GAP anual de 30 bilhões por ano em Recursos pro Brasil

que é inacreditável né só para São Paulo seriam 7.7 bilhões por ano então eh eu

acho que isso aqui é realmente assustador e provavelmente realmente não é

realista né Se eu colocar em 30% a mais do custo DPT em relação ao tradicional

ainda assim o GAP é muito grande você vê que se a gente expandir para 30% que já

é uma acho que uma expansão bastante relevante num primeiro Horizonte né quase

triplicar o número de matrículas que tá parecido com que tava nas metas né do do

pne então a gente teria aqui um GAP de quase 5 bilhões anual em relação ao que

custa hoje hoje né Eh para a gente financiar né Essas matrículas Então teria um

gap 5 bilhões tá aqui o ranking dos estados que que dá para ver aqui no topo do

ranking São Paulo e Minas aqui são realmente estados que se beneficiariam

diretamente pelo programa de luso por educação mas tem outros estados que se

beneficiariam que estão muito longe do topo do ranking no Rio Grande do Sul

mesmo aqui tá bem no meio por quê Porque já tem uma penetração alta DPT já tá

meio próximo dessa meta aqui eh e outros estados que não se beneficiaram diret

ente pelo programa tipo Pernambuco Bahia Ceará eles estão com altas necessidades

alta necessidade de recursos mas não são beneficiários do programa então isso

diz algo sobre eh focalização dele né E esse aqui são as minhas conclusões então

primeiro em questão de níveis a gente deve ter um GAP de financiamento mesmo no

cenário mais Modesto que é muito alto 5 bilhões por ano é similar ao pé de meia

né que tá ali mais ou menos em 7 bilhões por ano e é bem maior do que os

recursos que hoje o governo federal eh tem destinado à expansão do ensino médio

perdão do ensino em tempo integral n cinco vezes esse montante Então a gente vai

ter um GAP de recursos muito muito importante o cenário ambicioso ele é eu diria

é impagável né Lembrando que com a lógica dos novos fatores de ponderação não

tem mais dinheiro não é porque eu disse que ept é 2,55 que chegou mais dinheiro

para ept ela só vai tirar recursos das outras etapas então assim se custa mais

caro a ept vai faltar dinheiro para tudo o cobertor é curto então isso a

expansão ela é muito importante paraa educação e paraa economia mas ela vai

gerar um problema muito grave pro sistema pro sistema educacional como um todo

em termos de focalização como eu já apontei eh o programa ele em parte acerta

porque São Paulo Minas e Rio estão lá perto do da parte de cima do ranking em

parte porque eles têm muitos muitos alunos e um ponto de partida abaixo DPT mas

ele é em outros outras dimensões Né manda seriam muitos recursos destinados pro

Rio Grande do Sul tem não tem tanto espaço para quer dizer já tem um ponto de

partida muito mais alto de matrículas DPT e Outros tantos que vão ter gaps

enormes de recursos não vão ser né primeiramente beneficiados ali por esse

desenho do programa é a máxima que os economistas costumam dizer né quando a

gente tem dois problemas em geral você precisa de duas soluções então o auto

endividamento é um problema a lacuna de recurso é outro você precisar de dois

instrumentos diferentes acho que o programa pode ajudar mas ele tem focalização

imperfeito e meus colegas do próximo painel vão falar mais sobre isso mas para

juntar as duas coisas que eu falei a Epic lado ao meu ver pode mesmo ser a

revolução Educacional que o país precisa para ser mais produtivo para ser menos

desigual para ter menos evasão para ter uma população mais escolarizada e

produtiva mas para isso a gente vai precisar fazer muita lição de casa né a

gente precisa fazer melhores práticas para mapear demandas locais para definir e

medir qualidade para alocar recursos para promover essa qualidade né a gente

ouviu o secretário Ron falar de materais de qualidade recrutamento e formação de

professores especializados e eu tô falando aqui desses incentivos né Eh paraa

provisão eh seja conveniada seja própria né O que a gente vai precisar é de

recursos que não são são recursos para fazer frente a esse esse essa lacuna aí

que já que o aluno DPT custa mais caro mas para que essa formação seja feita com

qualidade que Garanta empregabilidade de renda para esse aluno senão a gente vai

gastar muito dinheiro e não vai ver mudar o nosso drama Educacional aí de evasão

muito alta e grande desigualdades eh na nas na tanto na graduação do ensino

médio quanto na colocação subsequente obrigado Mais uma vez obrigado Professor

vou aproveitar que o Senhor deu a deixa com conclusões tudo mais e já Seguindo

aqui pro nosso fim do nosso painel a secretária olhando o estudo do senhor o

senhor não conseguiu perceber mas ela tava ali olhando impressionada E aí não

sei se ela tava mais mais mais impressionado com os dados do estado dela ou

geral por favor não não Guilherme Parabéns pela apresentação incrível sabe assim

fazer a o cálculo da forma como ele fez por estado na verdade traz para nós

assim um choque de realidade muito grande né eu comecei falando da dificuldade

financeira que é hoje você elencar a prioridade e o acho que o Guilherme encerra

explicando o que não pode é deixar de fazer o que não pode é querer fazer o que

não pode é tentar encontrar a solução para fazer e isso a gente tem feito mesmo

estando no rrf mesmo tendo que priorizar outras prioridades por exemplo Quem de

nós aqui Consegue hoje não ajudar os municípios a investir na educação infantil

Nenhum de Nós Todos nós estamos focados no regime de cooperação no regime de

colaboração Então você vai desde o pequenininho até quando ele tá lá técnico

profissional eu eu há 3S anos atrás eu recebi an noi no Estado de Goiás nós

fizemos uma escuta e resolvemos de acordo com a escuta trazer para Goiás a

primeira escola técnica profissional Agro Agro mesmo tá inclusive com formação

para poder manusear os equipamentos Agro e tal há 3 anos atrás o edital para

para se inscrever e se matricular abre agora no segundo semestre Olha o tempo

que leva dentro do do público para você implantar uma política sabe no caso da

secti foi muito célere né Zé Frederico a gente caminhou em seis meses mas Quando

surge um problema a gente fica 30 dias para poder resolver porque realmente a

educação profissional com exceção de mina São Paulo agora o Paraná que tá muito

avançado pra grande maioria de nós a Paraíba como vocês viram Piauí mas assim A

grande maioria de nós nesse momento estamos focados na educação profissional a

educação profissional técnica profissional então Eh é muito importante que

aconteça essas discussões nós do Estado de Goiás o que nós queremos de verdade

oportunizar os nossos jovens colocar esse jovem no mercado de trabalho fazer

gerar emprego e renda é sabe fomentar a economia é isso que nós queremos como

esse dinheiro vai chegar para nós ou como nós vamos ter esses ajustes que foram

propostos aqui Tomara que de uma forma possível de se executar do jeito que tá

hoje quem conseguiu conseguiu porque tá fora do regime E também porque já tinha

essa política que acontecendo no seu estado há 10 anos atrás tá do jeito que tá

hoje você realmente tem que abrir mão de uma política que esteja para acontecer

para poder implantar a educação profissional então eu quero aqui Sem dúvida

nenhuma Agradecer o convite do todos e dizer a vocês que nós vamos trabalhar

juntos vamos continuar trabalhando juntos para que esse país possa colocar lá no

final da terceira série do Ensino Médio de verdade um jovem que durante o ensino

médio tenha conseguido escolher afinal de contas o que que eu gostaria de fazer

pelo ao menos até que eu eh faça o ensino superior e possa atuar de forma

profissional especializada mas de forma técnica ele ter escolhido Isso tá muito

obrigada pela oportunidade de estar aqui com o meu colega Ron e claro em nome do

estado de Goiás e do concede obrigada pelo convite Obrigado secretária

secretário também conferir os seus números lách que estão um pouco melhores né

quiser concluir já agradecendo eh acredito que primeiro Agradecer o convite eu

acho que é é válido sempre que tiver oportunidade ter o debate discussão sobre

políticas públicas educacionais sempre quem ganha é o Brasil é a nossa sociedade

eh em relação à educação profissional é uma decisão como o ministro colocou aqui

uma decisão política de gestão então eu posso falar aqui no Paraná para nós uma

decisão política de gestão ofertar educação profissional não é simples ela é

algo complexo Porque não basta você colocar um curso na escola e falar pro

professor se vira vá dar sua aula então sero bem direto é o que normalmente

acontece nas redes faz um documento eh orientativo vai dar aula sobre esse

assunto Professor o professor fica sozinho nesse nesse momento então precisa

apoiá-los Então tem que ser assertivo nas decisões da oferta dos cursos segundo

ponto e obviamente que o financiamento ele é é crucial é é importante ter o

apoio do recurso financeiro para você dar condições e estruturas de trabalho a

na ponta mas educação profissional eu vejo ela como uma algo como uma função de

transformação já falei isso aqui transformação social é uma oportunidade pro

jovem principalmente estudante de escola pública e obviamente que a gente tem

agora se avizinha uma reforma do ensino médio novamente que causa pra gente como

a Fátima colocou aqui a gente faz planejamento investimento prioridades Aí vem

uma nova reforma que acaba caindo tudo por terra aquilo que um trabalho que já

foi traçad aí aí há um a 2 3 anos que acaba tendo que retroceder então Eh que

que a Pricila a gente trabalhou muito nesse nesse nesse projeto de lei e e

sempre uma defesa pela educação profissional é algo que é importante porque é do

interesse do Estudante O estudante se conecta mais com a escola porque a gente

tinha um ensino médio que não tinha uma identidade era mais uma revisão dos anos

finais que o estudante fazia no ensino médio a o ept ele traz essa perspectiva

de futuro para esse para esse jovem para esse adolescente que tá se formando e e

a oportunidade de você levar o ensino técnico eh no Paraná a gente conseguiu

levar para mais de 50% das escolas do estado ensino técnico temos escola

especializada Escola Técnica na área de agricultura temos eh eh 26 escolas que o

estudante é filho do pequeno agricultor e ele passa a semana na escola é um

colégio integral ele passa a semana toda ele aprende na prática tem a fazenda

tem os animais que faz todo o trabalho e ele volta todo aquele aprendizado na

escola na escola e volta para casa como também temos a casa familiar Rural aonde

o estudante passa uma semana na escola e uma semana em casa então faz essa

pedagogia de alternância tudo isso é para levar é qualidade ô menino para

integrar os seus anseios de futuro e da do seu cotidiano com a escola a escola

tem que tá integrada a perspectiva desse adolescente ela não pode ser uma escola

distante foi o que foi dito aqui muitas vezes você pensa num curso né ô

Guilherme que não tem nenhum não tem nenhum contato com a perspectiva daquele

adolescente ou do arranjo produtivo local e o menino faz mais um curso que

depois ele se frustra ao final do curso que ele não consegue fazer estágio não

consegue eh fazer eh atuar na área que ele tá se formando então é o desafio é

grande mas tem que começar tem que começar por uma ponta a hora que começa já é

já 1% 2% que a gente já esgotou e vai faltar 97 E é assim que a gente pensa a

educação eh trabalhar com ela numa perspectiva que dialogue com os interesses

dos nossos adolescente mas também com a perspectiva de futuro dele ser integrado

porque quem é pobre quem e não tem perspectiva de futuro ele não tem tempo para

esperar então tem que começar obrigado obrigado professor o senhor quer fazer

uma amarração final rapidíssima pra gente passar pro próximo painel por favor

Claro só agradecer e reforçar imagina quase 30 anos ainda no governo Fernando

Henrique Paulo Renato disse que tinha uma visão que toda criança ia est

matriculada na escola ele atrás falaram para ele tá maluco como é que a gente

vai conseguir fazer isso e de fato essa visão quase se tornou idade aí antes da

pandemia a gente quase 100% dos alunos né nos anos de do ensino fundamental

matriculados na escola mas porque tinha os recursos suficientes para que isso

fosse verdade esse Ministério da Educação também tem uma visão todo aluno em

tempo integral e metade dos alunos do ensino médio Talvez 1/3 a metade no na IPT

só que para essa visão ser realidade vai precisar de recursos Ministério da

Fazenda vai precisar achar as maneiras de garantir os recursos suficientes para

que tempo integral e ept se tornem realidade né o GAP de recursos é enorme 1

bilhão por ano que tem hoje disponível para o aumento das matrículas em temp

integral é insuficiente e para IPT então nem se fala então espero que essa

conversa conjunta possa mobilizar aí os recursos para que essa visão do MEC

também deixe seu legado histórico obrigado mais uma vez muito obrigado obrigado

Murilo Obrigado obrigado muito obrigada secretária secretário Professor pela

contribuição extremamente objetiva muito obrigada ao Fernando e ao Murilo pela

mediação uma uma rápida foto por gentileza antes da gente D início ao último

painel muito obrigada ah PR pro Guilherme poder sair muito obrigada muito

obrigada a todos bom agora vamos ao painel quat de debates nosso último painel

desta manhã o tema desta vez a proposta sobre a ótica das Finanças Públicas nos

Estados para discutir o assunto estarão no palco Carlos Xavier presidente do

conaz e secretário de tributação do Rio Grande do Norte por gentileza secretário

[Aplausos] ecista da investimentos e exeto da fazenda e planejamento do Estado

de São Paulo por gentileza Obrigada pela internet teremos a participação remota

de Vilma Pinto diretora da instituição fiscal independente do Senado Federal

seja bem-vinda obrigada min gza seja bem-vindo e para fazer a mediação a dupla

de jornalistas do valor Fernando e luic por gentileza acomodem-se o espaço de

vocês um ótimo debate Muito obrigado a todos já quase Boa tarde acho que temos

em Lu a missão de fazer uma a grande amarração final tenho certeza que os nossos

convidados nos ajudarão com brilhantismo Eh vamos começar então com com as

nossas eh falas iniciais eu acho que a gente tem também o desafio de tentar eh

condensar nossas nossas considerações para que depois a gente possa trocar uma

ideia Enfim fazer as perguntas eh Felipe vamos lá a gente vamos começar bom

primeiro cumprimentar os organizadores agradecer o valor econômico pelo convite

cumprimentar a Todos Pela Educação também a Priscila que é uma amiga de longa

data e o Fernando que eu fonte eu sou fonte aqui de vários jornalistas que estão

aqui e são grandes amigos também e os meus dois ex-colegas de de Conce Fas e de

confaz o Cadu Xavier hoje presidente do Conce Fas também um prazer estar aqui

com você Cadu e o Luiz Cláudio Luiz Cláudio é um craque sabe tudo do ICMS ajudou

a superar a crise dos combustíveis em 2022 que não foi uma coisa fácil a lei

complementar 194 toda aquela confusão e a luico também né luico Falo desde que

ela era do Estadão depois Valor Econômico então um prazer também est aqui com

vocês eu trouxe alguns slides até para organizar melhor o tempo então se vocês

puderem colocar aqui indo direto ao ponto eh eu vejo da seguinte maneira eu acho

que a proposta é bem intencionada o secretário Rogério Ceron e o secretário

Dario durigan já explicitaram bem que o objetivo do governo é tentar renegociar

a dívida mas dessa vez com uma contrapartida que não vai ser aquela coisa da

exigência de reformas estruturantes ou então de exigir uma reforma da

Previdência ou que não se conceda reajuste salarial aquela coisa draconiana que

nunca acontece depois o estado vai pro Supremo Tribunal Federal e ganha uma

liminar e fica sem pagar as parcelas e os juros né O Chamado serviço da dívida

então a intenção é boa mas o Guilherme lichan que aliás eu fui calouro do lix Na

graduação em economia na GV lá em São Paulo ele não poder poderia ter sido mais

feliz e acho que foi muito boa essa sequência dos dois painéis porque ele

mostrou o seguinte Qual é o objetivo do programa é uma proposta paraa educação

ou é uma proposta para resolver a dívida dos Estados porque o primeiro que

conseguiu mostrar o seguinte colocar números do ponto de vista da expansão do

ensino técnico Quanto custa por cabeça né per capita E aí fazendo simulações etc

etc foi ele né o o do do ponto de vista do governo Pelo menos eu não vi com essa

clareza né qual seria o objetivo inclusive por estado porque passando aqui o O

slide né A gente vai ver o seguinte que os estados mais individados e não tá

passando que os estados mais endividados como já foi falado são quatro né Minas

Gerais São Paulo no próximo aqui ah Rio Grande do Sul e Rio de Janeiro né só que

dentro desses quatro desse grupo desses quatro primeiros ali Rio de Janeiro com

188 por. Rio Grande do Sul com 185 Minas Gerais com 167 me corrige aqui o

secretário e São Paulo com 127 esses dados são do relatório de gestão fiscal do

terceiro quadrimestre do Tesouro Nacional que o Tesouro Nacional compila né você

tem quatro realidades completamente diferentes por exemplo São Paulo nunca ficou

inadimplente Mas por que que o Governador Tarcísio quer a renegociação Porque

tem uma coisa na lei de responsabilidade fiscal que chama-se limite pro serviço

da dívida não limite para esse numão grande aqui que nós estamos vendo que é

aquela linha vermelha que é o 200% da receita corrente líquida é um outro limite

que é pro fluxo o juro e a parcela quando você soma a amortização Aliás o juro e

a amortização que compõe a parcela eles não podem passar de 11,5 da rcl em São

Paulo já tá batendo nesse montante só que tem excesso de oferta o bid quer

emprestar para São Paulo o Banco Mundial quer emprestar sempre foi assim o o

banco que vocês quiserem imaginar aí quer emprestar para São Paulo por qu porque

tem condições econômicas emente boas e que geram essa atratividade então o

problema de São Paulo não tem nada a ver com do Rio Grande do Sul que não tem

nada a ver com de minas que não tem nada a ver com do Rio de Janeiro né Rio

Grande do Sul por exemplo Minas e Rio tem um um problema que tem o Matiz e agora

falando não como como ex secretário mas como especialista em contas públicas que

a meu ver é uma questão do gasto de custeio o gasto de pessoal o gasto

Previdenciário que não se conseguiu atacar até o momento assim como a união

também não conseguiu atacar Eu lembro que quando foi aprovada a reforma da

previdência eu falei em 5 ou 10 anos vai ter que fazer uma nova reforma da

Previdência é sempre o o chato na sala né o fiscalista é sempre o chato na sala

em cinco ou acabou de aprovar o governo quase se matou para aprovar no Congresso

aí vem o chato e fala em C ou 10 anos vai ter que fazer outra reforma e o que

que tá acontecendo agora o ipia acabou de publicar um estudo mostrando que daqui

do anos anos vai precisar fazer provavelmente uma nova reforma da Previdência

porque a gente aprovou uma reforma paramétrica que fixou uma idade mínima que

não é corrigida pela evolução da sobrevida a gente não endereçou a questão da

Previdência dos militares não resolveu o problema da da Previdência da

aposentadoria Rural etc etc então o ponto é que esses três estados Rio Grande do

Sul Minas Gerais e Rio de Janeiro e cada um é um mundo à parte eles têm

problemas estruturais que precisam ser resolvidos com soluções específicas para

cada um desses estados e como é que você faz isso é fácil falar né a gente sabe

lá no conef e no confaz como é difícil E lá se trata de um tema só né que é o

tributário e normalmente só de benefício fiscal 90% do tempo exceto quando tem

um tema como reforma tributária e coisas do tipo né Cadu e e Luiz Cláudio agora

quando a gente eh entra então nesses desafios federativos eu acho que nós temos

que tirar do Papel uma proposta que na verdade não é uma proposta é um um

dispositivo que tá na lei de responsabilidade fiscal Ah não passa o slide aqui

ah tá pode passar então dois por favor mais um um dispositivo que tá na lei de

responsabilidade fiscal que chama conselho de gestão fiscal faz 24 anos que nós

promulgamos essa lei chamada lei de responsabilidade fiscal se ela tivesse sendo

respeitada não precisava de novo Arc bolso fiscal não precisava discutir

orçamento de Guerra Não precisava discutir teto de gastos nada disso só que o

problema é que cada Tribunal de Contas em cada estado interpreta a lei

complementar 101 do jeito que dá na telha essa que é a verdade e ainda tem o

Ministério Público de contas e os órgãos de controle eles são turbinados tem os

melhores salários e o Executivo a míngua isso eu senti na pele e eu fui

secretário de São Paulo e eu tenho certeza que todos os estados é a mesma coisa

você vê o TCE no caso de São Paulo era do outro lado da rua ali na Rangel

Pestana do lado da Catedral da Sé o prédio do TCE os servidores bem pagos

fazendo concurso a todo momento e a gente sem servidor por exemplo no tesouro

estadual para cuidar de um orçamento de 312,7 bilhões que era o número da minha

época só de ICMS era 202,3 bilhões então há algo de podre no reino da Dinamarca

e nós precisamos corrigir e eu acho que o começo da solução passa por instituir

esse conselho de gestão fiscal Ah mas mais uma instituição nós não temos nenhuma

instituição para discutir a Federação só tem o confaz que foi criado pelo ex-

ministro Delfim Neto e em 1975 sabe por quê Porque ele começou a ver que os

secretários de estado estavam reportando para ele que a situação fiscal dos

estad estava começando a dar problema ele falou bom deixa eu reunir todo mundo

numa mesa aqui em Brasília os 27 secretários de estado da fazenda e planejamento

antes que a coisa comece a degringolar foi por isso que foi criado o confaz e a

única estrutura que nós temos até hoje para discutir a Federação o cgf é

interessante porque é muito mais democrático porque ele já tem as todas as

atribuições definidas na lei complementar 101 a lei de responsabilidade fiscal

Você pode discutir normatização contábil pode discutir política tributária pode

discutir a política fiscal dos Estados pode discutir a renegociação da dívida

etc etc etc e por exemplo numa mesa como essa não estariam sentados só os

secretários do Estado estaria o Supremo Tribunal também ou seja é uma diferença

importante porque o Supremo normalmente ele entra só na etapa posterior então

quando dá problema o estado vai lá pede pro Supremo interrompe o pagamento né E

aí abre uma folga temporária etc etc como tá acontecendo agora em alguns estados

nesse modelo se a gente conseguisse instalar o conselho de gestão fiscal a

proposta juros por educação por exemplo poderia estar sendo discutida lá cada

estado e apresentar suas contas aí os especialistas como nós estamos fazendo

nesse evento organizado pelo valor que aliás eu parabenizo o valor econômico por

isso porque eu acho que é extremamente importante viu Ana em seu nome aproveito

para cumprimentar a todos eh poderia tá sendo feito lá no conselho de gestão

fiscal então não dá pras coisas saírem da caixola desse jeito de repente juros

por educação Como assim pera São Paulo tem 285 bilhões de dívida paga uma fábula

de juro o Acre não tem dívida Paraná tem dívida líquida negativa salva o melhor

juízo tá tudo ali na tabela que eu acabei de mostrar então como é que nós vamos

fazer quer dizer quem não tem dívida e não paga juro não vai ter direito a ter

uma política de educação Guilherme mostrou muito bem foi muito feliz o exercício

dele o Rio Grande do Sul por exemplo super endividado né aí na tabelinha do

Guilherme ele não tá proporcionalmente eh eh beneficiado eh como deveria se a

gente fosse considerar esse critério então não não me parece que a saída seja

essa e tem um outro problema pessoal como é que faz com os estados que não t

dívida se eu fosse Governador tava aqui o governador da Paraíba até há pouco se

eu fosse governador da Paraíba eu estaria pleiteando desde logo com o ministério

da fazenda o que já é um erro porque numa Federação não é o secretário ou o

governador peteando pro Ministro deveria ser uma coisa por igual homogênea

horizontal como numa Federação deveria ser Porque aqui nós temos três níveis de

Federação né de entes Federados o estado o município e união e estaria

preocupado com isso porque então o estado dele não vai ter direito né então me

parece que essa proposta tem um erro já de concepção eu acho que nós temos que

começar pelo conselho de gestão fiscal é uma baita oportunidade da gente tirar

do papel já tá pronto na lei de responsabilidade fiscal né Eu acho que é uma

grande oportunidade que a gente tem pode passar o próximo por favor já vou

terminar tá e a solução para cada caso né eu tô exemplificando aí os quatro

endividados né mostrando que cada um tem que ter uma saída diferente São Paulo

por exemplo por que que a gente não pode pensar numa proposta de flexibilização

da 9496 e deixa São Paulo se endividar sem custo paraa União deixa o risco para

São Paulo ele quer em vez de 11,5 por 12,5% ou 15% problema de São Paulo ele

quer aumentar investimento o problema dele não tem que o tesouro ficar de

maneira paternalista dando aval para São Paulo comprar trem para CPTM pro metrô

E aí fica pendurado aqui no tesouro Porque se o partido é o PT e e o o o

presidente é é o bolsonaro e vice-versa você trava o aval pra concessão do

empréstimo imagina quer dizer isso não é uma federação nós temos que mudar

radicalmente isso tem que dar mais liberdade pros Estados o governo Fernando

Henrique fez um milagre quando aprovou a lei 9496 porque nós estávamos em

pandarecos na questão Federativa fiscal O que que a união fez uma coisa

inteligente Assumiu a dívida dos estados e das capitais São Paulo emitia as

paulistinhas emitia títulos né tinha um prêmio em relação ao tesouro E aí o

tesouro falou o seguinte olha credores agora eu vou emitir letras financeiras do

tesouro para vocês e os estatos ficam devendo para mim qual vai ser o juro igdi

mais 6 igpd + 7 ou igdi + 9 só que lembra que na época a taxa Seli era de quase

50% lá nos idos de 2012 a 2014 o prefeito de São Paulo era era o o atual

ministro Fernando Haddad e conseguiu-se a renegociação que hoje na época eu

critiquei mas hoje eu vejo que foi boa que foi a troca retroativa do indexador

porque você tinha uma uma disparidade entre a correção da dívida mobiliária

Federal e essa maluquice de gpdi mais 9 né no caso do Município de São Paulo que

tinha ficado inadimplente e subiu de 6 para 9 uma taxa real de 9% então o que

que se fez corretamente você trocou por IPCA mais 4 ou Celi aquilo que fosse

menor então não cabe mais falar em trocar indexador na minha modesta opinião

porque IPCA + 4 ou celic já caminha junto com o custo médio da da dívida

Imobiliária Federal a ou a taxa implícita da dívida bruta o que você quiser

então o que que cabe fazer o que cabe fazer é ver caso a caso e discutir

renegociações tendo presente que o que nós vamos fazer é perdão de dívida e se

nós vamos dar perdão de dívida nós temos que ter contrapartidas que infelizmente

não devem ser aumento de gasto devem ser corte de gasto e aumento de receita

Essa é que é é é o duro recado que infelizmente eu gostaria de dar mas é é o que

eu penso e e gostaria de deixar isso para vocês e a solução para cada caso né

ela precisa ser eh eh distinta e o Locus para isso é o conselho de gestão fiscal

Se nós formos de novo por essa história de pensar as soluções dessa maneira que

vem o governador do estado no gabinete do ministro aí Pensa a solução aí São

Paulo corre aí vem o Rio Grande do Sul aí vem Minas Gerais aí vem Rio de Janeiro

não dá tem que ser uma coisa em que todos sejam tratados da mesma maneira

Inclusive a união não existe uma subordinação dos Estados à União ou nós somos

uma federação ou nós não somos próximo slide E aí eu termino e aí eu tenho

quatro propostas para não ficar só na crítica A primeira é que eu acho que para

estados que tem condições econômicas nós temos que flexibilizar os critérios

para tomada de crédito e financiamento ou alguém acha que o Japão é pior do que

o Brasil porque tem 200% de de dívida PIB e o Brasil tem 75 não o fato de ter

dívida alta não significa que São Paulo seja um estado economicamente ruim ao

contrário ele tem dívida alta e pode ter uma dívida ainda maior porque a dívida

é sustentável porque a receita cresce mais que o numerador a receita que tá no

numerador cresce mais do que o que tá no numerador que é a dívida consolidada

líquida Então tem que ter FIB ilização e essa flexibilização tem que ter

critérios objetivos obviamente definidos em lei em acordo com todos os estados a

união e o Supremo Tribunal Federal porque se não tiver o Supremo no meio vai

sempre judicializar quando a coisa apertar e é sempre assimétrico nós já vimos

porque é só ver o track Record das decisões do supremo renegociação com

contrapartidas no bojo do Conselho de gestão fiscal Ou seja eu acho que tem que

ter renegociação sim para Minas tem que ter pro Rio Grande do Sul e tem que ter

pro Rio de Janeiro porque não tem outra saída a dívida se isso não for feito vai

explodir e o custo vai recair sobre o tesouro ou seja sobre o grosso da

sociedade de novo então tem que ter renegociação mas tem que ter contrapartida

tem que fazer o arroz com feijão nós não temos que inventar moda tem que colocar

o seguinte Quais são as contrapartidas exequíveis o erro do regime de

recuperação fiscal chamado rrf foi o mesmo erro do Tet de gastos a emenda

constitucional 95 exigiu dos Estados uma montanha de contrapartida como se de

repente todo mundo fosse ficar fiscalista e daí ninguém consegue cumprir aí você

começa a ter desvios da regra começa a ter pedido de mudança do regime etc etc

né até para não falar na dificuldade de adequação para entrar no regime né E aí

eu coloco uma coisa também que até a a Vilma Pinto tá aí ela pode ver o que ela

acha também mas eu acho que nós temos que ter um órgão independente para checar

todas essas contas essas propostas lá no bolso no no bojo do Conselho de gestão

fiscal que seria a instituição fiscal independente que já tá instalada na casa

da Federação inclusive que é o Senado Então tem que ter alguém checando as

contas acompanhando o regime de recuperação ou o que quer que a gente faça de

novo lá no âmbito do Conselho de gestão fiscal e fazendo contas e claro que o

órgão tem que ser empoderado para isso tem que ter gente tem que ter orçamento

etc etc e a última coisa também que eu acho importante porque as propostas de

renegociação elas sempre acabam redundando em transferência de recurso que

aqueles estados que não t dívida obviamente o incentivo que você gera é o

seguinte bom eu não tenho dívida eu quero dinheiro então aumenta o fundo de

participação é óbvio que isso vai acontecer já tá acontecendo né E aí o que que

se faz a prova se no Congresso um aumento de mais um pedacinho do i pi do

Imposto de Renda pro fpm e pro fpe nós precisamos começar a ter uma coisa

diferente disso temos que ter burocracia qualificada em todas as regiões do país

começando pelas capitais para ter projetos de investimento Aqui nós temos um

tesouro em Brasília que chama senap escola nacional de administração pública

outra outra herança Bendita do governo Fernando Henrique Cardoso que foi a

reforma gerencial do do Estado né do ministro Bresser Pereira Então por que não

a enap que já faz isso ampliar o seu escopo e também dentro dessa dessas quatro

propostas né passar a oferecer treinamento eh junto com a união com os estados

que também tenham burocracia mais qualificada relativamente a dos outros nas

diferentes áreas né Eh oferecer esse tipo de eh treinamento e aí para terminar o

último slide só queria mostrar o o custo da proposta do juros por educação passa

pro próximo por favor essa proposta juros por educação custa a bagatela de

257,40 também pela waren que eu posso disponibilizar mas o resumo tá aí e o

resumo da Ópera secretário Luiz é que se a gente reduzir de IPCA + 4 para IPCA +

2 luico O que vai acontecer é que começa custando de 15 a 16 B por ano como os

contratos da lei 9496 da 159 etc etc eles vão por mais 20 anos 23 anos né até

2046 então se eu só somar isso nominalmente dá 257 27,4 bilhões de reais um tiro

no escuro porque nós não estamos sabendo Nem Qual é o objetivo o Guilherme

lichan colocou hipóteses aqui nós queremos aumentar em 30% em 50% o acesso eh

Que tipo de ensino é esse que regiões vão ser mais beneficiadas e aquelas que

não têm dívida etc etc etc Então são essas as coloc que eu gostaria de fazer

agradeço Mais Uma Vez pelo convite prometo que eu vou economizar na segunda

intervenção obrigado viu Fernando Obrigado Felipe eu achei que a Vilma tinha

desconectado depois que que você tinha sugerido mais trabalho para ela na if mas

graças a Deus ela ainda nos acompanha Vilma a gente queria ouvir a sua avaliação

Enfim uma consideração Inicial sobre os pontos positivos os desafios do programa

por favor bom obrigada quase desconectei Tô brincando eh gostaria de agradecer o

convite de participar desse debate eu acho que é um debate muito produtivo né na

medida em que a gente consegue eh observar a visão de diferentes áreas né e

consegue pensar eh soluções não só olhando do ponto de vista dos indicadores

educacionais mas também dos riscos e das oportunidades relacionados à questão

fiscal e agradecer pelo convite parabenizar né O Valor Econômico ao Todos Pela

Educação a todas as parcerias que Estão organizando esse evento bom eh também

gostaria de cumprimentar meus colegas de painel né Eh fiz várias anotações

relacionadas ao que o Felipe colocou eh concordo com boa parte delas né Eu acho

que a gente realmente eh precisa pensar essa questão eh do do do projeto à luz

do que ele se propõe né Qual é o objetivo eh a impressão Inicial é de que o

objetivo é fazer uma renegociação eh de dívida dos Estados eh pensando numa

contrapartida diferente das que a gente usualmente tem observado nos nas

diversas renegociações que ocorreram eh nos últimos anos e Muitas delas

necessidade de ajuste ou com judicialização ou até mesmo com questões eh

relacionadas ao não cumprimento dessas exigências eh e pensando também num

potencial resultado eh positivo do ponto de vista de produtividade eh dos

indicadores né produtividade da economia bom nesse sentido acho que o mérito da

proposta é positivo né gente pensar eh em melhorar os indicadores educacionais

eh sendo isso um dos incentivos para que ISO tenha uma uma contrapartida para

questão da dívida dos Estados eh e tem do ponto de vista de de de riscos né e de

questões a serem observadas esses pontos que o Felipe colocou né Eh a questão

dos Estados ela é muito heterogênea seja do ponto de vista fiscal eh também do

ponto de vista de necessidades educacionais né então acho que é importante a

gente pontuar essa questão da heterogeneidade eh dos entes da Federação Por

conseguinte eh essa proposta naturalmente não vai abarcar né não vai atender eh

as necessidades de todos os estados eh até porque alguns tem em taxa de

endividamento elevada outros não eh como o Felipe colocou a questão do de São

Paulo que apesar de ter um juros elevado eh uma dívida elevada eh tem

dificuldade tem uma capacidade de pagamento eh eh razoável mas tem dificuldade

de ampliar esse endividamento por conta de outros indicadores que são previstos

na lei de responsabilidade fiscal eh e nesse aspecto aí eu coloco uma outra

questão relacionada à questão do dos entes eh estaduais eh e os objetivos né

dessa proposta de tantas outras de renegociação de dívida dos Estados né Eu acho

que é importante Sim a gente fortalecer os instrumentos previstos na nossa lei

de responsabilidade fiscal isso passa pela criação eh do Conselho de gestão

fiscal que tá previsto na lei de responsabilidade fiscal e até hoje eh não foi

implementado eu acredito que a não eh criação desse desse conselho de gestão

fiscal eh é um dos motivos pelos quais a gente discute hoje eh crise eh nos

governos estaduais né Eh se a gente observar desde a criação da lei de

responsabilidade fiscal os indicadores dos Estados passou por um período de

melhoria contínua no nos indicadores fiscais na na primeira década de vigência

da lei de responsabilidade fiscal eh teve um período de expansão do

endividamento das op ações de créditos eh mesmo para estados que não tinham

quando a gente olha os indicadores de capacidade de pagamento não tinham

capacidade né de onrar esses compromissos a gente observou a expansão de crédito

inclusive para esses estados e com isso né Eh junto à crise econômica e e outros

fatores a gente observou um declínio na nas contas dos governos estaduais e por

essa questão a gente eh hoje e já há alguns anos eh a gente discute tantos

projetos de renegociação de Socorro a a aos entes eh estaduais e e com com ou

sem contrapartidas né então acho que nesse sentido né de necessidade de

renegociação de mais uma rodada de negociação de dívida eu acho que eh se

tivesse existido Né desde o início eh esse conselho de gestão fiscal a gente

poderia ter observado eh Talvez um um cenário diferente eh pro contexto atual eh

adiciona essa questão a necessidade de fortalecer também os instrumentos

previstos na lei de responsabilidade fiscal eh relacionado aos outros

indicadores e metas que precisam ser cumpridos Então nesse sentido a gente tem

eh assim o Felipe colocou a questão do limite pro serviço da dívida né a gente

tem vários outros eh indicadores que M das vezes se a gente não acompanha tão de

perto mas que precisam ser observado eh e um deles né que que é um indicador que

que eu gosto de utilizar como exemplo de necessidade de fortalecimento eh dos

princípios e objetivos da lei de responsabilidade fiscal é que a gente tem

limites relacionados ao gasto que tá eh que precisam ser observados nesse

sentido eh que tá relacionado por exemplo a questão de gasto com pessal né Eh

Muito provavelmente né a gente eh eu tive a oportunidade de acompanhar um

pouquinho pelo YouTube do do painel em que o o secretário serão participou eh e

uma das coisas que foi colocada é que esse essa proposta inicialmente não vai

entrar eh nos indicadores de mínimo da da educação mas esses gastos

eventualmente podem afetar outros indicadores da lei de responsabilidade fiscal

então a sustentabilidade do programa ela tem que ser observada não só do ponto

de vista eh do gasto do mínimo para para educação ou do do resultado que possa

ter de economia no serviço da dívida Mas quais são os custos Associados a isso

Felipe fez essa conta né que eh me assustou um pouco o número Felipe de 27,4

bilhões de custo eh Imagino que paraa União em função desse programa Mas isso

também pode gerar impacto em outros indicadores eh que precisam ser observados

para fins de sustentabilidade fiscal dos estados e um deles né Eh eh eu imagino

que mais matrículas vão necessitar de mais profissionais e sugera também

impactos e necessidad de avaliar a sustentabilidade dessas medidas à luz dos

gastos com pessoal eh Esse é um dos exemplos e tem tantos outros indicadores que

a gente precisa observar em relação a isso né Então nesse sentido eh quando a

gente Olha esses indicadores a gente consegue observar principalmente no gasto

com pessoal eh tem um estudo né que foi feito pela eh Celene Pérez Nunes eh tipo

a tese de doutorado dela ela faz uma pesquisa eh com os tribunais de conta dos

Estados de como eles interpretam a lei de responsabilidade fiscal eh e em cada

um dos seus estados à luz de alguns indicadores e quando a gente olha o

indicador de despesa com pessoal na maioria dos Estados Você tem uma

flexibilização um afrouxamento eh da regra em relação ao que a gente observa de

fato na na na lrf e quando a gente compara os indicadores reportados e e

oficiais né que são válidos por pelos Estados e o número a eh harmonizado né

pela a Secretaria do Tesouro Nacional a gente vê discrepâncias eh significativas

e Muitas delas era sempre no sentido de que o número harmonizado pelo tesouro

ser maior do que o número eh que consta nos relatórios eh de gestão fiscal dos

Estados Então nesse sentido eu acho que o fortalecimento dos instrumentos

previstos na lei de responsabilidade fiscal é um passo essencial para que a

gente consiga atingir esses objetivos de sustentabilidade eh das contas dos

Estados eh do ponto de vista Educacional acho que de fato a proposta tem esse

mérito de tentar eh melhorar a qualidade e o acesso Educacional né E com isso

trazer ganhos de produtividade pra economia Bom eu acho que é isso eh para uma

primeira provocação troue alguns temas muitos deles em em consonância com o que

o Felipe já troue obrigada Obrigado Vilma secretário Carlos Xavier por favor

Obrigado Fernando Boa tarde a todos e todas primeiramente parabenizar o valor

econômico pela realização do evento Todos Pela Educação a Fundação Itaú eh

cumprimentar meus colegas aqui de painel Luiz Cláudio nosso Felipe Fernando e Lu

aqui mediando e vendo Felipe aqui e eu acho que a gente precisa fazer para até

para iniciar e ser de forma bem objetiva a minha fala fazer um registro

histórico sabe Felipe é é muito bacana né a gente pode ter visões e eu vou

colocar algumas delas até contrárias ou críticas ao programa mas é muito legal a

gente viver essa quadra de uma forma completamente diferente que a gente viveu a

quadra anterior e o Felipe viveu o Luiz Cláudio também era já era já estava

secretário lá em Minas Gerais onde era quase que foi quase que inexistente um

diálogo como esse era um clima de de conflito federativo né a união olhava os

estados como inimigos sempre e culpando os estados pelos problemas do país então

só essa essa esse momento né de de de de eh debate né federativo a gente tem que

elogiar sempre que a gente tem oportunidade dito isso eh até pra gente

economizar o tempo a minha presença aqui a do e a do e a do Luiz Cláudio acho

que fica legal pra gente Eu sou secretário do Rio Grande do Norte tô aqui

representando o Conce Fas que eu estou presidente do Conce defaz eh eu vou

dividir com com com o luí pra gente economizar tempo eu vou falar sobre do sobre

o ponto de vista dos Estados não endividados que é o meu estado Rio Grande do

Norte e o Luís vai falar como gestor de um estado super endividado como como a

gente tá tratando aqui também cumprimentar as nossas colegas nossos colegas

secretários de educação Muito legal essa essa interação a gente tá falando aqui

de um tema eh que é fundamental pro desenvolvimento do país eu tenho que ter

muito cuidado Lu quando falo sobre esse tema que todos sabem a origem da

governadora do meu estado a Governadora Fátima Bezerra ela militou pela educação

a vida toda então eu tenho que ter muito cuidado com minhas palavras aqui senão

leva um puxão de orelha quando chegar lá no Rio Grande do Norte vamos lá pessoal

eh sobre o ponto de vista dos Estados não endividados né Eh a gente coloca

obviamente que a gente quer participar de alguma forma a gente não tem o

problema do superendividamento e o Felipe fez um raio x muito muito preciso aqui

de cada de da característica da dívida de cada estado desse superendividado nós

os outros 23 23 as 23 unidades da federações que não da Federação que não temos

essa característica nós temos outros problemas né eu falo com muita

tranquilidade do estado que tem uma maior comprometimento de gasto pessoal do

país nós temos outras dificuldades nós não temos um grande de endividamento com

a união mas nós temos outras demandas então um programa como esse que vem tratar

do federalismo brasileiro ele não pode deixar de fora A as 23 unidades federadas

que não tem esse superendividamento eu não gosto de usar nas minhas falas essa

questão que tá se colocando muito aos Estados que fizeram o dever de casa eu

acho que não é legal eh divide eu acho que talvez não seja nem justo com a

história cada endividamento desse tem sua história por trás E aí que não cabe

aqui a gente fazer esse debate agora mas a gente não pode esquecer os estados

que não tê esse perfil de endividamento e é preciso que a gente tava olhando e

mais um aqui a a a a a elogiar a apresentação do Guilherme Fantástico A análise

que ele fez ali eh colocando fazendo esse esse esse essa comparação dos Estados

superendividados com a carência lá no no ensino profissionalizante acho que isso

por por si só ele já traz uma crítica à estrutura do programa e é uma crítica

construtiva ninguém aqui quer ser contra o programa nada disso mas eu acho que é

é uma crítica basilar na premissa do programa na falta talvez no desvio do foco

do programa então do ponto de vista dos dos Estados não endividados a gente

precisa se inserir Nesse contexto eh seja através do do juros pela educação mas

com outro viés porque se a gente tem uma dívida pequena o juros é muito menor do

que a nossa a nossa necessidade lá do Rio Grande do Norte por exemplo que foi

colocada pelo Guilherme aqui a gente tá fazendo investimento em 10 ynes né que é

são 10 institutos estaduais eh de ensino profissionalizante A Governadora a

gente teve acesso àqueles precatórios do fundef e a Governadora usou esses

recursos para para investir nessas 10 nesses 10 institutos profissionalizantes

Então a gente tem interesse de ter recurso para financiar esse esse esse esse

programa mas também temos outras eh eh outras necessidades nós eu eu vou falar

isso que a gente tava conversando aqui luí para trazer o debate a gente tá

vivendo a realidade lá do Rio Grande do Sul e ficar aqui também registrar aqui a

minha a minha solidariedade a todo o povo gaúcho a nossa secretária Priscila

colega lá do do confaz e do Conce faaz eh Talvez esse quadro que foi trazido

aqui pelo Guilherme e esse contexto de tragédia que o Rio Grande do Sul hoje

vive essa essa essa necessidade pequena que o Rio Grande do Sul tem para cumprir

a meta da da da do ensino profissionalizante abra um viés para que o excedente

eh dos juros que eles vão economizar seja utilizado paraa infraestrutura porque

eles vão ter uma grande necessidade de infraestrutura de de refazer quase que

toda a infraestrutura do Estado então esse debate pertinente eu acho que a gente

tem que dialogar reconheço aqui enquanto presidente do Conce faaz a

disponibilidade do Governo Federal de fazer esse diálogo até porque como o

Felipe falou não vai ser eh um um uma e também não é essa a intenção deles uma

caixa preta com um programa pronto que vai resolver o grande problema e a grande

necessidade de eh ensino profissionalizante no país então em linhas Gerais para

economizar o tempo eu vou deixar o Luís falar mais sobre a realidade dos dos

Estados individados era isso que eu tinha para contribuir no debate aqui nessa

nessa primeira intervenção Obrigada secretário Então vamos passar a palavra pro

secretário Luís Cláudio para falar sobre a situação dos estados individados e

como é que eles estão vendo esse programa aí eh bom Boa tarde a todos antes de

tudo queria agradecer ao convite feito pela pelo valor econômico pela Fundação

Itaú Todos Pela Educação cumprimentar aqui os meus colegas e todos aqui que

estão eh nos vendo eh o desafio pro proposto pelo Governo Federal nos juros para

educação é Um Desafio enorme que tem que ser eh eh aplaudido porque trata de

duas políticas muito importantes política da educação e a política fiscal do dos

entes Federados particularmente os estados só que ele traz grandes desafios

distributivos e de homogeneização sob a Perspectiva da Educação ou sob a

perspectiva das da da política fiscal especificamente aqui equacionamento da

dívida dos Estados sobre a perspectiva da educação já foi amplamente eh debatido

temos eh eh diferenças de demanda escala população eh enormes entre os Estados

particularmente aí falando de Minas Gerais temos também um a questão de

Diagnóstico em 2021 nós iniciamos um ciclo de um programa específico de trilhas

de futuro exatamente sobre tratando do ensino médio profissionalizante que

dobrou o número de matrículas eh para para essa política pública e aí quer dizer

o estado de Minas Gerais em algum momento já iniciou essa política como é que

ficaria sob a perspectiva de incentivo eh para equacionamento da dívida esse

estágio desse ciclo do Estado de Minas Gerais é apenas um exemplo das

dificuldades de você é homogeneizar essa política pública de incentivo muito bem

queremos incentivar um ensino técnico profissionalizante em diversas situações

integral eh parcial eh eh eh é oferecido por eh estruturas municipais estaduais

que tem diversas soluções esquemas do soluções diferentes nos estados e

compatibilizar isso com a as questão da da da política fiscal dos Estados é

altamente complexo eu também concordo aqui com Cadu né com com o Felipe eh não

dá pra gente dividir o país em super individados e individados nós temos uma

perspectiva histórica da evolução Econômica de cada estado diferente a própria

União em diversas oportunidades já tratou a situação fiscal particularmente

dívida eh começando como por exemplo não começando Mas um grande evento foi a

9496 em 97 tivemos aí a lei de responsabilidade fiscal temos o a um marco

regulatório do regime de recuperação fiscal que todos tiam a a a a ideia né de

trazer uma situação de política fiscal saudáv prosent Claro as situações as

soluções os resultados foram distintos como foi trazido pelo secretário cão no

tesouro a dívida pública agregada do país melhorou né o perfil melhorou o que

significa que evoluímos na lei de responsabilidade fiscal evoluímos lá com a

9496 em 97 no entanto essas políticas não solucionaram todos os problemas

fiscais dos Estados temos aqui quatro Estados eh altamente eh eh populosos eh

que com situações graves de estoque e três com situações muito graves de fluxo

eh que é o estoque em relação ao a à receita corrente líquida desse estados

claro como Federação claro que como União nós temos que procurar soluções e

vamos vamos vamos ter soluções no longo prazo para essa situação a história tem

mostrado isso agora como foi trazido o tema não é não é simples ele é complexo a

forma de debate ela tá correta não a como trazer solução simples ou única para

problemas complexos ainda mais em duas áreas assim tão díspares Então vamos sim

participar da da discussão agora temos que chegar a um a um meio-termo que possa

induzir essa política pública de educação particularmente técnico né

profissional e ao mesmo tempo trazer uma solução para esses três quatro Estados

altamente endividados ou com forte comprom da sua situação financeira que tem aí

talvez 50% da população do país obrigado obrigada secretário eu queria começar

uma pergunta pro pro Felipe salto Felipe você diz que que você fez aí uma série

de críticas né ao programa Então queria saber como que o financiamento ao ensino

profissionalizante tá precisando de mais recursos fo muito colocado aqui durante

o evento como é que ele se encaixa nesse desenho aí que você tá dizendo de uma

solução eh individualizada para cada Estado na questão da dívida eu acho que nós

temos que separar as questões né o problema da renegociação da dívida não vai

resolver o problema da educação eu acho que para mim isso tá claro essa ideia

que foi vendida ela é muito atraente juros por educação juro é uma coisa péssima

educação é uma coisa ótima que todos queremos agora como é que eu troco juros

por educação a viúva de sempre vai pagar a conta que é a união porque juros por

educação alguém vai ganhar educação alguém vai deixar de ganhar receita de juro

quem vai deixar de ganhar receita de juro que não é uma receita como a gente

chama primária é uma receita financeira mas pouco importa porque a receita

financeira afeta també a evolução da dívida é a união e aí o que que vai

acontecer a união diz o seguinte tudo bem Eu tô disposto a abrir mão desse fluxo

financeiro por quê Porque eu acho que Abrindo mão desse fluxo financeiro estados

e vocês gastando esse fluxo que eu tô abrindo mão em educação de boa qualidade

no caso o ensino técnico né a o ept então Sim estamos acordados Agora falta

combinar com os russos no caso os russos são as nossas condições fiscais que

estão muito sérias o primeiro gráfico que eu pulei por conta do tempo né é a

trajetória da dívida a dívida pública hoje dívida bruta do governo geral que

inclui os estados os municípios as estatais a união e as operações

compromissadas do Banco Central porque o banco central também tem a sua parcela

na dívida é de 75% do pibe nós estamos crescendo muito pouco 2,5% a taxa real de

juros é 65% a 7% quanto é que precisa para equilibrar uma dívida de 75% do PIB

com uma taxa real de crescimento de 2,5 e um juro Real de 7% precisa de 35% do

pibe de superávit primário só que nós fizemos déficit de 2.2 no ano passado

tirar os 92,4 bi de precatório sobra 1.2 ainda de Déficit Então eu tenho que ir

de 1.2 de Déficit para 3,5 vai digamos três de superávit 4,2 pon percentual de

um pibe de 11 trilhões e meio vai R 500 bilhões de reais impossível E aí nós

vamos ainda estimular um gasto adicional não vai não dá né Esse é o problema

como é que resolve Então a questão da educação eu não sou especialista em

educação mas mas em contas públicas eu sou e essa proposta não para de pé eh a

gente tá aqui com o desafio do tempo mas eu queria ouvir qu uma bomba explodiu

não dá um medo daná nada mas eu eu queria ouvir do dos Senhores eh enfim muito

se falou no primeiro painel sobre tentativa de tirar da Justiça essa discussão

uma questão né Eh ao mesmo tempo esse tratamento enfim por isso um tratamento

igualitário com uma busca de incentivos né indução de uma política eh é é viável

que esse cenário se concretize ou inevitavelmente essa essa discussão acabará na

justiça eh a despeito do mérito eh louvável e e que e imagino que todos queiram

incentivar o ensino técnico tudo mais mas há um cenário de judicialização bem eu

vou dar E tudo passa pela política né Fernando eh e a gente tem transitado no

Congresso Nacional transitado lá na na no Ministério da Fazenda eh eu acho que a

ambiência política é para se buscar uma solução eu sinto isso eh a gente

conversou com com presidente do senado com vários atores eh e eu sinto que Pelo

menos é um feeling pessoal tá é um fing muito parecido com o que a gente teve

ano passado de que a reforma tributária iria iria prosperar eu acho que esse

tema vai ser endereçado sim talvez talvez não da forma como ele n nasceu até por

esses problemas que a gente tá colocando aqui tá mas eu acho que nesse ambiente

de debate isso aqui hoje ajuda muito esse debate aqui hoje ajuda muito eu acho

que vai ser endereçado porque eu concordo em gênero número e grau com a fala

acho que foi do do do de que não dá a gente precisa tirar isso do do da Justiça

né do STF não não vai ser por aí acho que o o É difícil Felipe querer que o

judiciário participe do nascedouro das ideias Eu acho que isso talvez não

aconteça mas eh eu sinto a ambiência respondendo objetivamente de que vai ser

endereçado talvez não dessa forma que tá sendo concebido com alguns ajustes para

resolver essas essas essa falta de foco em em algumas questões e principalmente

também resolver a questão dos estados que não são endividados secretário quer

complementar depois a gente vai passar pro Desafio Final de em um minuto fazer

as considerações finais clo não queria concordar com com o o presidente cadur né

do Conce Fas que realmente nós temos achar uma solução política que ela passa

pelo diálogo e a construção de um projeto viável so a perspectiva dos estados

que estão com essa situação fiscal né Eh nós como eu comentei resolvemos

historicamente n questões na nos últimos nas últimas quatro décadas e o caminho

do diálogo e ultimando ali na na casa né no Congresso Nacional é onde será

resolvido isso Obrigado Lu você quer passar a Palavra Final para Vilma já

fazendo uma provocação para que ela não passe EMC as perguntas tá bem eh viu eu

quero fazer uma pergunta E aí você já Aproveita faz as considerações finais tá

bom eh eu não quero te intrigar com o Felipe mas eu queria ouvir a sua opinião

sobre se o programa para em pé ou não Porque Você levantou um ponto aqui da do

aumento da produtividade também você tem outras impactos né de um programa como

esse na economia Então queria ouvir a sua ponderação sobre isso obrigada

Obrigada eh eu acho que de fato né como o Felipe colocou tem questões fiscais

que a gente precisa levar em em consideração e que não necessariamente eh São eh

aplicáveis eh para todos os estados quando a gente observa Principalmente as

necessidades eh da proposta do ponto de vista fiscal associado também as

necessidades de de ampliação do dos indicadores educacionais então eh eh até que

ponto a gente consegue É de fato compatibilizar essas duas eh questões né

questão Educacional e e a necessidade de ampliação de fortalecimento eh dos

indicadores e da governança eh Educacional com a questão da sustentabilidade das

contas dos governos estaduais eu acho que de fato a gente poderia pensar em

soluções eh distintas né e não num projeto em conjunto ou ainda que se seja

ainda que seja um projeto eh conjunto A gente pensar também eh eh em em em

medidas né em em em ações que vão de encontro a esses estados que não vão ser

também eh abarcados por essa proposta ou tenham necessidades diferentes do ponto

de vista de alcance dos indicadores educacionais Então acho que a solução ela é

um pouco mais complexa do que eh a gente a gente tem observado no no na proposta

em si e a gente precisa endereçar nesse nesse aspecto né tentando compatibilizar

as diferenças regionais nesse sentido eh eu acho que também é importante a gente

avançar também em estudos de impacto eh e aí de fato observar não só o impacto

fiscal como o Felipe trouxe os números mas também os impactos em educação Qual é

o potencial ganho eh da proposta do ponto de vista educacional eh Quais são os

custos Associados Quais são os benefícios quais são os impactos em produtividade

por exemplo e outros indicadores eh não vou me alongar muito eu gostaria de novo

de parabenizar o evento Agradecer o convite e é isso obrigada obrigado Vilma

Felipe outra coisa que não para em pé saco vazio então assim por favor

considerações finais sucintas Obrigado a todos pelo convite bom almoço a todos

eu também queria agradecer ao convite e ao debate Obrigado pessoal foi um grande

prazer participar aqui com vocês obgada muito obrigada pela participação Vilma

Felipe secretários obrigada a Lu e Fernando pela mediação e pelas contribuições

Muito obrigado uma ótima tarde pedi pra gentileza de vocês se juntarem para

fazer uma foto antes da gente encerrar muito obrigada muito obrigada obrigada

bom então com este último painel fechamos a programação deste primeiro fórum

Valor Econômico o seminário juros por educação chega ao final com a sensação de

dever cumprido graças ao envolvimento e a dedicação de todos que participaram

trazendo Dados importantes soluções inovadoras e muita vontade de realizar este

encontro será um Marco no fortalecimento da educação como motor de crescimento

inclusão social em nosso país em nome do valor econômico Agradeço aos nossos

parceiros Itaú educação e trabalho e Todos Pela Educação e a todos vocês muito

obrigada pela presença pela audiência por nos prestigiar e apoiar juntos sempre

podemos fazer mais e melhor uma ótima tarde e até a próxima

\end{document}
