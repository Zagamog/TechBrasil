\documentclass[a4paper,10pt]{article}
\usepackage[utf8]{inputenc}
\usepackage[left=0.5in, right=0.5in, top=0.8in, bottom=0.8in]{geometry}
\usepackage{enumitem}

\setlist[itemize]{topsep=0pt, partopsep=0pt, parsep=0pt} 

\begin{document}

\section*{Painel 1: A estratégia do programa 'Juros pela Educação'}

\begin{table}[htbp!]
\centering
\renewcommand{\arraystretch}{1.2}
\begin{tabular}{|p{1in}|p{1.4in}|p{4in}|}
\hline
Name & Assignation & Summary Bullets \\
\hline
Ana Inoue & Itaú Educação e Trabalho & \begin{itemize}
\item Explained how the program aligns with workforce development strategies.
\item Stressed the importance of vocational training and lifelong learning.
\item Outlined potential partnerships between financial institutions and education providers.
\end{itemize}\\
\hline
Priscila Cruz & Todos Pela Educação & \begin{itemize}
\item Advocated for equitable access to educational financing.
\item Addressed challenges in implementing the program at a national scale.
\item Called for data-driven policies to measure educational impact.
\end{itemize}\\
\hline
Camilo Santana & Ministro da Educação & \begin{itemize}
\item Presented the government’s commitment to funding and expanding the initiative.
\item Discussed federal policies supporting financial aid for students, emphasizing new financing mechanisms.
\item Highlighted the role of state-federal partnerships to ensure effective fund allocation and program sustainability.
\item Addressed the need for improving administrative efficiency in distributing educational resources.
\item Outlined the importance of equity-focused approaches to reach underprivileged communities.
\end{itemize}\\
\hline
João Azevedo & Governador do Estado da Paraíba & \begin{itemize}
\item Shared state-level experiences in implementing similar financial programs, detailing Paraíba’s initiatives.
\item Explored best practices from other states to address educational inequality through targeted funding.
\item Discussed regional disparities in education funding and proposed solutions tailored to specific socioeconomic contexts.
\item Emphasized the role of local governments in ensuring educational investments yield long-term benefits.
\item Highlighted successful case studies from Paraíba’s education system as models for national implementation.
\end{itemize}\\
\hline
Dário Durigan & Secretário Executivo do Ministério da Fazenda & \begin{itemize}
\item Explained the fiscal strategies underpinning the program’s funding, ensuring long-term financial viability.
\item Discussed budgetary allocations and financial sustainability concerns, considering economic constraints.
\item Outlined potential tax incentives for institutions participating in the program to encourage investment.
\item Proposed mechanisms for monitoring and evaluating the financial performance of the initiative.
\item Highlighted the importance of private sector involvement in supplementing public educational investments.
\end{itemize}\\
\hline
\end{tabular}
\end{table}

\end{document}
