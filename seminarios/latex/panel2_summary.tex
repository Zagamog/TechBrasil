\documentclass[a4paper,10pt]{article}
\usepackage[utf8]{inputenc}
\usepackage[left=0.5in, right=0.5in, top=0.8in, bottom=0.8in]{geometry}
\usepackage{enumitem}

\setlist[itemize]{topsep=0pt, partopsep=0pt, parsep=0pt} 

\begin{document}

\section*{Painel 2: O que é a proposta 'Juros pela Educação', ajustes e detalhes técnicos}

\begin{table}[htbp!]
\centering
\renewcommand{\arraystretch}{1.2}
\begin{tabular}{|p{1in}|p{1.4in}|p{4.2in}|}
\hline
Name & Assignation & Summary Bullets \\
\hline
Murilo Camaroto & Repórter do Valor Econômico & \begin{itemize}
\item Asked about the most significant adjustments made to the 'Juros pela Educação' program in recent weeks.
\item Inquired about the technical feasibility and long-term sustainability of the proposed financing mechanism.
\end{itemize}\\
\hline
Fernando Exman & Chefe da Sucursal do Valor Econômico em Brasília & \begin{itemize}
\item Questioned how the program aligns with broader economic policies and fiscal constraints.
\item Probed into the expected impact of the initiative on private sector engagement in education funding.
\end{itemize}\\
\hline
Gregório Grisa & Secretário Executivo-Adjunto do Ministério da Educação & \begin{itemize}
\item Explained the rationale behind recent adjustments to the 'Juros pela Educação' program, emphasizing the balance between ensuring broad access to financing and maintaining financial sustainability.
\item Discussed mechanisms to ensure fair access to financing, particularly for low-income students and underfunded institutions, including interest rate reductions and targeted financial aid programs.
\item Highlighted how the program integrates with existing federal education policies, aiming to create a long-term strategy that supports both student success and national economic growth.
\item Addressed concerns about administrative bottlenecks, outlining steps being taken to streamline application and fund distribution processes, ensuring efficiency and transparency.
\item Emphasized the importance of a continuous evaluation framework, where data analytics and impact assessments will be used to refine and improve the financing model over time.
\end{itemize}\\
\hline
Rogério Ceron & Secretário do Tesouro Nacional do Ministério da Fazenda & \begin{itemize}
\item Outlined the Treasury’s approach to ensuring long-term program funding while maintaining fiscal responsibility, detailing how the initiative aligns with broader government budget priorities.
\item Explained the mechanisms through which the government will subsidize interest rates without compromising macroeconomic stability, ensuring predictability in public accounts.
\item Discussed the financial safeguards in place to prevent excessive fiscal exposure, including expenditure ceilings and mechanisms for adjusting program parameters based on economic conditions.
\item Clarified the role of the National Treasury in coordinating fund distribution with public financial institutions, ensuring compliance with constitutional spending limits.
\item Highlighted the importance of accountability and transparency in managing public funds for education, detailing how periodic audits and reporting mechanisms will ensure efficient resource allocation.
\end{itemize}\\
\hline
\end{tabular}
\end{table}

\end{document}
